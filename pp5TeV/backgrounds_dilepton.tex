\subsection{The dilepton final state}
\label{sec:bkg_dil}

Final states with two genuine leptons can originate from background processes,
primarily from Z/$\gamma^{*} \to \tau^+\tau^-$ (where the $\tau$ leptonic decays can yield $\empm$ or $\mmpm$ plus \ptmiss due to the neutrinos), \tW, and \WV events. Other background 
sources, such as W+jets events or \ttbar\ production in the $\ell$+jets final state, can 
contaminate the signal sample if a jet is misidentified as a lepton, or if an event contains a
lepton from the decay of b or c hadrons. These are included in the ``\nonW''  category since genuine leptons are defined as originating from decays of W or Z bosons.  
The yields
from \tW and \WV events are estimated from simulation, while the contribution of the Z/$\gamma^{*}$ background is evaluated using control samples in data.
The rate of \nonW backgrounds is extracted from control samples in data for the \empm final state and is estimated from the simulation in the \mmpm final state.

A scale factor for the Z/$\gamma^{*}$ background normalization is estimated from the number of events within the Z boson mass window in data, which is extrapolated to
the number of events outside the window. 
This ratio, ${R_{\textrm{out/in}}}$, is estimated from MC simulation, and the contamination from processes other than the Z/$\gamma^{*}$ contribution 
in the \cPZ{} boson mass window is considered to be negligible. 
The number of extrapolated events can be inferred from data as
\begin{eqnarray}
N^{\ell^{\pm}\ell^{\mp},\,\textrm{obs}}_{\textrm{out}} = R^{\ell^{\pm}\ell^{\mp}}_{\textrm{out/in}}( N^{\ell^{\pm}\ell^{\mp}}_{\textrm{in}} -0.5 N^{\mathrm{e}^{\pm}\mu^{\mp}}_{\textrm{in}} k_{\ell^{\pm}\ell'^{\mp}})\, .
\label{eq:dyest}
\end{eqnarray}
$R_{\textrm{out/in}}$ is the ratio of the number of events outside (``out'') and inside (``in'') the \cPZ{} mass region, estimated from the Z/$\gamma^{*}$ MC simulation event samples (Table~\ref{tab:mc_5TeV}) 
\begin{equation}
R_{\textrm{out/in}}= \frac{N^{\textrm{out}}_{\mathrm{Z}/\gamma^{*},\,\textrm{MC}}}{N^{in}_{\mathrm{Z}/\gamma^{*},\,\textrm{MC}} }\, ,
\label{eq:routin}
\end{equation}
with $k$ a corrector factor that must be applied to take into account the differences between electron and muon reconstruction. 
It is calculated using the events in the  \cPZ{} mass region that fulfill the standard dilepton selection, and can be written as
\begin{equation}
k_{\ell^{\pm}\ell^{\mp}} = \sqrt{\frac{N^{\ell^{\pm}\ell^{\mp}}_{\textrm{in}}}{ N^{\ell'^{\pm}\ell'^{\mp}}_{\textrm{in}} }}\, ,
\label{eq:kll}
\end{equation}
with $\ell=e$ and $\ell'=\mu$ or $\ell=\mu$ and $\ell'=e$. 

Tables~\ref{tab:dy} and~\ref{tab:dymumu} summarize the input obtained from data and simulation at different selection levels in the \empm\ and \mmpm\ final states, respectively.
The data-based estimation is then compared to the MC simulation estimation to derive scale factors. In the \empm final state, the scale factor is defined as
\begin{equation}
\textrm{SF}_{\empm} = \sqrt{\textrm{SF}_{\empe}  \textrm{SF}_{\mmpm}}\, .
\label{eq:sfemu}
\end{equation} 
A scale factor of $0.96 \pm 0.78$\,(stat) is obtained in the \mmpm final state, and $0.91 \pm 0.14$\,(stat)  in the \empm final state. 
The estimation is performed using events with at least two jets, and the dependence on different jet multiplicities is discussed in Section~\ref{sec:syst}.

\begin{table}[h] 
\caption[Z/$\gamma^{*}$ background estimation in the \empm\ final state at $\sqrt{s}=5.02$\,\TeV]{
Z/$\gamma^{*}$ background estimation in the \empm\ final state for events with at least two reconstructed jets.
Uncertainties are of statistical nature. The breakdown of the results obtained for the different dilepton final states is given~\citeAN{AN-16-098,AN-16-369}.} \label{tab:dy}
\centering 
\begin{tabular}{l||c|c|c} 
\toprule
                   &   \empe                &         \mmpm    &  \empm \\
\midrule 
 $N_{\textrm{in}}$ (MC)     &   226.0 $\pm$   5.5 &   250.3 $\pm$   5.6 &                   \\
 $N_{\textrm{out}}$ (MC)    &    18.2 $\pm$   2.4 &    32.9 $\pm$   3.0 &                   \\
 $R_{\textrm{out}/\textrm{in}}$(MC)  &   0.081 $\pm$ 0.013 &   0.132 $\pm$ 0.015 &                   \\
 $k_{\ell^{\pm}\ell^{\mp}}$    &   0.950 $\pm$ 0.022 &   1.052 $\pm$ 0.025 &                   \\
 $N_{\textrm{in}}$ (Obs)      &     212 $\pm$  14.6 &     226 $\pm$  15.0 &   5.0 $\pm$  2.2  \\
\midrule
 $N_{\textrm{out}}$         &    16.9 $\pm$   0.8 &    29.4 $\pm$   1.0 &                   \\
\midrule
 SF (Obs/MC)         &    0.93 $\pm$  0.17 &    0.89 $\pm$  0.11 &  0.91 $\pm$ 0.14  \\
\midrule
Z/$\gamma^{*}$ (MC)    &    18.6 $\pm$   2.3 &    33.6 $\pm$   3.1 &   1.8 $\pm$ 0.4   \\
Z/$\gamma^{*}$ (Obs)     &    17.3 $\pm$   2.1 &    30.0 $\pm$   2.8 &   1.6 $\pm$ 0.4   \\
\bottomrule
\end{tabular}
\end{table}


\begin{table}[h] 
\caption[Z/$\gamma^{*}$ background estimation in the \mmpm\ final state at $\sqrt{s}=5.02$\,\TeV]{
Z/$\gamma^{*}$ background estimation in the \mmpm\ final state for different levels in the event selection. Uncertainties are of statistical nature~\citeAN{AN-16-098,AN-16-369}.} \label{tab:dymumu}
\centering 
\begin{tabular}{l||c|c|c} 
\toprule
                   &    \mmpm & $\ptmiss>35$\,\GeV & $\geq$ 2 jets   \\
\midrule 
 $N_{\textrm{in}}$ (\mmpm,\,MC)  & 6955.2   $\pm$ 21.4   &   7.3 $\pm$ 1.0   & 4.2   $\pm$ 0.8   \\
 $N_{\textrm{out}}$ (\mmpm,\,MC) &  799.7   $\pm$ 12.0   &   3.0 $\pm$ 0.7   & 1.2   $\pm$ 0.5   \\
 $R_{\textrm{out}/\textrm{in}}$(MC)       &    0.115 $\pm$  0.002 & 0.407 $\pm$ 0.151 & 0.300 $\pm$ 0.173 \\
 $N_{\textrm{in}}$ (\empe,\,MC)      & 6408.3   $\pm$ 21.2   & 7.5   $\pm$ 1.0   & 4.1   $\pm$ 0.7   \\
 $k_{\ell^{\pm}\ell^{\mp}}$         &    1.042 $\pm$  0.003 & 0.985 $\pm$ 0.129 & 1.010 $\pm$ 0.185 \\
 $N_{\textrm{in}}$ (\mmpm,\,Obs)   & 6609     $\pm$ 81.3   &    10 $\pm$ 3.2   & 6.0   $\pm$ 2.4   \\
 $N_{\textrm{in}}$ (\empm,\,Obs)     &   12.0   $\pm$  3.5   & 5.0   $\pm$ 2.2   & 4.0   $\pm$ 2.0   \\
\midrule
 $N_{\textrm{out}}$              &  763.9   $\pm$  2.2   & 3.1   $\pm$ 0.7   & 1.2   $\pm$ 0.5  \\
\midrule
 SF (Obs/MC)              &    0.949 $\pm$  0.017 & 1.036 $\pm$ 0.481 & 0.957 $\pm$ 0.788  \\
\bottomrule
\end{tabular}
\end{table}


The \nonW background in the \empm final state is estimated using an extrapolation from a control region of same-sign ($\mathrm{SS}$) dilepton
events to the signal region of opposite-sign ($\mathrm{OS}$) lepton pairs. 
The $\mathrm{SS}$ control region is defined using the same criteria as for the
nominal signal region, except requiring dilepton pairs of the same
charge. The muon isolation requirement is relaxed in order to enhance the number of events.
The $\mathrm{SS}$ dilepton events predominantly contain at least one
misidentified lepton.
Other SM processes produce genuine $\mathrm{SS}$ or charge-misidentified dilepton events
with significantly smaller rates; these are estimated using simulation and subtracted from the
observed number of events in data.

The scaling from the $\mathrm{SS}$ control to the signal region in data is 
performed using an extrapolation factor extracted from MC simulation,
given by the ratio of the number of $\mathrm{OS}$ events with misidentified
leptons to the number of $\mathrm{SS}$ events with misidentified (``misID'') leptons. 
Taking these into account, the fake lepton contribution $N_{\textrm{data}}^{\mathrm{OS}\,\textrm{fakes}}$ to the measurement is estimated as
\begin{equation}
N_{\textrm{data}}^{\mathrm{OS}\,\textrm{fakes}} = \textrm{SF}_{\textrm{fakes}}  N_{\textrm{MC}}^{\mathrm{OS}\,\textrm{fakes}}\, ,
\label{eq:fakeLeptons}
\end{equation}
where $\textrm{SF}_{\textrm{fakes}}$ is defined to be
\begin{equation}
\textrm{SF}_{\textrm{fakes}} = \frac{N_{\textrm{data}}^{\mathrm{SS}}-N_{\textrm{misID}}^{\mathrm{SS}}-N_{\textrm{rare}}^{\mathrm{SS}}}{N_{\textrm{MC}}^{\mathrm{SS}\,\textrm{fakes}}}
= \frac{N_{\textrm{data}}^{\mathrm{'\,SS}}}{N_{\textrm{MC}}^{\mathrm{SS}\,\textrm{fakes}}}
\label{eq:fakessf}
\end{equation}
The contribution from \ttbar, Z/$\gamma^{*}$ and tW processes are subtracted from $N_{\textrm{data}}^{\mathrm{SS}}$, giving rise to the numerator ($N_{\textrm{data}}^{\mathrm{'\,SS}}$) on the right-hand side 
of Eq.\,(\ref{eq:fakeLeptons}). 
The resulting estimate for the non-W/Z background is thus $1.0 \pm 0.9$\,(stat) events in the \empm final state, where the central value comes from the estimation using events 
with at least two reconstructed jets. 
No particular dependence of this scale factor is observed for different jet multiplicities within the sizeable statistical uncertainty.


\begin{table}[!htp] 
\caption[The \nonW background estimation in the \empm\ final state at $\sqrt{s}=5.02$\,\TeV]{
The \nonW background estimation in the \empm\ final state using events with at least two reconstructed jets.
The uncertainties are of statistical nature~\citeAN{AN-16-098,AN-16-369}.} 
\label{tab:nonW}
\centering 
\begin{tabular}{l|c} 
\toprule
          Source    & $N_{\textrm{jets}}\geq 2$ \\
\midrule
Prompt $\mathrm{SS}$ (MC)      & 1.4  $\pm$ 0.2 \\
Fake $\mathrm{SS}$ (MC)        & 5.2  $\pm$ 0.7 \\
Fake $\mathrm{OS}$ (MC)        & 0.5  $\pm$ 0.3 \\
R = fake $\mathrm{OS}$/fake $\mathrm{SS}$ & 0.10 $\pm$  0.08 \\
\midrule
$N_{\textrm{data}}^{\mathrm{SS}}$  &  11   \\
\midrule
$N_{\textrm{data}}^{\mathrm{OS}}$  & 1.0 $\pm$ 0.9\\
\bottomrule
\end{tabular}
\end{table}