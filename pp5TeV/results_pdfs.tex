\section{\QCD\ analysis}
\label{sec:pdfs}

To illustrate the impact of the \stt measurements at $\sqrts = 5.02$\,\TeV on the knowledge of the proton PDFs the results are used in a \QCD\ analysis at 
NNLO, together with the combined measurements of neutral- and charged-current cross sections for deep inelastic electron- and positron-proton 
scattering (DIS) at \texttt{HERA}~\cite{Abramowicz:2015mha}, and the \texttt{CMS} measurement~\cite{Khachatryan:2016pev} of the muon charge asymmetry in W boson 
production at $\sqrt{s}=8$\,\TeV. 
The precise \texttt{HERA} DIS data are directly sensitive to the valence and sea quark distributions and probe the gluon distribution through scaling violations. 
The latter data set is used to check the feasibility of improving the constraints on the light-quark distributions. 

Version 2.0.0 of \textsc{xFitter}~\cite{Alekhin:2014irh, herafitter}, an open-source \QCD-analysis framework for PDF 
determination, is employed, with the partons evolved using the Dokshitzer--Gribov--Lipatov--Altarelli--Parisi 
equations~\cite{Gribov:1972ri,Altarelli:1977zs,Curci:1980uw,Furmanski:1980cm,Moch:2004pa,Vogt:2004mw} 
at NNLO, as implemented in the {$\mathrm{QCDNUM 17-01/13}$} program~\cite{Botje:2010ay}. The treatment and the choices for the central values 
and variations of the c and b quark masses, the strong coupling, and the strange-quark content fraction of the proton follow that of 
earlier \texttt{CMS} analyses, e.g., Ref.~\cite{Khachatryan:2016pev}. The $\mu_\textrm{R},\mu_\textrm{F}$ scales are set to the four-momentum transfer 
in the case of the DIS data (restricted to $\mathrm{Q}^2_{\textrm{min}}>3.5$\,\GeV$^2$), the W boson mass for the muon charge asymmetry results, and the top quark mass in the case of \stt.

The systematic uncertainty in all three measurements of \stt and their correlations are treated the same way as in 
the combination described in Section~\ref{sec:comb}. The theoretical predictions for \stt are obtained at NNLO using the 
\textsc{Hathor} (v2.1) calculation~\cite{Aliev:2010zk}, assuming $m_{\textrm{top}}= 172.5$\,\GeV.
The bin-to-bin correlations of the experimental uncertainty in the muon charge asymmetry and DIS 
measurements are taken into account. The theoretical predictions for the muon charge asymmetry are obtained as 
described in Ref.~\cite{Khachatryan:2016pev}. 

The procedure for the determination of the PDFs follows the approach of \HERAPDF as used in the \QCD\ analysis of Ref.~\cite{Khachatryan:2016pev}. 
 The parametrized PDFs are the gluon distribution, $x\Pg$, the valence quark distributions, $x\cPqu_v$, $x\cPqd_v$, and 
the $\cPqu$-type and $\cPqd$-type antiquark distributions, $x\overline{U}$, $x\overline{D}$. The relations $x\overline{U} = x\cPaqu$ 
and $x\overline{D} = x\cPaqd + x\cPaqs$ are assumed at the initial scale of the \QCD\ evolution $\mathrm{Q}_0^2$ = 1.9\,\GeV$^2$. At this scale, the parametrizations are of the form: 
\begin{align}
\begin{split}
%\begin{eqnarray}
x\Pg(x) &= A_\Pg x^{B_\Pg}\times(1-x)^{C_{\Pg}}\times (1+D_\Pg x)\ ,
%\label{eq:g}\\
\\
x\cPqu_v(x) &= A_{\cPqu_v}x^{B_{\cPqu_v}}\times(1-x)^{C_{\cPqu_v}}\times(1+D_{\cPqu_v}x+E_{\cPqu_v}x^2)\ ,
%\label{eq:uv}\\
\\
x\cPqd_v(x) &= A_{\cPqd_v}x^{B_{\cPqd_v}}\times(1-x)^{C_{\cPqd_v}}\ ,
%\label{eq:dv}\\\
\\
x\overline{U}(x)&= A_{\overline{U}}x^{B_{\overline{U}}}\times (1-x)^{C_{\overline{U}}}\times (1+E_{\overline{U}}x^2)\ ,
%\label{eq:Ubar}\\
\\
x\overline{D}(x)&= A_{\overline{D}}x^{B_{\overline{D}}}\times (1-x)^{C_{\overline{D}}}\ .
%\label{eq:Dbar}
%\end{eqnarray}
\end{split}
\label{eq:PDF_param}
\end{align}

The normalization parameters $A_{\cPqu_v}$, $A_{\cPqd_v}$, and $A_\cPg$ are determined by the \QCD\ sum 
rules, the $B$ parameters are responsible for the small-$x$ behavior of the PDFs, and the $C$ parameters describe the shape of 
the distribution as $x \to 1$. Additional constraints $B_{\overline{U}} = B_{\overline{D}}$ and 
$A_{\overline{U}} = A_{\overline{D}}(1 - f_\cPqs)$ are imposed, with $f_\cPqs$ being the strangeness 
fraction, $\cPaqs/( \cPaqd + \cPaqs)$, which is set to $0.31\pm0.08$ as in Ref.~\cite{Martin:2009ad}, 
consistent with the value obtained using the \texttt{CMS} measurements of W+c production~\cite{Chatrchyan:2013mza}.  

\begin{table}[!tb]
\centering
\renewcommand{\arraystretch}{1.25}
\normalsize
\caption[The quality ($\chi^2$ goodness of fit) of the overall \QCD\ analysis including the \stt measurements at $\sqrt{s}=5.02$\,\TeV]{
Partial $\chi^2$ per number of data points, $n_{\textrm{dp}}$, and the global $\chi^2$ per 
degrees of freedom, $n_{\text{dof}}$, as obtained in the \QCD\ analysis of DIS data, the \texttt{CMS} muon charge asymmetry measurements, and the \stt results at $\sqrts = 5.02$\,\TeV\ from this analysis. 
For the \texttt{HERA} measurements, the energy of the proton beam ($E_{\Pp}$) is listed for each data set, with the electron/positron energy of $27.5\,\mathrm{GeV}$. 
The correlated part of the global $\chi^2$ value is also given. 
The sources of correlated systematic uncertainty in the \stt measurement are treated as nuisance parameters; for each parameter, a penalty term is added to the $\chi^2$~\citeTH{topobs_CMS_jhep}.
}

\begin{tabular}[!ht]{l | c}
\toprule
Data sets   & Partial $\chi^2/n_{\textrm{dp}}$ \\
\midrule
\texttt{HERA} neutral current,  $\Pep \Pp$,  $E_{\Pp}=920\,\textrm{GeV}$ &  $449/377$  \\
\texttt{HERA} neutral current,  $\Pep \Pp$,  $E_{\Pp}=820\,\textrm{GeV}$ &  $71/70$ \\
\texttt{HERA} neutral current,  $\Pep \Pp$,  $E_{\Pp}=575\,\textrm{GeV}$ &  $224/254$ \\
\texttt{HERA} neutral current,  $\Pep \Pp$,  $E_{\Pp}=460\,\textrm{GeV}$ &  $218/204$ \\
\texttt{HERA} neutral current,  $\Pem \Pp$,  $E_{\Pp}=920\,\textrm{GeV}$ &  $218/159$ \\
\texttt{HERA} charged current,  $\Pep \Pp$,  $E_{\Pp}=920\,\textrm{GeV}$ &  $43/39 $ \\
\texttt{HERA} charged current,  $\Pem \Pp$,  $E_{\Pp}=920\,\textrm{GeV}$ &  $53/42 $ \\
\texttt{CMS} $\mathrm{W}^\pm$ muon charge asymmetry  &  $2.4/11$ \\
\texttt{CMS} \stt, \empm, 5.02\,\TeV & $1.03 /1$ \\
\texttt{CMS} \stt, \mmpm, 5.02\,\TeV & $0.01 /1$ \\
\texttt{CMS} \stt, $\ell$+jets, 5.02\,\TeV & $0.70 /1$ \\
\midrule
Correlated $\chi^2$                                 & $100$ \\
Global $\chi^2/n_{\text{dof}}$                         & $1387/1145$\\
\bottomrule
\end{tabular}

\label{chi2_pdffit_table}
\end{table}

The predicted and measured cross sections for all the data sets, together with their corresponding uncertainties, are used to build a global $\chi^2$, 
minimized to determine the PDF parameters~\cite{Alekhin:2014irh, herafitter}. 
The parameters are selected by first fitting with all $D$ and $E$ parameters set to zero, 
and then including them independently one at a time in the fit. The improvement in the global $\chi^2$ of the fit is monitored, 
and the procedure is terminated when no further improvement is found.
Using the measured values for \stt allows the addition of a new free parameter, $D_{\cPqu_v}$, in Eq.\,(\ref{eq:PDF_param}), as compared to 
the analysis in Ref.~\cite{Khachatryan:2016pev}, leading to a 14-parameter fit. The results of the fit are given in Table~\ref{chi2_pdffit_table}.

The quality of the overall fit can be judged based on the global $\chi^2$ divided by the number of degrees of freedom, $n_{\textrm{dof}}$. 
For each data set included in the fit, the partial $\chi^2$ divided by the 
number of the measurements (data points), $n_{\textrm{dp}}$, is also provided. The correlated part of $\chi^2$, also given in Table~\ref{chi2_pdffit_table}, 
quantifies the influence of the correlated systematic uncertainty in the fit. 
The global and partial $\chi^2$ values indicate a general agreement among all the data sets.
The low $\chi^2/n_{\textrm{dp}}$ value in the \mmpm\ final state reflects the large uncertainty in the measurement.
The somewhat high $\chi^2/n_{\textrm{dp}}$ values for the combined DIS data are very similar to those observed in 
Ref.~\cite{Abramowicz:2015mha}, where they are investigated in detail. 

\begin{figure}[!ht]
\center
   \includegraphics[width=0.6\textwidth]{figures/pp5TeV/gluon_pdf.pdf}
\setlength{\unitlength}{1cm}
\caption[The uncertainty in the gluon PDF of the proton as a function of $x$ at $\mu^2_{\textrm{F}}=10^5\,\mathrm{GeV}^2$ including \stt data at $\sqrt{s}=5.02$\,\TeV]{
The relative uncertainty in the gluon distribution function of the proton as a function of $x$ at $\mu^2_{\textrm{F}}=10^5\,\mathrm{GeV}^2$ from a 
\QCD\ analysis using the \texttt{HERA} DIS and \texttt{CMS} muon charge asymmetry measurements (hatched area), and also including the \texttt{CMS} \stt results 
at $\sqrts = 5.02$\,\TeV\ (solid area). 
The relative uncertainty is found after the two gluon distributions have been normalized to unity.
The solid line shows the ratio of the gluon distribution function found from the fit with the \texttt{CMS} \stt measurements included to that found without~\citeTH{topobs_CMS_jhep}.}
\label{PDFs_gluon}
\end{figure}


The experimental uncertainty in the measurements are propagated to the extracted \QCD\ fit parameters using 
the MC method~\cite{Giele:1998gw, Giele:2001mr}. In this method, 400 replicas of pseudo-data are 
generated with the measured values for \stt allowed to vary within the statistical and systematic uncertainties, and taking into account their correlations (Table~\ref{tab:inputs}). 
For each replica, the PDF fit is performed, and the uncertainty is estimated as the RMS around the central value. 
In Fig.~\ref{PDFs_gluon}, the ratio and the relative uncertainty in the gluon distributions, as obtained in the 
\QCD\ analyses with and without the measured values for \stt at $\mu^2_{\textrm{F}}=10^5\,\mathrm{GeV}^2$, are shown. A moderate reduction of the uncertainty in the 
gluon distribution at $x \gtrsim 0.1$ is observed, once the measured values for \stt are included in the fit. 
The uncertainty in the valence quark distributions remains unaffected (Fig.~\ref{fig:PDFs_light_and_gluon}).
All changes in the central values of the PDFs are well within the fit uncertainty.
The latter is determined using the tolerance criterion of $\Delta\chi^2=1$.

\begin{figure}[!ht]
\centering
\subfloat[][]{\label{PDFs_light_and_gluon:a}\includegraphics[width=0.32\textwidth]{figures/pp5TeV/mc_6_2}}
\subfloat[][]{\label{PDFs_light_and_gluon:b}\includegraphics[width=0.32\textwidth]{figures/pp5TeV/mc_7_2}}
\subfloat[][]{\label{PDFs_light_and_gluon:c}\includegraphics[width=0.32\textwidth]{figures/pp5TeV/mc_1_2}}\\
\subfloat[][]{\label{PDFs_light_and_gluon:d}\includegraphics[width=0.32\textwidth]{figures/pp5TeV/mc_6_4}}
\subfloat[][]{\label{PDFs_light_and_gluon:e}\includegraphics[width=0.32\textwidth]{figures/pp5TeV/mc_7_4}}
\subfloat[][]{\label{PDFs_light_and_gluon:f}\includegraphics[width=0.32\textwidth]{figures/pp5TeV/mc_1_4_prelim}}
\caption[Central values and uncertainties in the gluon and valence quark PDFs of the proton as a function of $x$ at $\mu^2_{\textrm{F}}=10^2$ and $10^5\,\mathrm{GeV}^2$ including \stt data at $\sqrt{s}=5.02$\,\TeV]{
  The valence quark (a,\,b,\,d,\,e) and gluon (c,\,f) distribution functions of the proton and their relative uncertainties (bottom panels)
  as a function of $x$ at $\mu^2_{\textrm{F}}=10^2$ and $10^5\,\mathrm{GeV}^2$ from a 
  \QCD\ analysis using the \texttt{HERA} DIS and \texttt{CMS} muon charge asymmetry measurements (hatched area), and also including the \texttt{CMS} \stt results at $\sqrts = 5.02$\,\TeV\ (solid area)~\citeAN{AN-16-358}. 
}
\label{fig:PDFs_light_and_gluon}
\end{figure}


Possible effects from varying the model input parameters and the initial PDF parametrization are 
investigated in the same way as in the similar analysis of Ref.~\cite{Khachatryan:2016pev}.
The modeling uncertainty arises from the variations in the values assumed for the c quark mass, the strangeness fraction, and the value of $\mathrm{Q}^2_{\textrm{min}}$ imposed on the \texttt{HERA} data. 
The parametrization uncertainty is estimated by varying the functional form of the PDFs with the parameters $D$ and $E$ added or removed one at a time, and the value of $\mathrm{Q}_0^2$. 
The two cases when the measured values for \stt are included or excluded from the fit are considered, resulting in the same associated model 
and parametrization uncertainties.

In conclusion, the \stt measurements at $\sqrts = 5.02$\,\TeV\ provide improved uncertainty in the gluon PDF at high $x$, though the impact is 
small, owing to the large experimental uncertainty.

