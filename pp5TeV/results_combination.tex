\subsection{Combination of the $\ell$+jets and dilepton final states}
\label{sec:comb}


The three individual \stt measurements are combined using the \texttt{BLUE} method~\cite{Lyons:1988rp,Valassi:2013bga} to determine the overall \ttbar\ cross section.
All sources of systematic uncertainty are considered as fully correlated across all final states, with the following exceptions:
the uncertainty associated with the finite event size of the simulated samples is taken as uncorrelated; 
the electron identification is not relevant for the  \mmpm final state; 
and the b tagging and \QCD\ multijet background uncertainties are only considered for the $\ell$+jets final state. 
In the $\ell+$jets final state, the WV and $\mathrm{Z}/\gamma^*$ backgrounds are not considered separately but as part of the ``Other background'' component, which is dominated by tW events. 
The uncertainty associated with this category is therefore treated as fully correlated with the tW uncertainty in the dileptonic final states and uncorrelated with the WV and $\mathrm{Z}/\gamma^*$ uncertainties.
The individual results used as input to the combination are summarized in Table~\ref{tab:inputs}, including the correlations assumed between the individual sources of uncertainty.


\begin{table*}[!ht]
\begin{center}
\caption[Inputs to the combination of the $\ell$+jets and dilepton final states, and the \QCD\ analysis, at $\sqrt{s}=5.02$\,\TeV]{
Inputs to the combination of the $\ell$+jets and dilepton final states, and the \QCD\ analysis (see Section~\ref{sec:pdfs}). 
The relative importance of statistical and systematic uncertainties in the combined result is attained by implementing the BLUE algorithm with all sources of systematic uncertainty set to zero; 
the impact of systematic variations is then calculated by quadratically subtracting the statistical and luminosity from the total uncertainty. 
The post-fit correlation matrix of the nuisance parameters in the $\ell$+jets analysis indicated correlations between the $\mu_\textrm{R},\mu_\textrm{F}$ scales in the W+jets background, and 
between JES uncertainty in signal and the  $\mu_\textrm{R}$ scale in the W+jets background. Additional cross-checks to evaluate whether the assumed correlations significantly 
affected the outcome of the combination thus involved setting to zero the correlation for the W+jets and JES uncertainty across the three final states, once at a time, 
while preserving all other correlations~\citeAN{AN-16-358}.
\label{tab:inputs}}
\begin{tabular}{lcccc}
\toprule
Final state & \empm &  \mmpm & $\ell$+jets &  \\
\midrule
Central value (pb) & 76.5 & 59.2 & 68.9 & \\
\midrule
Sources of uncertainties (\%) &  &  &  & Correlation \\
\midrule
b tagging efficiency             & -     & -    & 3.4   & - \\
Electron efficiency           & 1.4   & -    & 1.1   & 1 \\
Muon efficiency               & 3.0   & 6.1  & 1.7   & 1 \\
Jet energy scale                   & 1.3   & 1.3  & 3.0   & 1 \\
Jet energy resolution              & \hspace{-3.2mm}$<$0.1  & \hspace{-3.2mm}$<$0.1 & 0.6   & 1 \\
\ptmiss          & -     & 0.7  & -     & - \\
$\mu_\textrm{R},\mu_\textrm{F}$ scales of \ttbar\ signal (PS)   & 1.2   & 1.7  & 4.4   & 1 \\
$\mu_\textrm{R},\mu_\textrm{F}$ scales of \ttbar\ signal (ME)   & \hspace{-3.2mm}$<$0.1  & 1.1  & \hspace{-3.2mm}$<$0.1  & 1 \\
Hadronization model of \ttbar\ signal  & 1.2   & 5.2  & 3.7   & 1 \\
PDF                   & 0.5   & 0.4  & \hspace{-3.2mm}$<$0.1  & 1 \\
MC sample event count        & 1.4   & 2.4  & 0.1   & 0 \\
\tW background         & 1.4   & 1.6  & 1.3   & 1 \\
WV background         & 0.7   & 0.9  & -     & 1 \\
Z/$\gamma^{*}$ background  & 2.7   & 15.4 & -     & 1 \\
W+jets background     & 2.5   & 0.7  & 3.5   & 1 \\
\QCD\ multijet background        & -     & -    & 2.4   & - \\
\midrule
Data sample event count           & 24.5  & 51.7 & 9.5   & 0 \\
\midrule
Integrated luminosity            & 2.3   & 2.3  & 2.3   & 1 \\
\bottomrule
\end{tabular}
\end{center}
\end{table*}


The combined inclusive \ttbar\ cross section is measured to be
\begin{eqnarray*}
 \stt & = & 69.5 \pm 6.1\,(\stat) \pm 5.6\,(\syst) \pm 1.6\,(\lumi)\,{\mathrm{pb}}~=~69.5~\pm~8.4\,({\textrm{total}})\,{\mathrm{pb}}\, ,
\end{eqnarray*}
where the total uncertainty is the sum in quadrature of the individual sources of uncertainty. 
The weights of the individual measurements, to be understood in the sense of Ref.~\cite{Valassi:2013bga}, are 81.8\% for $\ell+$jets, 13.5\% for \empm, and 4.7\% for \mmpm final states.

The combined result is found to be robust by performing an iterative variant of the BLUE method~\cite{Lista:2014qia}---at each iteration the uncertainty is rescaled and the weights are recalculated---and varying some assumptions on the correlations of different combinations of systematic uncertainty. 
Also, the post-fit correlations between the nuisance parameters in the $\ell$+jets final state have been checked, and found to have negligible impact by retrieving 
the combination result as the input $\ell$+jets measurement in the limit of very large uncertainty in the dilepton final states.

Figure~\ref{fig:sqrt} presents a summary of CMS measurements~\cite{Khachatryan:2016mqs,Khachatryan:2016yzq,CMS-PAS-TOP-16-005,CMS-PAS-TOP-16-006}
of $\sigma_{\ttbar}$ in pp collisions at different \sqrts in the $\ell$+jets and dilepton final states, compared to the NNLO+NNLL
prediction using the NNPDF3.0 PDF set with $\alpha_{\textrm{s}}(M_{\mathrm{Z}})=0.118$ and $m_{\textrm{top}} = 172.5$\,\GeV.
In the inset, the results from this analysis at $ \sqrts$ = 5.02\,\TeV\ are also compared to the predictions from the MMHT14~\cite{Harland-Lang:2014zoa}, CT14~\cite{Dulat:2015mca}, and 
ABMP16~\cite{Alekhin:2017kpj}
PDF sets, with the latter using $\alpha_{\textrm{s}}(M_{\mathrm{Z}})=0.115$ and $m_{\textrm{top}} = 170.4$\,\GeV.
Theoretical predictions using different PDF sets have comparable values and uncertainties, once consistent values of $\alpha_{\textrm{s}}$ and $m_{\textrm{top}}$ are associated with the respective PDF set.


\begin{figure*}[!ht]
\centering
{\includegraphics[width=0.98\textwidth]{figures/pp5TeV/ttbar_sqrts_5TeV_PDFInfo.pdf}}
\caption[Inclusive \stt measurements from \texttt{CMS} in  pp collisions as a  function of the center-of-mass energy]{
  Inclusive \stt in
  pp collisions as a
  function of the center-of-mass energy; 
  previous CMS measurements at $ \sqrts$ = 7, 8~\cite{Khachatryan:2016mqs,Khachatryan:2016yzq}, and 13~\cite{CMS-PAS-TOP-16-005,CMS-PAS-TOP-16-006}\,\TeV\ in the separate $\ell$+jets and dilepton 
  final states are displayed, along with the combined measurement at 5.02\,\TeV\ from this analysis~\citeTH{topobs_CMS_jhep}.
  The NNLO+NNLL theoretical prediction~\cite{mitov} using the NNPDF3.0~\cite{Ball:2014uwa} PDF set with $\alpha_{\textrm{s}}(M_{\mathrm{Z}})=0.118$ and $m_{\textrm{top}} = 172.5$\,\GeV\ is shown in the main plot. 
  In the inset, additional predictions at $ \sqrts$ = 5.02\,\TeV\ using the MMHT14~\cite{Harland-Lang:2014zoa}, CT14~\cite{Dulat:2015mca}, and ABMP16~\cite{Alekhin:2017kpj} PDF sets, 
  the latter with $\alpha_{\textrm{s}}(M_{\mathrm{Z}})=0.115$ and $m_{\textrm{top}} = 170.4$\,\GeV, are compared, along with the \NNPDFzero prediction, to the individual and combined results from this analysis.
  The vertical bars and bands represent the total uncertainty in the data and in the predictions, respectively.
}
\label{fig:sqrt}
\end{figure*}
