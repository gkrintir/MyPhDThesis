\subsection{The dilepton final state}
\label{sec:res_dil}

In the dilepton analysis, the \ttbar\ cross section is extracted from an event counting measurement.
Figure~\ref{fig:jetsAnddilepton} shows the distributions of the jet multiplicity and the scalar \pt sum of all jets ($H_{\textrm{T}}$),
for events passing the dilepton criteria in the \empm final state.
In addition, it displays the lepton-pair invariant mass and \pt distributions,
after requiring at least two jets in the event in the \empm final state. 
Figure~\ref{fig:dimuon} shows the \ptmiss and the lepton-pair invariant mass distributions in the \mmpm final state for events passing 
the dilepton criteria, and the Z boson veto with the $\ptmiss > 35$\,\GeV\ requirement, in the second case.
 The predicted distributions take into account the efficiency corrections described in
 Section~\ref{sec:evtSel} 
and the background estimations discussed in Section~\ref{sec:bkg_dil}.
Good agreement is observed between
the data and predictions for both signal and background.
\begin{figure}[!ht]
\centering
\includegraphics[width=0.49\textwidth]{figures/pp5TeV/H_NJets_ElMu_dilepton}
\includegraphics[width=0.49\textwidth]{figures/pp5TeV/H_HT_ElMu_dilepton}
\includegraphics[width=0.49\textwidth]{figures/pp5TeV/H_InvMass_ElMu_2jets}
\includegraphics[width=0.49\textwidth]{figures/pp5TeV/H_DiLepPt_ElMu_2jets}
\caption[Predicted and observed distributions of the jet multiplicity and $H_\textrm{T}$ for the dilepton final states at $\sqrt{s}=5.02$\,\TeV]{
Predicted and observed distributions of the (upper row) jet multiplicity and scalar \pt sum of all jets ($H_{\textrm{T}}$) for events passing the dilepton criteria, and of the (lower row) 
invariant mass and \pt of the lepton pair after requiring at least two jets, in the \empm final state.
The Z/$\gamma^{*}$ and \nonW backgrounds are determined from data (see Section~\ref{sec:bkg_dil}).
The cross-hatched band represents the statistical and integrated luminosity uncertainties in the expected signal and background yields added in quadrature.
The vertical bars on the data points represent the statistical uncertainty.
The last bin of the distributions contains the overflow events~\citeTH{topobs_CMS_jhep}.
}
\label{fig:jetsAnddilepton}
\end{figure}
\begin{figure}[!ht]
\centering
\includegraphics[width=0.49\textwidth]{figures/pp5TeV/H_MET_Muon_ZVeto}
\includegraphics[width=0.49\textwidth]{figures/pp5TeV/H_InvMass_Muon_MET}
\caption[Predicted and observed distributions of the \ptmiss and $M_{\ell^{\pm}\ell^{\mp}}$ for the dilepton final states at $\sqrt{s}=5.02$\,\TeV]{
Predicted and observed distributions of the (left) \ptmiss in events passing the dilepton criteria and Z boson veto, and of the (right) invariant mass
of the lepton pair after the $\ptmiss > 35$\,\GeV\ requirement in the \mmpm final state.
The cross-hatched band represents the statistical and integrated luminosity uncertainties in the expected signal and background yields added in quadrature.
The vertical bars on the data points represent the statistical uncertainty.
The last bin of the distributions contains the overflow events~\citeTH{topobs_CMS_jhep}. 
}
\label{fig:dimuon}
\end{figure}

The fiducial \ttbar\ production cross section is measured by counting events in the visible phase space (defined by the same \pt, $\vert \eta \rvert$, and  multiplicity requirements 
for leptons and jets as in Section~\ref{sec:evtSel}, but including the transition region for electrons) and is denoted by $\sigma_{\textrm{fid}}$. 
It is extrapolated to the full phase space in order to determine the inclusive \ttbar\ cross section using the expression
\begin{equation}
\stt = \frac{N - N_\mathrm{B}}{\varepsilon \times\, \mathcal{A} \,\times {\mathcal{L}}} = \frac{\sigma_{\textrm{fid}}}{\mathcal{A} }\, ,
\label{eqn:xsecA}
\end{equation}
where $N$ is the  total number of dilepton events observed in data, $N_\mathrm{B}$ the
number of estimated background events,
$\varepsilon$ the selection efficiency, 
$\mathcal{A}$ the acceptance, 
 and $\mathcal{L}$ the integrated luminosity. Table~\ref{tab:yields_dilepton} gives the total number of events observed in
data, together with the total number of signal and background events expected
from simulation or estimated from data, after the full set of selection criteria.
The total detector, trigger, and reconstruction efficiency is estimated from data to be $\varepsilon = 0.55~\pm~0.02$ ($0.57~\pm~0.04$) in the \empm (\mmpm) final state.
Using the definitions above, the yields from Table~\ref{tab:yields_dilepton}, and the systematic uncertainty from Table~\ref{tab:breakdown_comb}, the measured fiducial cross section for \ttbar\ production is 
\begin{equation*}
\sigma_{\textrm{fid}}  =  41 \pm 10\,(\stat) \pm 2\,(\syst) \pm 1\,(\lumi)\,{\mathrm{pb}}\, ,
\end{equation*}
in the \empm\ final state, and
\begin{equation*}
\sigma_{\textrm{fid}}  =  22 \pm 11\,(\stat) \pm 4\,(\syst) \pm 1\,(\lumi)\,{\mathrm{pb}}\, ,
\end{equation*}
in the \mmpm\ final state.


\begin{table}[!ht]
\centering
\caption[The predicted and observed number of events in the dilepton final states at $\sqrt{s}=5.02$\,\TeV]{
The predicted and observed numbers of dilepton events obtained after applying the full selection.
The values are given for the individual sources of background, \ttbar\ signal, and data.
The uncertainties are of statistical nature~\citeTH{topobs_CMS_jhep}.
}
\begin{tabular}{lcc}
\toprule
Source                  & \empm\ & \mmpm      \\
\midrule
  \tW\             & 0.92 $\pm$ 0.02  &   0.29 $\pm$ 0.01\\
   Non-W/Z leptons      & 1.0  $\pm$ 0.9   &   0.04 $\pm$ 0.01\\
   Z/$\gamma^{*}$       & 1.6  $\pm$ 0.2   &   1.1  $\pm$ 0.8\\
  \WV                   & 0.44 $\pm$ 0.02  &   0.15 $\pm$ 0.01\\ 
  \midrule
  \ttbar\ signal         & 18.0 $\pm$ 0.3   &   6.4 $\pm$ 0.2  \\
  \midrule
  Total                 & 22.0  $\pm$ 0.9   &  7.9 $\pm$ 0.8\\
  \midrule
  Observed data         & 24		   &   7\\
\bottomrule
\end{tabular}
\label{tab:yields_dilepton}

\end{table}


\begin{table}[!ht]
\caption[Summary of the individual contributions to the systematic uncertainty in the $\stt$ measurement for the dilepton final states at $\sqrt{s}=5.02$\,\TeV]{
Summary of the individual contributions to the systematic
uncertainty in the $\stt$ measurements for the dilepton final states. The relative uncertainty $\Delta\stt /\stt$ (in \%),
as well as the absolute uncertainty in \stt, $\Delta\stt$ (in pb), are presented. The statistical and total uncertainties are also given, 
where the latter are the quadrature sum of the statistical and systematic uncertainties~\citeTH{topobs_CMS_jhep}.}
\centering
\begin{tabular}{lcccc}
\toprule
                                     & \multicolumn{2}{c}{\empm}  & \multicolumn{2}{c}{\mmpm}   \\
\cmidrule(l){2-3} \cmidrule(l){4-5}
Source                               & $\Delta\stt /\stt$ (\%)   & $\Delta\stt$ (pb)  &  $\Delta\stt /\stt$ (\%)  &  $\Delta\stt$ (pb) \\
\midrule
Electron efficiency                & 1.4  & 1.0   & ---  &  --- \\
Muon efficiency                    & 3.0  & 2.3   & 6.1  &  3.6 \\
Jet energy scale                     & 1.3  & 1.0   & 1.3  &  0.7 \\
Jet energy resolution                & \hspace{-4.1mm}$<0.1$ & \hspace{-4.1mm}$<0.1$  & \hspace{-4.1mm}$<0.1$ & \hspace{-4.1mm}$<0.1$ \\
Missing transverse momentum          & ---  & ---   & 0.7  &  0.4 \\

$\mu_\textrm{R},\mu_\textrm{F}$ scales of \ttbar\ signal (PS)     & 1.2  & 0.9   & 1.7  & 1.0 \\
$\mu_\textrm{R},\mu_\textrm{F}$ scales of \ttbar\ signal (ME)     & 0.2  & 0.1   & 1.1  & 0.6 \\
Hadronization model of \ttbar\ signal & 1.2  & 0.9   & 5.2  & 3.1 \\
PDF                                  & 0.5  & 0.4   & 0.4  & 0.2 \\

MC sample size & 1.4  & 1.1   & 2.4  & 1.4 \\

\tW background                 &  1.4 & 1.1   & 1.6  & 0.9 \\
WV background                        &  0.7 & 0.5   & 0.9  & 0.5 \\
Z/$\gamma^{*}$ background            &  2.7 & 2.1   & 15   & 9.1 \\
Non-W/Z background                   &  2.5 & 1.9   & 0.7  & 0.4 \\
\midrule
Total systematic uncertainty         & \multirow{2}{*}{ 5.8 } & \multirow{2}{*}{ 4.4 }  & \multirow{2}{*}{ 18 }  & \multirow{2}{*}{ 11 } \\
(w/o integrated luminosity)          &        &        &      &  \\
\midrule 
Integrated luminosity                &   2.3  &   1.8  &  2.3 &  1.4 \\
\midrule
Statistical uncertainty              &  25  &  19  & 48 & 29 \\
\midrule
Total uncertainty                    &  25  &  19  & 52  & 31 \\
\bottomrule
\end{tabular}
\label{tab:breakdown_comb}
\end{table}



The acceptance, as estimated from MC simulation, is found to be $\mathcal{A} = 0.53~\pm~0.01$ ($0.37~\pm~0.01$) in the \empm (\mmpm) final state. The statistical uncertainty (from MC simulation) is included in the uncertainty in $\mathcal{A}$. By extrapolating to the full phase space, the inclusive \ttbar\ cross section is measured to be 
\begin{equation*}
\stt  =  77 \pm 19\,(\stat) \pm  4\,(\syst) \pm 2\,(\lumi)\,{\mathrm{pb}}\, ,
\end{equation*}
in the \empm\ final state, and
\begin{equation*}
\stt  =  59 \pm 29\,(\stat) \pm 11\,(\syst) \pm 1\,(\lumi)\,{\mathrm{pb}}\, ,
\end{equation*}
in the \mmpm\ final state. 
Table~\ref{tab:breakdown_comb} summarizes the 
relative and absolute statistical and
systematic uncertainties from different sources contributing to \stt. 
The separate total systematic uncertainty without the uncertainty in the integrated luminosity, the part attributed to the integrated luminosity, and the statistical
contribution are added in quadrature to obtain the total uncertainty.
The cross sections, measured with a relative uncertainty of 25 and 52\%, are in agreement with the SM prediction (Eq.\,(\ref{eq:theory})) within the large uncertainty in the measurements.

