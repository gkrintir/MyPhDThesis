
The event sample of proton-nucleus collisions collected by the \texttt{CMS} detector in 2016 corresponds to an integrated luminosity of $\mylumipA$~\citeTH{CMS-PAS-LUM-17-002}.
The lead nuclei and protons had beam energies of 2.56 and 6.5\,\TeV\ per nucleon, respectively, corresponding to a nucleon-nucleon centre-of-mass energy of $\rootsNN = 8.16$\,\TeV.
The direction of the proton beam was initially clockwise (``Pbp'' configuration) and was then reversed (``pPb'' configuration). 
Owing to the energy difference between the proton-lead colliding beams, the nucleon-nucleon center-of-mass (CM) frame is not at rest with respect to the laboratory (lab) frame. 
Massless particles emitted at a pseudorapidity $\eta_{\textrm{CM}}$ in the CM frame experience a longitudinal boost according to.
\begin{equation}
\lvert \Delta{\eta}_{\textrm{CM}} \rvert = \frac{1}{2}\times\left|\ln\left(\frac{Z_{\ce{^{208}_{82}Pb+}}\times A_{\mathrm{p}}}{Z_{\mathrm{p}}\times A_{\ce{^{208}_{82}Pb+}}}\right)\right| = \frac{1}{2}\times\ln\left(\frac{208}{82}\right) = 0.465
\label{eq:CMShift}
\end{equation}
The pseudorapidity $\eta_{\textrm{lab}}$ is defined such as to have positive value in the direction of motion of the proton in both Pbp and pPb data samples.
The average number of collisions per bunch crossing is unity in the combined data set, and assuming a pp inelastic cross section of 69.2\,$\mathrm{mb}$ multiplied by $A = 208$.


The $\mathrm{pN} \rightarrow \ttbar + \mathrm{X}$ process ($\mathrm{N = p,n}$) is simulated using the LO \PYTHIA~(v6.424~\cite{Sjostrand:2006za}, tune $\mathrm{Z}2^{*}$~\cite{Chatrchyan:2013gfi,Khachatryan:2015pea})
 and the NLO \POWHEG~(v2~\cite{powheg,powheg2}, tune CUETP8M1~\cite{Khachatryan:2015pea,Skands:2014pea}) generators with a mixture of \pp and \pn interactions corresponding 
 to their ratio in pPb collisions. 
The nuclear modification of the up- and down-type valence quark and gluon distribution functions are rendered using the EPPS16 \cite{EPPS16} nuclear PDFs for the {\ce{^{208}_{82}Pb+} ions. 
The quark densities are scaled according to the isospin symmetry
\begin{equation}
\begin{aligned}
f^{\cPqu}_{\ce{^{208}_{82}Pb+}}(A,\,Z) &= \frac{Z}{A}\left(R^{\cPqu}_{s}f^{\cPaqu}_{\Pp} + R^{\cPqu}_{v}\left(f^{\cPqu}_{\Pp}-f^{\cPaqu}_{\Pp}\right)\right) + \frac{A-Z}{A}\left(R^{\cPqd}_{s}f^{\cPaqd}_{\Pp} + R^{\cPqd}_{v}\left(f^{\cPqd}_{\Pp}-f^{\cPaqd}_{\Pp}\right)\right) \\
f^{\cPqd}_{\ce{^{208}_{82}Pb+}}(A,\,Z) &= \frac{Z}{A}\left(R^{\cPqd}_{s}f^{\cPaqd}_{\Pp} + R^{\cPqd}_{v}\left(f^{\cPqd}_{\Pp}-f^{\cPaqd}_{\Pp}\right)\right) + \frac{A-Z}{A}\left(R^{\cPqu}_{s}f^{\cPaqu}_{\Pp} + R^{\cPqu}_{v}\left(f^{\cPqu}_{\Pp}-f^{\cPaqu}_{\Pp}\right)\right)
\end{aligned}
\label{eq:PbScaling}
\end{equation}
where $f^{\cPqu}_{\ce{^{208}_{82}Pb+}}$ and $f^{\cPqd}_{\ce{^{208}_{82}Pb+}}$ represent the up- and down-type valence quark PDF inside the \ce{^{208}_{82}Pb+} ion, and $R_{v}$ ($R_{s}$) parameterize 
the nuclear modification factor for valence (sea) quarks. 
The value of $m_{\cPqt}$ used in all simulated samples is 172.5\,\GeV. 

Simulated samples of \PW+jets and Drell--Yan 
production of charged-lepton pairs with an invariant mass larger than 30\,\GeV\ are generated using \PYTHIA~6. 
The MC is used solely for efficiency measurements and validation of the functional forms used for the background distributions
since the latter is determined {\it in situ} from the data.
All signal and background samples are embedded (see Section~\ref{sec:pileup}) into pPb events generated with \EPOS-{\sc lhc}~\cite{epos_lhc} (v.3400), 
tuned to reproduce the global pPb event properties experimentally measured, and reconstructed with the same analysis code as used for the data. 
The kinematics of all MC-generated events are boosted to account for the different energies of the proton and lead beam.
Simulated samples include an emulation of the full detector response, based on \GEANTfour~\cite{geant4}, 
with simulated alignment and calibration conditions tuned on data, and a realistic description of the luminous region (see Section~\ref{sec:beamspot}) produced by the collisions.


\begin{sidewaystable}[htp]
\caption[MC event samples used for the inclusive \stt measurement at $\rootsNN=8.16$\,\TeV]{
MC data samples for the inclusive \stt measurement using pPb collisions at$\rootsNN = 8.16$\,\TeV~\citeAN{AN-17-043}. 
The samples are generated either inclusively or with a final state restricted to the leptonic mode,
including either electrons or muons. 
The absence of reference delineates cross section values calculated from the generator. 
For the samples restricted to specific decay channels, the branching ratio is included in the quoted cross section value.
The samples with "Emb" in the name are embedded using the \textsc{epos-lhc} (v.3400) generator (see Section~\ref{sec:pileup}).
}
\label{tab:mc_pPb8TeV}
\centering
\begin{tabular}{c||c|l}
                        \toprule
                        Process        & $\sigma[{\mathrm{nb}}]$  & MC event sample indicative description  \\
                        \midrule
                        \multirow{4}{*}{\ttbar}  & \multirow{4}{*}{
                          59.0 {\small (NNLO+NNLL)}~\cite{mcfm}} & \small{\verb!TTbar_pPb-EmbEPOS_8160GeV_pythia6!}\\
                                                                & & \small{\verb!TTbar_PbP-EmbEPOS_8160GeV_pythia6!}\\
                                                                & & \small{\verb!TT_TuneCUETP8M1_pPb-EmbEPOS_8160GeV-powheg-pythia8!} \\
                                                                & & \small{\verb!TT_TuneCUETP8M1_Pbp-EmbEPOS_8160GeV-powheg-pythia8!} \\
                        \midrule
                                \multirow{3}{*}{$\PW+$\,jets}
                        &  \multirow{3}{*}{ 1970 } 
                                                   &\small{\verb!WJetstoLNu_TuneZ2_pPb-EmbEPOS_8160GeV_pythia6!}\\
                                                   & &\small{\verb!WJetstoLNu_TuneZ2_Pbp-EmbEPOS_8160GeV_pythia6!}\\ 
                                                   
                         \midrule
                         \multirow{3}{*}{$\cPZ/\gamma^{*}(\to\ell^{\pm}\ell^{\mp})+$\,jets}
                        &  \multirow{3}{*}{ 229 } & \small{\verb!DYJetstoLL_TuneZ2_M-30_pPb-EmbEPOS_8160GeV_pythia6!}\\
                                                  & & \small{\verb!DYJetstoLL_TuneZ2_M-30_Pbp-EmbEPOS_8160GeV_pythia6!}\\
                                                  

                         \bottorule
                         \end{tabular}
\end{sidewaystable}
