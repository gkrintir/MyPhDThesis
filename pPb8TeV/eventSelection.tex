\subsubsection{Event selection}
\label{sec:evcat}

%The two-tier trigger system (see Section~\ref{sec:egm_trigger}) selects events of interest for offline analysis.
 This analysis is restricted to events that fired trigger paths requiring the presence of at least one muon (electron) candidate 
 with transverse momentum (energy) $\pt > 12\,\GeV$ ($\et > 20\,\GeV$). 
Looser online identification criteria are applied as compared to the offline selection, and no requirement on additional analysis objects is imposed at this level.

Particle candidates are reconstructed offline with the \texttt{CMS} PF algorithm, which 
identifies and provides a list of particles using an optimized combination of information from the various elements of the \texttt{CMS} detector. 
Events are required to contain exactly one tight muon (see Section~\ref{sec:muons}) or medium electron (see Section~\ref{sec:electrons}) candidate, with $\pt > 30\,\GeV$ and $\lvert\eta\rvert < 2.1$, 
excluding in the electron case the transition region $1.444 < \lvert\eta\rvert < 1.566$ between the \texttt{ECAL} barrel and endcap, where the reconstruction of electron objects is less efficient. 
The muon and electron candidates are required to be isolated from nearby hadronic activity within a cone of $\Delta R = 0.3$ 
around the direction of the track at the primary event vertex, 
where $\Delta R = \sqrt{\smash[b]{(\Delta \eta)^2 + (\Delta \phi)^2}}$, and $\Delta
\eta$ and $\Delta \phi$ are the separations in pseudorapidity and azimuthal angle. 
A charged lepton is selected if its relative isolation discriminant value satisfies $I_\text{rel}<0.15$ (muon),  
$0.07$ (electron in the barrel), or $0.08$ (electron in one of the endcaps). 
These thresholds have been optimized to reduce the contamination from nonprompt leptons. 
To remove the Drell--Yan background, events are rejected from the analysis if they contain extra electrons (muons) that are reconstructed 
using a looser set of identification criteria and have $\pt>20$ ($15$)\,\GeV\ within $\lvert\eta\rvert < 2.5$ ($2.4$).
The efficiency of the lepton selection is measured using a tag-and-probe method in events enriched with \PZ\ boson candidates 
and selected by the same trigger requirements as the signal candidate events.
The combined reconstruction, lepton identification, and trigger efficiency is determined as a function of lepton \pt and $\eta$.


Events are required to have at least four reconstructed jets with $\pt> 25\,\GeV$ and $\lvert\eta\rvert< 2.5$, 
that are separated by at least $\Delta R = 0.3$ from the selected muon or electron. 
Jets are reconstructed from the PF candidates using the anti-\kt clustering algorithm with a distance parameter of 0.4. 
Jet energy corrections extracted from the full detector simulation are applied as functions of jet $\pt$ and
$\eta$~\cite{cms_jets,CMS-DP-2018-028} to both data and simulated samples. A residual correction to the data is applied to account for a
small data-MC discrepancy in the jet energy response. 
Jets from \cPqb\ quarks are tagged based on the presence of a secondary vertex from B-hadron decays, identified using 
a multivariate algorithm combining tracking information.

The distinct \ttbar\ signature of two \cPqb\ jets in the event, which rarely occurs in background processes such as \PW+jets and \QCD\ multijet 
(collectively labeled as ``non-top'' background), is used to extract the signal.
The number of jets passing a threshold on the \cPqb-jet identification discriminant, corresponding to a \cPqb\ tagging efficiency of 
approximately 70\% with a misidentification rate of less than 0.1\% for light-flavor jets, as estimated in simulated pp and cross-checked with pPb events (Table~\ref{tab:OP}), 
is used to classify the selected events into no (0\cPqb), exactly one (1\cPqb{}), or at least two ($\geq\ $2\cPqb{}) tagged-jet categories, described in the following. 
All three event categories are exploited in a
maximum-likelihood fit in order to extract the signal cross section, and simultaneously constrain the background contamination 
and determine the efficiency of the \cPqb\ jet identification.

Table \ref{tab:cutFlow} summarizes the selection requirements that are applied in the analysis to select a high purity $\ttbar\rightarrow \ell+\textrm{jets}$, $\ell$ = $\mathrm{e}$ or $\mu$, sample.

\begin{table}[!ht]
\begin{center}
  \caption[Selection criteria applied in the $\mu$+jets and $\mathrm{e}$+jets final states for the inclusive \stt measurement at $\rootsNN=8.16$\,\TeV]{
Selection criteria applied for the inclusive \ttbar\ cross section measurement in the $\mu$+jets and $\mathrm{e}$+jets final states using pPb collisions at $\rootsNN=8.16$\,\TeV.}
\label{tab:cutFlow} 
\makebox[\textwidth][c]{  
\begin{tabular}{c||c}
\toprule
Filter or physics object & Selection \\
\midrule
Trigger          & one $\mu$ (e) candidate, \pt$>$12 (\et$>$20)\,\GeV        \\
Electrons        & medium ID, $\pt>30\GeV$, $|\eta|<2.1$, $\PFrelIsoRho<0.08$ \\
Muons            & tight ID,  $\pt>30\GeV$, $|\eta|<2.1$, $\PFrelIsoDb<0.15$   \\
Jets             & loose ID, $\pt>25\GeV$, $|\eta|<2.5$                        \\
Jet multiplicity & $\geq 4$                                                   \\
\cPqb{} tagging  & CSVv2\,\texttt{M} (Table~\ref{tab:OP}) $\Rightarrow$ 1$\ell$4j0b, 1$\ell$4j1b, and 1$\ell$4j2b exclusive categories \\
\bottomrule
\end{tabular}}
\end{center}
\end{table}


\subsubsection{Event categorization}
\label{sec:evcat}

As we observed in the measurement of the inclusive \ttbar\ cross section in the $\ell$+jets final state using pp collisions at $\sqrt{s}=5.02$\,\TeV~\citeTH{topobs_CMS_jhep}, 
it is expected the main contamination from background processes to originate from \QCD\ multijet and \PW+jets events mainly. 
Given the characteristics of these processes, categorization of events, according to the number of \cPb-tagged jets, separates signal from background. 
The reconstruction of the kinematics of the \PW{} boson and top quark decays should further distinguish between the resonant nature of the signal and the continuum from the background.

Figure~\ref{fig:anaflow} summarizes the event categories of the analysis that are attained based on identification and isolation requirements of the lepton candidates, 
and the heavy-flavor content of the jets, i.e., identified as \cPqb or light quark jets. 
The signal dominates the event categories with at least one lepton and four jets, two of which are \cPqb tagged (the ``1$\ell$4j2b'' category). 
If only one jet passes the \cPqb-tagging requirement the event is classified to the ``1$\ell$4j1b'' category, 
while in case none of the jets satisfies the \cPqb-tagging requirement the event falls in the ``1$\ell$4j0b'' category. 
To model the \QCD\ multijet background an additional category is considered where the lepton fails identification or isolation requirements, and to further 
reduce the contamination from \PW{}+jets and \ttbar\ events in the \QCD\ multijet control region no jet should fulfill the \cPqb-tagging requirement. 
The category is referred to as ``1f4j0b'', with ``f'' standing for fake.

\begin{figure}[!ht]
\centering
\subfloat[][]{\includegraphics[width=0.99\textwidth]{figures/pPb8TeV/analysisflow}}
\caption[Representation of the physics objects and event categories in the $\ell$+jets final state at $\rootsNN=8.16$\,\TeV]{
Representation of the physics objects and event categories (analysis boxes) for the inclusive \ttbar\ cross section in the $\ell$+jets final state 
using pPb collisions at $\rootsNN=8.16$\,\TeV~\citeAN{AN-17-043}.
}
\label{fig:anaflow}
\end{figure}

\subsubsection{Reconstruction of the \PW{} boson leptonic and hadronic decays}
\label{sec:wrec}

In the case of \PW{} boson leptonic decay in a charged lepton and associated neutrino the reconstructed \ptmiss (see Section~\ref{sec:ptmiss}) is used to infer the \pt of the neutrino. 
The longitudinal component of the neutrino momentum, $p_{\nu,z}$, is computed by using the energy-momentum conservation at the $\PW{}_{\ell\nu}$ vertex 
and constraining the W boson mass to $\mW=80.4$\,\GeV~\cite{pdg_2018}. This leads to a quadratic equation in $p_{\nu,z}$ with solutions as
\begin{linenomath}
\begin{equation}
\label{eq:pznu}
p_{z,\nu} =\frac{\Lambda p_{z,\ell}}{p_{\textrm{T},\ell}^2}\pm\frac{1}{p_{\textrm{T},\ell}^2}\sqrt{\Lambda^2 p_{z,\ell}^2-p_{\textrm{T},\ell}^2(E_{\ell}^{2} \ptmisssquared-\Lambda^2)}\, ,
\end{equation}
\end{linenomath}
where
\begin{linenomath}
\begin{equation}
\label{eq:lambda}
\Lambda=\frac{m_{\PW}^2}{2}+\vec{p}_{\textrm{T},\ell}\times{ \vecptmiss }\, , 
\end{equation}
\end{linenomath}
and $E_{\ell}^{2}=p_{\textrm{T},\ell}^2 + p_{z,\ell}^2$ denotes the energy of the charged lepton.
In most of the cases this leads to two real solutions for $p_{z,\nu}$ and the solution that minimizes $|p_{\nu,z}-p_{\ell,z}|$ is chosen~\citeTH{top16003}. 
For some events the discriminant in Eq.\,(\ref{eq:pznu}) becomes negative leading to complex solutions for $p_{z,\nu}$. 
In that case, the imaginary component is eliminated by the modification of $\vecptmiss$ so that $\mTW = \mW$, while still respecting the $\mW$ constraint. 
This is achieved by imposing that the determinant, and thus the square-root term in Eq.\,(\ref{eq:pznu}), is null. 
This condition gives a quadratic relation between $p_{x,\nu}$ and $p_{y,\nu}$ with two possible solutions, and one remaining degree of freedom. 
The solution is chosen by finding the neutrino transverse momentum $\vec{p}_{\textrm{T},\nu}$  that has the minimum vectorial distance from the \vecptmiss in the $p_x\!-\!p_y$ plane. 
With the kinematics of the neutrino fully specified, the kinematics of the leptonically decaying \PW{} boson can be then computed as $P_\PW=P_\ell+P_\nu$, 
where $P_i$ is the four-vector of particle species $i$.

\begin{figure}[!ht]
\centering
\subfloat[][]{\label{wjjsel:a}\includegraphics[width=0.48\textwidth]{figures/pPb8TeV/Figure_aux_001-a}}
\subfloat[][]{\label{wjjsel:b}\includegraphics[width=0.48\textwidth]{figures/pPb8TeV/Figure_aux_001-b}}
\caption[Dijet invariant mass distributions using different algorithms in the $\ell$+jets final state at $\rootsNN=8.16$\,\TeV]{
Dijet invariant mass (\mjj) spectrum in the 1$\ell$4j2b event category calculated using different algorithms in the pairing of the jets in each event: 
distance-based (\minDeltaR), \pt-based (leading \pt), and minimizing the distance to the \PW{} boson mass $m_\PW$ ($\min|\mjj-m_\PW|$)~\citeTH{topobs_CMS_prl}.
Pairs fully matched to $\PW \rightarrow \cPq\cPqbar$ decays are shown in (a), while pairs with at least one reconstructed jet not matched at parton level are shown on panel (b). 
The results are based on PYTHIA~(v6.424~\cite{Sjostrand:2006za} \ttbar\ simulation, and using the $\mathrm{Z}2^{*}$~\cite{Chatrchyan:2013gfi,Khachatryan:2015pea} tune.
}
\label{fig:wjjsel}
\end{figure}

In case of \PW{} boson hadronic decays in two light quarks, \cPq and \cPqbar, all selected jets, $j$ and $j^{\prime}$, are used after removing up to 
two \cPqb-tagged candidates using the CSVv2\,\texttt{M} working point (Table~\ref{tab:OP}). 
When more than two jets are left a sorting algorithm based on the proximity of the jets by $\Delta R$ is applied. 
The $\Delta R_{\textrm{min}}$ criterion preserves a high efficiency with respect to the number of events in which both jets from
the $\PW\rightarrow \cPq\cPqbar$ decay are available while reducing the bias, i.e., any mimicking effects from the background.
A criterion based on the closeness of $\mjj$ to the \PW{} boson mass maximizes efficiency but at the same time biases considerably the background. 
To that end, in the analysis the \minDeltaR sorting algorithm is used, similar to the what has been found to be optimal in~\citeTH{topobs_CMS_jhep}.

A comparison of the sorting algorithms based on simulation can be found in Fig.~\ref{fig:wjjsel} separately for ``correct'' and ``wrong''  assignments. 
Successful combinations of $j$ and $j^{\prime}$ for which the algorithms are able to reconstruct both jets geometrically matched to the corresponding generator level quarks 
from the \PW{} or \cPqt{} decays are shown in Fig.~\ref{wjjsel:a}. 
Wrong or unmatched combinations of $j$ and $j^{\prime}$ for which the algorithms either select jets that cannot be geometrically matched to generator level information 
or mix these physics objects from different decays are displayed in Fig.~\ref{wjjsel:b}.


\begin{figure}[!ht]
\begin{center}
\subfloat[][]{\label{wjjreco:a}\includegraphics[width=0.32\textwidth]{figures/pPb8TeV/Figure_aux_003-a}}
\subfloat[][]{\label{wjjreco:b}\includegraphics[width=0.32\textwidth]{figures/pPb8TeV/Figure_aux_003-b}}
\subfloat[][]{\label{wjjreco:c}\includegraphics[width=0.32\textwidth]{figures/pPb8TeV/Figure_aux_003-c}}
\caption[Pre-fit dijet invariant mass distributions in the $\ell$+jets final state at $\rootsNN=8.16$\,\TeV]{
Dijet invariant mass (\mjj) distributions in the 1$\ell$4j0b (a), 1$\ell$4j1b (b), and 1$\ell$4j2b (c) event categories after the complete event selection. 
On the upper panels, the reconstructed data are compared to the
stacked expected contributions from the signal and the main background processes~\citeTH{topobs_CMS_prl}.
The \ttbar, \PW+jets, and Drell--Yan (DY) processes are simulated with \PYTHIA (v6.424~\cite{Sjostrand:2006za}, tune $\mathrm{Z}2^{*}$~\cite{Chatrchyan:2013gfi,Khachatryan:2015pea}) and normalized to the theoretical cross sections (Table~\ref{tab:mc_pPb8TeV}). 
The \QCD\ multijet (Multijets) contribution is estimated from data using a control region.
The bottom panels display the ratio between the data and the expectations.
The shaded band represents the relative uncertainty due to the limited statistics in the simulated samples and in the estimate of the normalization of the \QCD\ multijet background.
}
\label{fig:wjjreco}
\end{center}
\end{figure}


\subsubsection{Top pair decay reconstruction}
\label{sec:toprec}

Once the \PW{} bosons have been reconstructed they should be optimally combined with the \cPqb{}-jet candidates to retrieve the $\cPqt\rightarrow\PW\cPqb$ decay chains.
In cases where jets do not fulfill the \cPqb-tagging threshold, the two jets with the highest CSVv2 discriminator value are used as the \cPqb-jet candidates.
In the rest of the cases, the pairing of each \cPqb-jet candidate to \PW{} boson candidates can be tested using different ranking algorithms.  
On the one hand, the proximity-based metric $\Delta R_{\textrm{min}}$ is expected to outperform in high-\pt events, and top-quark-mass-based criteria may induce a bias on the background. 
On the other hand, a less biased mass requirement consists of minimizing the difference between the mass of the hadronic (all jets,\,$\cPqt_{\textrm{had}}$) and leptonic ($\cPqb\ell\nu$,\,$\cPqt_{\textrm{lep}}$) 
\begin{table}[!ht]
\begin{center}
\caption[Metrics used to evaluate the quality of the top pair decay reconstruction at $\rootsNN=8.16$\,\TeV]
{Fraction of events in which two \cPqb-tagged jets are successfully matched to parton level
\cPqb{} quarks from top decays (2b) and are subsequently correctly matched to reconstruct
the $\cPqt_{\textrm{had}}$ and $\cPqt_{\textrm{lep}}$ decay chains. 
The pairing was tested using three metrics, i.e., a proximity-based ($\Delta R$) and two mass-based metrics that 
minimize the mass difference of either $\cPqt_{\textrm{had}}$ and $\cPqt_{\textrm{lep}}$ quarks ($\min|m_{\cPqt_{\textrm{had}}}-m_{\cPqt_{\textrm{lep}}}|$) 
or the mass difference of $\cPqt_{\textrm{had}}$ quark to the world average~\cite{pdg_2018} $m_\cPqt$ value ($\min|m_{\cPqt_{\textrm{had}}}-m_{\cPqt}|$)~\citeAN{AN-17-043}.
}
\label{tab:teff} \begin{tabular}{lcccc}
\hline
Category & 2\cPqb & $\cPqt_{\textrm{had}}$ & $\cPqt_{\textrm{lep}}$ & Algorithm \\
\hline
    \multirow{3}{*}{1$\ell$4j0b} & \multirow{3}{*}{$0.170\pm0.004$} &
   $0.098\pm0.003$ & $0.307\pm0.005$ & $\min\Delta R$ \\
&& $0.108\pm0.003$ & $0.242\pm0.005$ & $\min|m_{\cPqt_{\textrm{had}}}-m_{\cPqt}|$ \\
&& $0.089\pm0.003$ & $0.289\pm0.005$ & $\min|m_{\cPqt_{\textrm{had}}}-m_{\cPqt_{\textrm{lep}}}|$\\\hline
    \multirow{3}{*}{1$\ell$4j1b} & \multirow{3}{*}{$0.544\pm0.008$} &
   $0.198\pm0.006$ & $0.468\pm0.008$ & $\min\Delta R$ \\
&& $0.200\pm0.006$ & $0.458\pm0.008$ & $\min|m_{\cPqt_{\textrm{had}}}-m_{\cPqt}|$ \\
&& $0.194\pm0.006$ & $0.454\pm0.008$ & $\min|m_{\cPqt_{\textrm{had}}}-m_{\cPqt_{\textrm{lep}}}|$\\\hline
    \multirow{3}{*}{1$\ell$4j2b} & \multirow{3}{*}{$0.791\pm0.005$} &
   $0.265\pm0.006$ & $0.540\pm0.006$ & $\min\Delta R$ \\
&& $0.269\pm0.006$ & $0.535\pm0.006$ & $\min|m_{\cPqt_{\textrm{had}}}-m_{\cPqt}|$ \\
&& $0.266\pm0.006$ & $0.541\pm0.006$ & $\min|m_{\cPqt_{\textrm{had}}}-m_{\cPqt_{\textrm{lep}}}|$\\
\hline
\end{tabular}
\end{center}
\end{table}

Table~\ref{tab:teff} compares the expected efficiency for the different sorting algorithms, and 
further comparisons of the distributions for both correctly and wrongly assigned objects in the reconstruction of top quarks in simulated W+jets events are given in the following. 
We conclude that the strategy based on minimizing the $\cPqt_{\textrm{had}}$ and $\cPqt_{\textrm{lep}}$ mass difference is expected to have comparable efficiency to the other sorting algorithms. 
The resolution is expected to be improved with respect to a $\Delta R_{\textrm{min}}$ ranking while avoiding to induce significant bias from the background, i.e., a peak close to the top mass in the reconstructed distribution. 
Therefore in the analysis the \cPqb-jets are combined with the \PW{} boson candidates based on the $\min|m_{\cPqt_{\textrm{had}}}-m_{\cPqt_{\textrm{lep}}}|$ criterion. 
The output prior to any fit are given in Figs.~\ref{fig:thadsel} and~\ref{fig:tlepsel}. 


\begin{figure}[!ht]
\centering
\subfloat[][]{\includegraphics[width=0.32\textwidth]{figures/pPb8TeV/Figure_aux_004-a}}
\subfloat[][]{\includegraphics[width=0.32\textwidth]{figures/pPb8TeV/Figure_aux_004-b}}
\subfloat[][]{\includegraphics[width=0.32\textwidth]{figures/pPb8TeV/Figure_aux_004-c}}\\
\caption[Pre-fit hadronic top mass distributions in the $\ell$+jets final state at $\rootsNN=8.16$\,\TeV]{
Hadronic top mass (\mtop) distributions in the 0 (a), 1 (b), and 2 (c) \cPqb-tagged jet categories after all selections. 
On the upper panels, the reconstructed data are compared to the stacked expected contributions from signal and the main background processes~\citeTH{topobs_CMS_prl}.
The \ttbar, \PW+jets, and Drell--Yan (DY) processes are simulated with \PYTHIA (v6.424~\cite{Sjostrand:2006za}, tune $\mathrm{Z}2^{*}$~\cite{Chatrchyan:2013gfi,Khachatryan:2015pea}) and normalized to the expected cross sections and integrated luminosity.
The \QCD\ multijet (Multiijets) contribution is estimated from data using a control region. 
The bottom panels display the ratio between the data and the expectations.
The shaded band represents the relative uncertainty due to the limited statistics in the simulated samples and in the estimate of the normalization of the \QCD\ multijet background.
}
\label{fig:thadsel}
\end{figure}

\begin{figure}[!ht]
\centering
\subfloat[][]{\includegraphics[width=0.32\textwidth]{figures/pPb8TeV/Figure_aux_005-a}}
\subfloat[][]{\includegraphics[width=0.32\textwidth]{figures/pPb8TeV/Figure_aux_005-b}}
\subfloat[][]{\includegraphics[width=0.32\textwidth]{figures/pPb8TeV/Figure_aux_005-c}}\\
\caption[Pre-fit leptonic top mass distributions in the $\ell$+jets final state at $\rootsNN=8.16$\,\TeV]
{
  Leptonic top mass (\mtop) distributions in the 0 (a), 1 (b), and 2 (c) \cPqb-tagged jet categories after all selections. 
  On the upper panels, the reconstructed data are compared to the stacked expected contributions from signal and the main background processes~\citeTH{topobs_CMS_prl}.
  The \ttbar, \PW+jets, and Drell--Yan (DY) processes are simulated with \PYTHIA (v6.424~\cite{Sjostrand:2006za}, tune $\mathrm{Z}2^{*}$~\cite{Chatrchyan:2013gfi,Khachatryan:2015pea}) and normalized to the expected cross sections and integrated luminosity.
  The \QCD\ multijet (Multiijets) contribution is estimated from data using a control region. 
  The bottom panels display the ratio between the data and the expectations.
  The shaded band represents the relative uncertainty due to the limited statistics in the simulated samples and in the estimate of the normalization of the \QCD\ multijet background.
}
\label{fig:tlepsel}
\end{figure}


\subsubsection{\PW{} boson and top pair-like reconstruction in \PW+jets background samples}
\label{sec:backgroundeffect}

The expected effect of the reconstruction and sorting algorithms on the W+jets background
is evaluated using MC event simulation. 
Because of the low event count in the default pPb simulation, a larger inclusive \PW+jets simulation is also investigated, 
using pp collisions at 8\,\TeV\ and based on the LO \textsc{MadGraph5}~\cite{mg5} generator.
The results are shown in Figs.~\ref{fig:wtselinwjetspPb} and~\ref{fig:wtselinwjetspp} for the pPb and pp MC event samples, respectively. 
Although the statistical precision is low in both samples, and especially in the 1$\ell$4j2b category,  
It can be seen that for the chosen sorting algorithm  $\min|m_{\cPqt_{\textrm{had}}}-m_{\cPqt}|$ no bias is expected either in the \mjj or the hadronic and leptonic top quark mass distributions, 
irrespective of the event category within statistical uncertainties. 
An additional algorithm for reconstructing the $\cPqt_{\textrm{had}}$ candidates based on their closeness to the world average $m_\cPqt$~\cite{pdg_2018} 
reveals top mass distributions that are severely biased for the W+jets background. 


\begin{figure}[!ht]
\centering
\subfloat[][]{\includegraphics[width=0.32\textwidth]{figures/pPb8TeV/mjj_1l4j2q_wpPb}}
\subfloat[][]{\includegraphics[width=0.32\textwidth]{figures/pPb8TeV/mjj_1l4j1b1q_wpPb}}
\subfloat[][]{\includegraphics[width=0.32\textwidth]{figures/pPb8TeV/mjj_1l4j2b_wpPb}}\\
\subfloat[][]{\includegraphics[width=0.32\textwidth]{figures/pPb8TeV/mthad_1l4j2q_wpPb}}
\subfloat[][]{\includegraphics[width=0.32\textwidth]{figures/pPb8TeV/mthad_1l4j1b1q_wpPb}}
\subfloat[][]{\includegraphics[width=0.32\textwidth]{figures/pPb8TeV/mthad_1l4j2b_wpPb}}\\
\subfloat[][]{\includegraphics[width=0.32\textwidth]{figures/pPb8TeV/mtlep_1l4j2q_wpPb}}
\subfloat[][]{\includegraphics[width=0.32\textwidth]{figures/pPb8TeV/mtlep_1l4j1b1q_wpPb}}
\subfloat[][]{\includegraphics[width=0.32\textwidth]{figures/pPb8TeV/mtlep_1l4j2b_wpPb}}
\caption[Dijet invariant mass, hadronic and leptonic top mass distributions using different algorithms in the $\ell$+jets final state at $\rootsNN=8.16$\,\TeV]{
Dijet invariant mass (a--c), and hadronic (d--f) and leptonic (g--i) top mass distributions in the 0 (a,\,d,\,g), 1 (b,\,e,\,h), and 2 (c,\,f,\,i) \cPqb-tagged jet categories 
after all selections~\citeAN{AN-17-043}, using the simulated \PW+jets MC event sample of pPb collisions (Table~\ref{tab:mc_pPb8TeV}). 
The curves correspond to the different sorting algorithms as described in Table~\ref{tab:teff}.
}
\label{fig:wtselinwjetspPb}
\end{figure}

\begin{figure}[!ht]
\centering
\subfloat[][]{\includegraphics[width=0.32\textwidth]{figures/pPb8TeV/mjj_1l4j2q_wpp}}
\subfloat[][]{\includegraphics[width=0.32\textwidth]{figures/pPb8TeV/mjj_1l4j1b1q_wpp}}
\subfloat[][]{\includegraphics[width=0.32\textwidth]{figures/pPb8TeV/mjj_1l4j2b_wpp}}\\
\subfloat[][]{\includegraphics[width=0.32\textwidth]{figures/pPb8TeV/mthad_1l4j2q_wpp}}
\subfloat[][]{\includegraphics[width=0.32\textwidth]{figures/pPb8TeV/mthad_1l4j1b1q_wpp}}
\subfloat[][]{\includegraphics[width=0.32\textwidth]{figures/pPb8TeV/mthad_1l4j2b_wpp}}\\
\subfloat[][]{\includegraphics[width=0.32\textwidth]{figures/pPb8TeV/mtlep_1l4j2q_wpp}}
\subfloat[][]{\includegraphics[width=0.32\textwidth]{figures/pPb8TeV/mtlep_1l4j1b1q_wpp}}
\subfloat[][]{\includegraphics[width=0.32\textwidth]{figures/pPb8TeV/mtlep_1l4j2b_wpp}}
\caption[Validation of the dijet invariant mass, hadronic and leptonic top mass reconstruction using pp collisions at $\sqrt{s}=8$\,\TeV]{
  Dijet invariant mass (a--c), and hadronic (d--f) and leptonic (g--i) top mass distributions in the 0 (a,\,d,\,g), 1 (b,\,e,\,h), and 2 (c,\,f,\,i) \cPqb-tagged jet categories 
  after all selections, using a higher event count \PW+jets MC sample of pp collisions at 8\,\TeV~\citeAN{AN-17-043}. 
  The curves correspond to the different sorting algorithms as described in Table~\ref{tab:teff}.
}
\label{fig:wtselinwjetspp}
\end{figure}


