\section{The Large Hadron Collider complex }
\label{sec:lhc}


\subsection{Performance and achievements of synchrotron accelerators }
\label{sec:synch}

Within a synchrotron accelerator, charged particles  can gain energy from a high amplitude alternating $\vec{E}$ field 
that is contained in metallic chambers, i.e., radiofrequency (RF) cavities, while they are steered and focused by magnets. 
At relativistic velocities $\upsilon=\beta c$, such an extra factor in the magnetic force makes a large difference in the scale of the needed magnetic compared to electric field: 
A magnetic field of 1\,$\mathrm{T}$ would be equivalent to a gargantuan electric field of $3\times 10^{8}$\,$\mathrm{V m^{-1}}$~\cite{acceleration}.
Producing such an electric field is far beyond current technical limits, meaning magnetic fields are copiously used to steer the beams. 
The physical fundamentals of beam steering and focusing are the so-called ``beam optics.''

Although the advantage of directing two beams of approximately equal energy at each other with respect to fixed-target collisions (regarding usable energy) had been early realized, 
the low particle intensities obtained in accelerators in the late 1950s rendered colliders an impractical option.
This changed in 1957 (Fig.~\ref{hadroncolliders_a}), when the idea of stacking particles into circular accelerators was first put forward with an alternating-gradient accelerator~\cite{altgradient}.
Collider storage rings can in principle be designed for a variety of particle species.

Two main categories of synchrotron accelerators can be distinguished in the field of collider physics, namely hadron and lepton colliders.
For instance, the Large Electron-Positron collider (\texttt{LEP}) was the previous large-scale project at \texttt{CERN}, i.e., 
an electron–positron accelerator that was constructed between 1984 and 1989 and whose ring circumference was almost 27\,$\mathrm{km}$. 
The synchrotron radiation produced from bending the particle trajectory is highly dependent on the mass to charge ratio as
\begin{equation}
P  \propto \frac{E^4 p}{\rho^2m^4}\, ,
\label{eq:synch_radiation}
\end{equation}
where $P$ is the power radiated from the steered particle, $E$ is the energy of the particle, $\rho$ is the radius of curvature of the trajectory, 
and $m$ is the rest mass of the particle. 
The energy loss due to synchrotron radiation was 3\% per turn in the \texttt{LEP} apparatus; were the \texttt{RF} cavities turned off, the beam would be lost only in a fraction of a turn. 
To counteract the energy loss RF cavities were extensively installed, rendering \texttt{LEP} the largest circular lepton collider built so far. 
For the specific case of protons, the emitted radiation is approximately eight orders of magnitude 
lower, for the same beam energy and accelerator size, relative to electrons. 
After \texttt{LEP} was dismantled, the construction of a hadron collider in the same tunnel started in 2001 (Fig.~\ref{hadroncolliders_b}), although 
the main idea for a multi-TeV hadron collider to investigate the origin of mass and to search 
for signs of unification beyond the SM was initially conceived in the late 1970s and early 1980s.

\begin{figure}[tb]
\begin{minipage}{.3\paperwidth}
\centering
\subfloat[]{\label{hadroncolliders_a}\includegraphics[scale=0.2]{figures/apparatus/hadroncolliders.png}}
\end{minipage}%
\begin{minipage}{.45\paperwidth}
\centering
\subfloat[]{\label{hadroncolliders_b}\includegraphics[scale=0.22]{figures/apparatus/LHC_area.png}}
\end{minipage}%
\caption[\texttt{CERN} geographical layout and overview of hadron-hadron colliders]{\label{fig:hadroncolliders} 
  (a)
  Overview of pp and \ppbar\ colliders, their beam energies and major achievements~\cite{lhc_chapter1}.
  The nominal \texttt{LHC} design aims at proton beam energies of 7\,\TeV.
  (b)
  \texttt{LHC} is a two-ring superconducting hadron accelerator. The 26.7\,$\mathrm{km}$ tunnel has eight straight sections and eight arcs,
  and lies between 45 and 170\,$\mathrm{m}$ below the surface on a plane inclined at 1.4\% sloping towards the L\'eman lake~\cite{cern_layout}.  
  Approximately 90\% of its length is immersed in molasse rock, and the remaining 10\% lies in limestone under the Jura mountain.  
  There are two transfer tunnels, each approximately 2.5\,$\mathrm{km}$ long, linking \texttt{LHC} to the \texttt{CERN} accelerator complex that acts as injector.
  }
\end{figure}


The Large Hadron Collider (\texttt{LHC}) is a hadron accelerator with a circumference of 26.7\,$\mathrm{km}$ designed to run at a nominal collision energy of 14\,\TeV, 
approximately 7 times higher than that of the Tevatron collider. The key objective of \texttt{LHC} is 
  the exploration of the SM in the TeV energy range and the search for potential new physics signatures.
The \texttt{LHC} design parameters, as shown in Table~\ref{table:lhc_lep2_tevatron}, had meanwhile evolved significantly over the following years 
in a continuous comparison and competition with the Superconducting Super Collider (\texttt{SSC}) project---nicknamed the ``Desertron''---in the United States 
leading to a tenfold increase of the \texttt{LHC} performance from the first design projections~\cite{lhc_proj} to the nominal performance specification~\cite{lhc_design}.
The maximum achieved energy, or more precisely, the beam momentum is given by the maximum bending field strength, and it follows a fairly clear exponential trend as a function of time, 
as also shown in Fig.~\ref{hadroncolliders_a}. Hadron accelerators can reach higher energies than their lepton counterparts, as Eq.\,(\ref{eq:synch_radiation}) delineates. 
For that reason, they are deemed better suitable to explore new energy regimes, albeit the collisions are intrinsically more complex. 
The use of protons implies that the collisions occur actually between the proton constituents that carry only a fraction of the total proton beam energy. 
Aiming at having at least 10\,TeV collision energy requires proton beam energies well above 1\,\TeV.


\begin{table}[ht]
\makebox[\textwidth][c]{ 
  \footnotesize
  \begin{tabular}{ l|| p{2cm} | p{2.5cm} | p{2cm}  }
   \toprule
   Parameter   & \texttt{LHC} & \texttt{LEP2} & Tevatron  \\
   \midrule
    Colliding species   & p,p & $\mathrm{e^{+}}$,$\mathrm{e^{-}}$ & $\mathrm{p}$,$\mathrm{\overline{p}}$\\
    \midrule
    Dipole field at top energy (T) & 8.33 & 0.11 & 4.4 \\
    Momentum at collisions (TeV) & 7 & 0.1 & 0.98 \\
    \midrule
    %\midrule
    Number of bunches per beam & 2\,808 & 4 & 36 \\
    %\midrule
    Particles per bunch ($\times 10^{11}$) & 1.15 & 4.2 & 2.9, 0.8 \\
    %\midrule
    Typical beam size in the ring ($\mu\mathrm{m}$) & 200-300 & 1\,800/140 (H/V) & 500 \\
    Beam size at IP ($\mu\mathrm{m}$) & 16 & 200/3 (H/V) & 24 \\
    Peak luminosity ($\mathrm{cm}^{-2}\mathrm{s}^{-1}$) & $10^{34}$ & $10^{32}$ & $4.3\times 10^{32}$ \\
    \midrule
    %\midrule
    Fraction of energy lost in synchr.rad. per turn & $10^{-9}$ & 3\%  & $10^{-11}$ \\
    Total current per beam (A) & 0.58 & 0.003 & 0.08 \\
    %\midrule
    Total energy stored in each beam ($\mathrm{MJ}$) & 362 & 0.03 & 0.9 \\
    \bottomrule
  \end{tabular}
}
  \caption[The main \texttt{LHC} parameters compared to \texttt{LEP2} and Tevatron colliders]{ 
    The main \texttt{LHC} parameters are compared to its predecessor during the final operational phase, i.e., \texttt{LEP2}, and Tevatron. 
    The   total   number   of   bunches   in   the   \texttt{LHC}   ring was reduced from the initial 2\,835~\cite{lhc_yellow} to 2808~\cite{lhc_design}, translated  into  
    a  modest  reduction  in luminosity.
    Synchrotron radiation from protons at \texttt{LHC} energies becomes noticeable but is not a limitation.
    The \texttt{LHC} is built with superconducting NbTi magnets that operate at superfluid \ce{^{4}_{2}He+} temperature of 1.9\,$\mathrm{K}$ and allow fields up to 8.33\,$\mathrm{T}$.
    The notations ``H'' and ``V,'' that are used in the beam size, stand for the horizontal and vertical plane, respectively. 
    %The design \texttt{LHC} parameters for the magnetic field and beam intensity are particularly ambitious. Comparison of \texttt{LHC} and \texttt{LEP2} design beam parameters.
  }
\label{table:lhc_lep2_tevatron}
\end{table}

\subsection{Delivering collisions to the seven \texttt{LHC} experiments}
\label{sec:synch}
During the 1992 workshop ``Towards the LHC Experimental Program'' in Evian,
proto-collaborations presented ``expressions of interest'' describing their detector plans~\cite{evian1992}. 
The interest in contributing to the \texttt{LHC} experimental program was enormous: 12 proposals were made in total, out of which four intended for general-purpose experiments, 
three meant for b-flavor and heavy ion physics, respectively, while two additional proposals were targeting to neutrino physics. 
\texttt{LHC} beams could in principle collide at eight points. Four of these coincided with the four big experiments at \texttt{LEP},
 while the remaining four points had to be used for the ``beam-cleaning'' system to ensure high performance by reducing troublesome beam halo. 
Another was reserved for the beam dump where protons can be absorbed, once the circulating beams are no longer required. 

It was clear from the beginning that only two new general-purpose experiments would be accepted at \texttt{LHC}, 
one of them potentially being a toroidal apparatus like the Apparatus with SuperCOnducting Toroids (\texttt{ASCOT}) and the Experiment for Accurate Gamma, Lepton and Energy
measurements (\texttt{EAGLE}). Therefore, these two proto-collaborations, in a voluntary move, merged to form A Toroidal LHC ApparatuS (\texttt{ATLAS}) in 1992. 
Among the remaining general-purpose experiments, \texttt{ATLAS} and Compact Muon Solenoid (\texttt{CMS}) were invited to provide detailed technical proposals by 1994. 
\texttt{ATLAS} and \texttt{CMS} were finally accepted in January 1996, and later approved for construction on 31st of January 1997 with an expenditure 
ceiling of 475\,$\mathrm{MCHF}$~\cite{terascale_chapter21} (1995 currency rate). 
The resulted four big experiments, i.e., \texttt{ALICE}, \texttt{ATLAS}, \texttt{CMS} and \texttt{LHCb}, were followed 
by three smaller, more focused proposals for experiments: The \texttt{TOTEM} experiment (LoI 1997) 
is investigating the total pp cross section, elastic pp scattering and diffraction dissociation; \texttt{MoEDAL} (LoI 1998) is searching 
for magnetic monopoles and other exotic phenomena; \texttt{LHCf} (LoI 2003), finally, uses very forward particles created in pp collisions to simulate cosmic rays.

\begin{figure}[tb]
\centering
\includegraphics[scale=1.5]{figures/apparatus/lhc_pre.png}
\caption[The \texttt{CERN} accelerator complex]{\label{fig:cern} 
  The accelerator complex at \texttt{CERN} is a succession of machines with increasingly higher energies~\cite{BRUNING2012705}; 
  two beams circulate in \texttt{LHC} following  a  clockwise  and an anticlockwise direction. They are accelerated up to the record energy of 6.5\,\TeV. 
  Most of the other accelerators in the chain have their own experimental halls, where their beams are used for experiments at lower energies.
  In addition to accelerating protons, the \texttt{CERN} complex can also accelerate lead ions by first passing them through \texttt{LINAC3} and then \texttt{LEIR} 
  before continuing on the same path as protons.


}
\end{figure}

The acceleration of protons at the \texttt{CERN} complex is performed in stages as they pass through the different accelerators
along the injector chain, shown Fig.~\ref{fig:cern}. The protons are generated by a duoplasmatron source from
which they are extracted with an energy of 100\,$\mathrm{keV}$ and injected into \texttt{LINAC2}\,\footnote{The Linear accelerator 4 (\texttt{Linac4}) 
is designed is scheduled to become the source of proton beams for \texttt{LHC} after the long shutdown 2}, i.e., an 80\,$\mathrm{m}$ long
linear accelerator with an extraction energy of 50\,\MeV. During the acceleration process in the \texttt{LINAC} the
proton beam is also bunched using RF cavities. Once extracted from the \texttt{LINAC} the protons are injected into
the Proton Synchrotron Booster (\texttt{PS Booster}), a 157\,$\mathrm{m}$ circular accelerator complex capable of accelerating protons
up to 1.4\,\GeV, which consists of a stack of four separate rings. From the \texttt{PS Booster} the particles are injected into
the Proton Synchrotron (\texttt{PS}), a 628\,$\mathrm{m}$ ring where they are accelerated up to an energy of 26\,\GeV. All these
machines are installed at ground level. At this point the beams are sent in an underground machine, the Super
Proton Synchrotron (\texttt{SPS}), which is a 6.9\,$\mathrm{km}$ circular accelerator lying 50\,$\mathrm{m}$ under the surface. They are
then accelerated to 450\,\GeV, corresponding to the \texttt{LHC} injection energy. The injection of the \texttt{LHC} beams from the \texttt{SPS} is
done in IR2 for beam 1, going clockwise, and IR8 for beam 2, going counter-clockwise. All these accelerators
are linked through transfer lines which allow for the particles to travel from one machine to the other.

The \texttt{LHC} has also been designed to collide heavy ions that follow a slightly different path relative to protons. 
They are produced from a highly-purified lead sample---enriched with the isotope \ce{^{208}Pb}---heated to a temperature of about 800\degree\,$\mathrm{C}$ 
where the lead vapor is ionized by  an  electron  current. 
The \ce{^{208}_{27}Pb+} ion state is selected and accelerated to a energy of 4.2\,\MeV\ (per nucleon) before it is directed towards a carbon foil, 
the latter stripping most of them to \ce{^{208}_{54}Pb+}. 
The formed \ce{^{208}_{54}Pb+} beam is first accumulated, then accelerated to 72\,\MeV\ in the Low Energy Ion Ring (\texttt{LEIR}), which transfers it to the \texttt{PS} (5.9\,\GeV), 
and finally sent to the \texttt{SPS} (177\,\GeV) fully stripped to \ce{^{208}_{82}Pb+}, and after passing through a second foil. 
The SPS finally injects the beams to the \texttt{LHC}, that accelerates them to the record beam energy of 6.5\,$Z$\,\TeV.

The original design of \texttt{LHC} did not foresee the operation with species other than protons or \ce{^{208}_{82}Pb^{82+}} ions, let alone a mixed particle mode.
However,  the \texttt{CERN} heavy ion injectors and \texttt{LHC}  have demonstrated the feasibility of asymmetric (see Section~\ref{sec:pPb}) and xenon-xenon, i.e., 
a medium-mass nuclear species, collisions. In 2017, high-intensity xenon (\ce{^{129}_{54}Xe+}) beams were produced in the injector chain, and brought into collision 
at a nucleon-nucleon centre-of-mass energy (\rootsNN) of 5.44\,\TeV~\cite{XeXe}. During 6\,$\mathrm{h}$ of stable XeXe collisions about 3\,\invmub were delivered to ATLAS and CMS, 
while fractions of 1\,\invmub were delivered to \texttt{ALICE} and \texttt{LHCb} because of the larger demagnification values.
Such production outcome is comparable to the 10\,\invmub of PbPb collisions delivered per experiment in the first one month-long heavy ion run in 2010~\cite{PbPb_2010}. 
%The XeXe integrated luminosity suffices to a sample large enough for new physics results had been already presented, 
 Since this was probably the last time---at least for several years---that a species other than Pb would be available from the injector complex
valuable data were acquired on the beam-cleaning and collimation efficiency with lighter ions~\cite{XeXe_clean,XeXe_clean_2}.
In 2018, for the very first time, operators injected and accelerated not just atomic nuclei but partially stripped \ce{^{208}_{82}Pb^{81+}} ions into \texttt{LHC}.
 This represents a proof-of-principle test for broadening the present \texttt{CERN} research program making use of such a novel concept of light source~\cite{CERN_egamma}.

\subsection{Fundamental principles of synchrotron accelerators}
\label{sec:acc_principles}

\subsubsection{Transverse particle motion}
\label{sec:vertical_motion}

If  a  particle  is  deflected  in  the presence of a magnetic  field,  the Lorentz and the centripetal force---directed perpendicular to the direction of motion---are always equal.  
Using a Taylor expansion, the magnetic field transverse components, $B_x\hat{x}$ and $B_y\hat{y}$ (Fig.~\ref{sync_coordinate_a}), can be expressed as a function of a dipolar and quadrupolar term:
\begin{equation}
B_x=B_x(0,0)+\frac{\partial B_x}{\partial y}y+\mathcal{O}(y^2)\ 
\label{eq:betax}
\end{equation}
and 
\begin{equation}
B_y=B_y(0,0)+\frac{\partial B_y}{\partial x}x+\mathcal{O}(x^2)\, ,
\label{eq:betay}
\end{equation}
respectively. The total bending angle ($\theta$) of a circular accelerator is $2\pi$, and the integrated dipole field is
\begin{equation}
\int_{s_1}^{s_2} Bdl=\frac{2\pi p}{q}=2\pi B \rho\, .
\label{eq:dipole}
\end{equation}
This defines the curvature ($\rho$) of a particle with charge $q$ and momentum $p$ in the magnetic field of strength $B$, 
from which the so-called ``beam rigidity'' ($B\rho$) can be derived:
\begin{equation}
B[\mathrm{T}]\rho[\mathrm{m}]  \approx \frac{p [\GeV]}{0.3}\, .
\label{eq:rigidity}
\end{equation}
Since synchrotron radiation for hadrons is less of an issue than leptons, the maximum attainable energy in hadron colliders is limited by the magnetic field strengths 
available for bending and focusing the particles on their circular design trajectory. 
In the case of \texttt{LHC}, the magnetic bending radius is determined by the \texttt{LHC} tunnel and it is restricted to $\rho=2\,804$\,$\mathrm{m}$. 
To fulfill the designed proton beam momentum of 7\,\TeV the maximum required magnetic dipole field has to be at least 8.33\,$\mathrm{T}$ (Eq.\,(\ref{eq:rigidity})). 
This relatively high field can only efficiently be provided by superconducting magnets. 
%Warm magnet technology, due to cooling issues of the resistive coils and saturation of iron yokes, is effectively limited to fields up to about 2\,T.
The accelerator lattice is designed to mainly provide bending and focusing fields for a reference particle of a defined particle species.
For particles of the main beam species (monoisotopic case), dispersive effects arise only from momentum offsets. 
For particles of other species, additional dispersive offsets are caused by the different mass and charge with respect to the reference isotopes \ce{^{$A_0$}_{$Z_0$}$\textrm{X}_0$+} 
(see Section~\ref{sec:pPb}).

\begin{figure}[tb]
\begin{minipage}{.45\linewidth}
\centering
\subfloat[]{\label{sync_coordinate_a}\includegraphics[scale=0.60]{figures/apparatus/figures_frenet_serret.png}}
\end{minipage}%
\begin{minipage}{.45\linewidth}
\centering
\subfloat[]{\label{sync_coordinate_b}\includegraphics[scale=0.60]{figures/apparatus/figures_phase_space_ellipse.png}}
\end{minipage}%
\caption[The trajectory coordinates in the Frenet--Serret frame and their motion in phase space]{
  (a) The trajectory coordinates with respect to the Frenet--Serret frame~\cite{CERN_YR228}. 
  An ideal particle is performing transverse oscillations, as indicated by the dotted line, around the design orbit.
  (b) 
  The motion of a single particle, at an arbitrary longitudinal location s, defines an ellipse in phase space $(x,x')$ characterized by the parameters $\alpha$, $\beta$, and $\gamma$~\cite{CERN_YR228}.
  The beam emittance, i.e., the area in the $(x,x')$ space that contains 68.3\% of an ensemble of particles, equals to the area $\pi\varepsilon=\alpha^2$ enclosed by the ellipse of a single particle; 
  the beam width and divergence can be then proven to be $\sqrt{\beta\varepsilon}$ and $\sqrt{\gamma\varepsilon}$, respectively.
}
\label{fig:sync_coordinate}
\end{figure}

To describe the particle trajectories in a synchrotron, the task for solving the equation of motion relative to the design orbit is simplified using a comoving coordinate system, 
known as the ``Frenet--Serret'' coordinate system (Fig.~\ref{sync_coordinate_a}).
Around the accelerator, the focusing properties ($K$) of dipoles and quadrupoles are not constant but depend on $s$.
However, $K(s)$ is periodic with the lattice period ($L$), e.g., $L$  can be the circumference of the accelerator, 
leading to a second order homogeneous differential equation for the transverse motion of a particle in the magnetic structure of an accelerator
\begin{equation}
x''(s)+K(s)x(s)=0\, .
\label{eq:Hill}
\end{equation}
This type of motion with non constant but periodic restoring force is described by the so-called ``Hill equation,''
and its general solution is a quasi-harmonic ``betatron'' oscillation:
\begin{equation}
x(s)=\sqrt{\varepsilon\beta(s)}\mathrm{cos}(\psi(s)+\phi)\, .
\label{eq:Hill_solution}
\end{equation}
The amplitude and phase of the oscillation depend on the exact position in the ring; $\varepsilon$ and $\phi$ are integration constants that  
can be defined from an initial position $x_{0}$ and angle $x'_{0}$ at location $s(0)=s_{0}$ and $\psi(0) = 0$.
The so-called ``beta function,'' $\beta(s)$, is also a periodic function of $L$, $\beta(s+L)=\beta(s)$, and is determined numerically by the focusing properties of the lattice;
the value of the $\beta$ function at the IP is colloquially known as $\beta^{*}$.
The number of betatron oscillations per turn, i.e., the machine ``tune,''
\begin{equation}
\mathrm{Q}=\frac{\psi(s)}{2\pi}=\frac{1}{2\pi}\oint_L \frac{\mathrm{d}s}{\beta(s)}\, ,
\label{eq:tune}
\end{equation}
is of great importance for beam stability, since it regulates the orbit response around the ring for dipole field (terms of $\mathrm{sin}(\pi \mathrm{Q})$) and gradient (terms of $\mathrm{sin}(2\pi \mathrm{Q})$) errors.
With the trajectory at any point of the ring expressed in terms of position and angle, $\varepsilon$ can be calculated at any point as
\begin{equation}
\varepsilon=\gamma(s)x(s)^2+2\alpha(s)x'(s)+\beta(s)x'(s)^2\, ,
\label{eq:ellipsis}
\end{equation}
where two commonly used functions
\begin{equation}
\alpha(s)=-\frac{1}{2}\frac{\partial \beta(s)}{\partial s}\, ,
\label{eq:alpha}
\end{equation}
and
\begin{equation}
\gamma(s)=\frac{1+\alpha(s)^2}{\beta(s)}\, ,
\label{eq:gamma}
\end{equation}
have been introduced.
The quantities $\beta(s)$, $\alpha(s)$ and $\gamma(s)$ are the so-called ``Twiss parameters,'' also referred to as the optical functions, since 
they are defined by the magnetic lattice of the machine that transforms the beam equivalently to a lattice of lenses in classical optics. 
The evolution of an initial set of Twiss parameters is equivalent to the transformation of the particle coordinates described by transfer matrices for the individual beamline elements of 
length $L$ and strength $K$, i.e., the transfer matrices of a drift space ($\mathcal{M}_\mathrm{D}, K = 0$), a focusing ($\mathcal{M}_\mathrm{Q,f} , K > 0$) and defocusing ($\mathcal{M}_\mathrm{Q,d}, K < 0$) quadrupole.

The parametric representation in the $(x,x')$ space (Fig.~\ref{sync_coordinate_b}) defines a constant of motion in the absence of nonconservative forces; 
the area of the ellipse $\pi \varepsilon$ is a phase space invariant. 
This constant describes the amplitude of a single particle, and it can be further generalized to an intrinsic beam property. 
Assuming a Gaussian profile for the particle distribution in the accelerator, normalized to unity, the area in the $(x,x')$ space that contains 68.3\% of 
an ensemble of particles, i.e., the ``beam emittance'' (Fig.~\ref{fig:emit_CMS}), is defined as
\begin{equation}
\epsilon_{x}=\langle x^2 \rangle\langle x'^2 \rangle-\langle xx' \rangle^2\, 
\label{eq:emittance}
\end{equation}
such that the standard deviation of the Gaussian distribution corresponds to
\begin{equation}
\sigma_{x}=\langle x^2 \rangle=\sqrt{\epsilon \beta_{x}}\ 
\label{eq:sigma_x}
\end{equation}
and 
\begin{equation}
\sigma_{x'}=\langle x'^2 \rangle=\sqrt{\epsilon \beta_{x}\gamma_{x}}\ 
\label{eq:sigma_xprime}
\end{equation}
in $x$ and $x'$, respectively. 
In Eq.\,(\ref{eq:emittance}) the symbols $\langle\rangle$ denote the expectation value over the considered ensemble, while $\sigma_{x}$ and $\sigma_{x'}$ are referred to as the beam size and divergence, respectively. 

\begin{figure}[!ht]
\centering
\includegraphics[width=0.6\columnwidth]{figures/apparatus/Ev_emitanceX_Y.pdf}
\caption[Example of the measured beam emittance using the convolved beam widths from \texttt{CMS}]{\label{fig:emit_CMS} 
  The (normalized) beam emittance values in the horizontal and transverse plane per bunch crossing identification number calculated based on the model of Ref.~\cite{lpc_lumi}, and using 
  as inputs the convolved beam widths measured by \texttt{CMS}~\cite{cms_emit}. The bunch length values are the published entries in the logging database from \texttt{LHC}, 
  whereas the nominal values of the crossing angle,the relativistic factor, and $\beta^{*}$ are used.
}
\end{figure}

\subsubsection{Longitudinal particle motion}
\label{sec:longitudinal_motion}

Charged particles are traveling on two design orbits in opposite directions and are colliding at the interaction point(s) (IP(s)).
The  RF  system  of a synchrotron is  required  to operate  in  three  different  modes~\cite{rf_desc}.  
\textit{It  must  capture  and  accumulate  in  stationary buckets  (closed  trajectories  in  longitudinal  phase  space)  successive  groups  of
bunches  from  its  injector;  accelerate  these  bunches  in  moving  buckets  up  to  the
design  energy;  and  finally  store  them,  for  several  hours,  while  maintaining  a
minimum  ratio  of  bunch-to-bucket   area.  In  the  storage  mode,  it  provides  the
nominal  energy  gain  per  turn  required  to  make  up  for  synchrotron   radiation
losses   and   for   the  power   loss  due   to   voltages   induced   in   the   impedance
presented  to  the  beam  by  the  vacuum  chamber  and  accelerating  structures}.
More specifically, the angular radiofrequency, $\omega_\textrm{RF}$, needs to be synchronous to the angular revolution frequency, $\omega_{\textrm{r}}$. 
Synchronous particles need to repeatedly arrive at the cavity with the same phase. This implies that $\omega_\textrm{RF}$ has to be an integer multiple of $\omega_{\textrm{r}}$
\begin{equation}
\omega_\textrm{RF}=h\omega_{\textrm{r}}\, ,
\label{eq:t_gain}
\end{equation}
where $h$ is  an  integer  and  is  called  the ``harmonic  number.''   As  a  consequence,  the  number  of  stable synchronous particle locations equals the harmonic number $h$; 
they are equidistantly spaced around the circumference of the accelerator. 
All synchronous particles have the same nominal energy and follow the nominal trajectory.

For synchrotron accelerators, the metric that regulates the stability of the longitudinal motion is the scale of the so-called ``transition energy.''
This energy is a property of the transverse lattice and is linked to the stability of the revolution frequency, $\omega_{\textrm{r}}$, for particles with a small momentum deviation relative to the synchronous particle; the 
momentum-dependent change of velocity and path-length compensate each other at the transition energy.
Yet, the change of revolution frequency with momentum offsets is opposite below and above transition, featuring a confining range for stable longitudinal oscillations, as described in the following.

\subsubsection{Dispersion effects in longitudinal and transverse dynamics}
\label{sec:longitudinal_motion}

If a particle is slightly shifted in momentum relative to a synchronous particle, it will have a different velocity and orbit, hence orbit length. 
The ``momentum compaction'' parameter quantifies the relative change in orbit length, $\Delta L/L$, with momentum, i.e., 
\begin{equation}
\alpha_\textrm{C}=\frac{\Delta L/L}{\Delta p/p}\, .
\label{eq:mom_compaction}
\end{equation}
At the transition energy there is no change of the revolution frequency for particles with a small momentum deviation. Below or above this critical point the longitudinal oscillations are governed by the so-called ``slip factor,'' $\eta$, which is given by
\begin{equation}
\eta=\left( \frac{1}{\gamma^2}-\alpha_\textrm{C}\right)\, .
\label{eq:slip_factor}
\end{equation}

From the definition of $\eta$, it is clear that an increase in momentum is linked to the opposing effects of
\begin{enumerate}[topsep=0pt]
\item a higher revolution frequency below the transition energy ($\eta>0$) resulting to an increase of velocity 
\item a lower revolution frequency above the transition energy ($\eta<0$) resulting to an increase of path length.
\end{enumerate}
Since the change in revolution frequency with momentum is opposite below and above transition, this is reflected on the range for stable oscillations. 
For small phase deviations, $\Delta\phi$, with respect to the synchronous particles, the conditions for longitudinal stability can be derived assuming 
constant $R_{\textrm{s}},p_{\textrm{s}},\eta, \omega_{\textrm{r,\,s}}$ parameters
\begin{equation}
\frac{\mathrm{d^2}\phi}{\mathrm{d}t^2} + \frac{\Omega^2_{\textrm{r,\,s}}}{\Delta\phi}=0\, .
\label{eq:synch_oscillation}
\end{equation}
Eq.\,(\ref{eq:synch_oscillation}) corresponds to an harmonic oscillator equation of motion with
\begin{equation}
\Omega_{\textrm{r,\,s}} = \frac{h\eta\omega_{\textrm{r,\,s}}qV_0\mathrm{cos}\phi_{\textrm{s}}}{2\pi R_{\textrm{s}}p_{\textrm{s}}}
\label{eq:synch_freq}
\end{equation}
the so-called ``synchrotron angular frequency.''
Stability is obtained when $\Omega_{\textrm{r,\,s}}$ is a real number so that $\Omega^2_{\textrm{r,\,s}}>0$.  
Since most of the terms in Eq.\,(\ref{eq:synch_freq}) are positive, this finally reduces to
\begin{equation}
\eta\mathrm{cos}\phi_{\textrm{s}}>0\, ,
\label{eq:synch_condition}
\end{equation}
rendering the stable region for the synchronous phase dependent on the energy with respect to the transition energy (Fig.~\ref{fig:separatrix}). 
%The conditions for stability are summarized in.

 In the case of \texttt{LHC}, the momentum compaction factor is of the order of $10^{-4}$, and the zero-crossing condition of Eq.\,(\ref{eq:slip_factor})
implies $\gamma_\textrm{Tr}=\frac{1}{\alpha_\textrm{C,Tr}}\sim 100$ meaning LHC operates always above transition.
The cavities at \texttt{LHC} operate at $f_{\textrm{RF}}=400$\,$\mathrm{MHz}$ that is a multiple of the revolution frequency $f_{\textrm{r}}=11\,245.5$\,$\mathrm{Hz}$~\cite{PbPb_2011} with an harmonic 
number $h=35\,640$ (Eq.\,(\ref{eq:t_gain})). 
The harmonic number defines the maximum number of longitudinally stable regions, referred to as ``buckets,'' that can be used to capture, store and accelerate particle species. 
The collectivity of those species inside a bucket is called ``bunch.''
At \texttt{LHC} only about every tenth bucket will nominally capture a bunch, which accordingly corresponds to a bunch spacing of about 25\,$\mathrm{ns}$.

\begin{figure}[tb]
\centering
\includegraphics[width=0.6\columnwidth]{figures/apparatus/separatrix.png}
\caption[The separatrix and the RF bucket within a longitudinal electric field]{\label{fig:separatrix} 
 The applied longitudinal electric field as a function of the particle phase $\phi$. 
 Particles that arrive later (1) than the reference particle receive a larger energy transfer, while those arriving before the reference particle (2) receive a smaller energy kick.
 The restoring force vanishes when $\phi$ approaches $\phi$-$\phi_{\textrm{s}}$, and becomes nonrestoring either above or below the transition point (Eq.\,(\ref{eq:slip_factor})). 
 This phase space trajectory---the separatrix---separates the region of stable motion from the unstable region, contrary to the area within the separatrix---the RF bucket---which 
 corresponds to the maximum acceptance in phase space for a stable motion~\cite{CERN_YR228}.
}
\end{figure}

\subsection{LHC layout and global structure }
\label{sec:lhc_layout}


A circular collider project can follow two distinct design options: either the collider features collisions between particles and antiparticles with opposite charges, 
allowing an efficient accelerator design where both beams share the same vacuum chamber and magnetic elements, e.g., in \texttt{LEP}, $\mathrm{Sp\overline{p}S}$ and Tevatron colliders, 
or the collider features separate vacuum and magnet systems for the two counter-rotating beams, e.g., in the ISR or RHIC collider, allowing collisions between 
a wider range of particle species and larger number of bunches resulting in higher luminosity. 
Hadron colliders relying on particle and antiparticle collisions are intrinsically limited by the rate at which antiparticles can be generated. 
The high-luminosity requirements  at \texttt{LHC}, i.e., luminosity in excess of $10^{34}$\,$\mathrm{cm}^{-2}\mathrm{s}^{-1}$, exclude the use of $\mathrm{\overline{p}}$, 
and hence require a two-ring design with separate magnet and vacuum systems for the two counter-rotating beams.

\begin{figure}[!ht]
\begin{minipage}{.45\linewidth}
\centering
\subfloat[]{\label{lhc_layout_dipole:a}\includegraphics[width=0.3\paperwidth]{figures/apparatus/lhc_layout_v2.png}}
\end{minipage}%
\begin{minipage}{.45\linewidth}
\centering
\subfloat[]{\label{lhc_layout_dipole:b}\includegraphics[width=0.3\paperwidth]{figures/apparatus/lhc_dipole.jpg}}
\end{minipage}%
\caption[Schematic layout of the \texttt{LHC} and its dipole magnet cryostat]{
  (a) Schematic layout of the \texttt{LHC} with its eight straight sections and two-ring design~\cite{BRUNING2012705}.
  There are four experimental insertions of similar design (IR1, 2, 5 and 8) that allow particles of the same (proton or nucleus) 
  or unequal (proton and nucleus) charge to be focused and eventually collide.
  The other four long straight sections are used for collimation (IR3 and 7), acceleration using RF cavities (IR4) and beam extraction (IR6).
  The two counter-rotating beams are conventionally called beam 1 and 2, which circulate in clockwise and counter-clockwise directions, respectively.
  (b) Schematic cross section of the \texttt{LHC} dipole magnet cryostat with its two separate vacuum chambers and magnet coils for the two 
  counter-rotating beams, and the common infrastructure for the powering and cooling of the magnet~\cite{BRUNINGNATURE}.
}
\label{fig:lhc_layout_dipole}
\end{figure}

The \texttt{LHC} apparatus is designed as a two ring-like accelerator bearing an eightfold symmetry (Fig.~\ref{lhc_layout_dipole:a}) with separate
magnet fields and beam chambers, and with common straight sections intercepted by the experimental caverns, where the beams collide.
The interaction regions (IRs) consist of separation dipole magnets (not highlighted in Fig.~\ref{lhc_layout_dipole:a}) and main quadrupoles left and right 
of the interaction point (IP), i.e., the symmetry point of the IR. 
Three magnets on either side of the IP, the triplets (\texttt{T}), are situated only in the common region and are used for the final focusing. 
The triplets thus affect both beams, whereas the rest of the quadrupoles act separately on each beam.
The remaining matching section (\texttt{MS}) and the dispersion suppressor (\texttt{DS}) consist of magnets with separate beam pipes for each ring. 
\texttt{LHC} consists of a total of 9\,593 superconducting magnets out of which 1\,232 are dipoles of about 15\,$\mathrm{m}$ long equally shared over the eight arcs, and 392 main quadrupoles. 
Because of spatial restrictions imposed by the same tunnel that hosted \texttt{LEP}, it was thought beneficial \texttt{LHC} to use twin magnets, i.e., 
two sets of coils and beam channels sharing the mechanical structure and cryostat, as illustrated in Fig.~\ref{lhc_layout_dipole:b}. 


The IRs host the RF cavities (IR4), which accelerates the beams and keeps them bunched, 
the beam dump (IR6) system, used for extraction from the \texttt{LHC}, and cleaning devices (IR3 and 7), which are critical for the machine protection~\cite{collimation}. 
The  two  beams counter-rotate around the ring in two separate beam pipes; beam 1 is injected at IR2 and circulates clockwise, while beam 2 is injected at IR8 
and travels counter-clockwise. To provide collisions and data of high quality to the experiments with the desired rates the beam parameters have to be precisely controlled, e.g., Refs.~\cite{instrumentation_overview,instrumentation_status}.

The  arcs (Fig.~\ref{lhc_arc_IR:a})  are separated by IRs and extend over most of the length of each 3.3\,$\mathrm{km}$ long sectors.
They are built by periodically repeating 23 times a common lattice, the so-called ``FODO'' cell (107\,$\mathrm{m}$ long), that is composed of a horizontally 
focusing quadrupole (\texttt{MQ}), three dipoles (\texttt{MB}), a vertically focusing quadrupole and another three dipoles~\cite{lhc_design}. 
The main bending magnets and quadrupoles are further equipped with sextupoles (\texttt{MS}), octupoles (\texttt{MO}), higher order (\texttt{MCO}) and orbit correctors (\texttt{MCB}) 
for adjustments of the various beam parameters around the ring. 
The magnets, equipped with their helium vessel and end covers, constitute the ``cold masses,'' which, in normal operation, contain superfluid helium at 1.9\,$\mathrm{K}$ and 0.13\,$\mathrm{MPa}$, 
and are thermally insulated from the vacuum enclosure. 

\begin{figure}[!ht]
\begin{minipage}{.99\linewidth}
\centering
\subfloat[]{\label{lhc_arc_IR:a}\includegraphics[scale=0.25]{figures/apparatus/LHC_arc.png}}
\end{minipage}%
\\
\begin{minipage}{.99\linewidth}
\centering
\subfloat[]{\label{lhc_arc_IR:b}\includegraphics[scale=0.25]{figures/apparatus/Focus_IP.png}}
\end{minipage}%
\caption[Schematic layout of the \texttt{LHC} lattice structure and IR5 insertion]{
  (a) Schematic layout of the \texttt{LHC} lattice structure.
  Each element installed in the LHC tunnel has its individual identification, constructed following a special convention as part of the \texttt{LHC} methodical accelerator design (MAD) 
  sequence~\cite{CERN-AB-2003-024}.
  (b) 
  The low values of the $\beta$ function at the IPs of the experiments are generated with the help of a triplet quadrupole assembly (\texttt{Q1} to \texttt{Q3}). 
  The free space around the IPs that is reserved for the experiments is $\pm 23$\,$\mathrm{m}$~\cite{BRUNING2012705}.
  The two rings share the same vacuum chamber, the same low-$\beta$ triplet (superconducting) magnets, 
  and the \texttt{D1} separation dipole (warm) magnets. 
  The remaining region is comprised of a second dipole (\texttt{D2}), a superconducting magnet operating at a cryogenic temperature due to lower radiation levels, 
  and four matching quadrupole magnets (\texttt{Q4} to \texttt{Q7}, the latter not highlighted).
}
\label{fig:lhc_arc_IR}
\end{figure}


The \texttt{LHC} has four IRs of very similar design but with different beam optics: 
two diametrically opposite insertions at IR1 (IR5) for ``high-''luminosity collisions in \texttt{ATLAS} (\texttt{CMS}),
and ``medium-'' and ``low-''luminosity operation at IR8 (\texttt{LHCb}) and 2 (\texttt{ALICE} ), respectively. 
The lower luminosity operation should be only considered as a relative term, e.g., the \texttt{LHCb} experiment nominally obtains about the same luminosity as 
the \texttt{CDF} and \texttt{D{\O}} detectors at Tevatron (Table~\ref{table:lhc_lep2_tevatron}).
All four experimental IRs have similar designs and are equipped with dipoles and main quadrupoles 
symmetrically (asymmetrically) placed left and right of the IPs (IP8), as shown in Fig.~\ref{lhc_arc_IR:b} for IR5. 

More specifically, the beam separation is steered by six dipole orbit corrector magnets per plane and beam. 
Two of them (\texttt{D2} in Fig.~\ref{lhc_arc_IR:b}) are located further away from the IP and act on each beam separately. They are used for fine-tuning of the offsets. 
The other magnets are installed closer to the IP and affect both beams at the same time. 
By creating a local distortion in the beam orbit, which is called ``closed orbit bump,'' the \texttt{D1} magnets are able to steer the beam separation---the cornerstone of the absolute luminosity calibration scans---and generate a crossing angle ($\theta_\textrm{C}$) between the colliding bunches. 
The values of $\theta_\textrm{C}$, that lies in the vertical (horizontal) plane for IP1 and 2 (IP5 and 8) to minimize the impact of beam--beam effects 
(see Section~\ref{sec:beambeam}), can be chosen large enough to minimize parasitic long-range encounters since the beams traverse a common vacuum chamber for about 120\,$\mathrm{m}$. 


\subsubsection{LHC filling schemes and magnetic cycle}
\label{sec:cycle}
As discussed in Section~\ref{sec:longitudinal_motion}, there is a chain of 35\,640 RF buckets around \texttt{LHC} which could potentially be filled with bunches. 
In the nominal pp filling scheme, bunches are spaced by 25\,$\mathrm{ns}$, meaning 3\,564 potential slots are available, each given a unique bunch crossing identifier (BCID).
The injection from \texttt{SPS} has a bunch train structure, i.e., a specific number of equally spaced bunches. 
Between the trains, short gaps for the injection kicker magnets must be accounted for, and a 3\,$\mu\mathrm{s}$ abort gap is kept free for a safe abort of the \texttt{LHC} beam. 
In general, not all bunches are ``paired,'' i.e., colliding in both beams in the same BCID, but different groups of BCIDs can be defined, e.g., ``empty'' BCIDs without a proton or ion bunch. 
By convention, the first BCID after the abort gap is numbered as 1. 

A schematic view of the bunch distribution for the nominal pp filling scheme is given in Fig.~\ref{filling_LHC_a}. 
The scheme involves 12 injections from \texttt{SPS} with each cycle consisting of either three or four batches from the pre-injectors. 
The order of \texttt{SPS} cycles is \textit{333\,\,334\,\,334\,\,334} meaning  that  the  first  \texttt{SPS}  cycle  injected  into  \texttt{LHC}  
contains $3 \times 72$ batches  from  the  pre-injectors,  while the  6th, 9th  and  the last  contain  four  batches.  
With  each  \texttt{SPS}  cycle  taking 21.6\,$\mathrm{s}$ to complete, filling of each \texttt{LHC} ring takes around 4\,$\mathrm{min}$.
In practice, \texttt{LHC} is flexible to operate with several different filling schemes meant for various purposes. 
In addition to the nominal 25\,$\mathrm{ns}$ spacing for protons, the bunch-splitting in \texttt{PS} allows different bunch spacing: 50\,$\mathrm{ns}$ (physics operation for Run 1 and part of Run 2 but with beams 
of reduced intensity and transverse emittance), 75\,$\mathrm{ns}$ (initial period of physics operation) and greater than 75\,$\mathrm{ns}$ (very first machine commissioning). 

The nominal filling scheme for ions is optimized to cope with space charge problems in the injector chain. 
The baseline scheme has been based on 100\,$\mathrm{ns}$ bunch spacing, although relatively  few  bunches  are  actually  at  this  spacing. 
Figure~\ref{filling_LHC_b}  illustrates  such a  rather  complex  scheme; there  are  891  possible  bunch  positions  in  \texttt{LHC} with  a  total  of  592  filled  bunches. 
In that case, bunches are fabricated four at a time in the pre-injectors. The  \texttt{SPS}  constructs trains by taking several set of bunches injected 
with a spacing of 225\,$\mathrm{ns}$. These are then accelerated and delivered to the \texttt{LHC}. The order of \texttt{SPS} cycles is: \textit{8\,13\,13\,\,12\,13\,13\,\,12\,13\,13\,\,12\,13\,13}. 
Hence,  the  majority  of  \texttt{SPS}  cycles  involves  13  sets  of  four  bunches  transferred  from  the  pre-injectors to the \texttt{LHC}. 
As there are a large number of injections per \texttt{SPS} cycle, each cycle lasts about 54\,$\mathrm{s}$. With 12 \texttt{SPS} cycles it will take around 10\,$\mathrm{min}$ to fill one \texttt{LHC} ring.

\begin{figure}[!ht]
\centering
\subfloat[]{\label{filling_LHC_a}\includegraphics[width=0.35\paperwidth]{figures/apparatus/25ns_scheme.png}}
\centering
\subfloat[]{\label{filling_LHC_b}\includegraphics[width=0.35\paperwidth]{figures/apparatus/100ns_scheme.png}}
\caption[Schematic of the nominal bunch disposition around \texttt{LHC} for protons and ions]{\label{fig:filling_LHC} 
Schematic of the nominal bunch disposition around one \texttt{LHC} ring for the 25\,$\mathrm{ns}$ proton (a) and 100\,$\mathrm{ns}$ heavy ion (b) filling schemes; ``b'' indicates a position 
with beam and ``e'' indicates an empty bunch~\cite{filling_LHC}.
}
\end{figure}

The evolution of the main beam parameters during a physics Fill is illustrated in Fig.~\ref{fig:cycle}. 
Note that the luminosity is just plotted for illustration since the beams are kept separated until the end of the ``squeeze.'' 
The beams are injected at large $\beta^{*}$ values and then are ``ramped'' to high energy before squeezed at IP1 and 5 to a smaller $\beta^{*}$. 
During ramping the longitudinal momentum of the particles increases, while the transverse component is left unchanged, leading to a reduction of the transverse
emittance. A smaller beam emittance, in turn, leads to a higher luminosity, while during the squeeze $\beta^{*}$ gets decreased, increasing luminosity production further.
Measurements indicated that the emittances are typically preserved during the \texttt{LHC} ramp and squeeze, whereas possible increasing and decreasing emittances 
during the same stages are mainly caused by a nonmonotonically changing $\beta$ function. 
The knowledge of the latter is thus required in order to measure the beam emittance. 
Differences in time spent for each stage relative to Run 1 are fairly  small~\cite{nominal_cycle_LHC},  which  is  an  indication  that  the long shutdown 1 did not affect 
the operational performance  of \texttt{LHC}. 
The   feasibility   of   combining   the  energy  ramp  with  the  betatron  squeeze  has been technically proved during a machine development study~\cite{CERN-ACC-NOTE-2015-0023}, 
while this method was first applied to the combined 2.51\,\TeV\ ramp with a squeeze  to $\beta^{*}=4$\,$\mathrm{m}$ in November 2015. 
Merging  the two  operations has resulted in  a  considerable gain (up  to about 10\,$\mathrm{min}$) for each magnetic cycle, reducing the \texttt{LHC} turnaround, i.e., 
the time  needed  to  establish  physics (``stable  beams'') conditions after a ``beam dump'' has occurred.
  
\begin{figure}[tb]
\centering
\includegraphics[scale=0.3]{figures/apparatus/magnetic_cycle.png}
\caption[The magnetic cycle of \texttt{LHC}]{\label{fig:cycle} 
  The \texttt{LHC} operation cycle is a defined protocol that ensures safe operation, and is characterized from the dipole magnetic field and dipole magnet current, 
  that both receive their maximum values of 8.33\,$\mathrm{T}$ and 11.7\,$\mathrm{kA}$, respectively, at physics conditions.
  The associated evolution of the nominal beam parameters (energy, luminosity, and beam size, the latter as defined in Eq.\,(\ref{eq:sigma_x}) assuming $\epsilon_\textrm{N}=3.75$\,\mum) 
  is also shown for the different operation stages (here simplified), whose relative duration is indicated by the double-headed arrows.
  The luminosity represents the maximum achievable pp luminosity, in case the beams were to collide head-on at IP1 and 5.
}
\end{figure}

\subsection{Proton-nucleus collisions at \texttt{LHC}}
\label{sec:pPb}

In  the  first  stage (2010--2011~\cite{PbPb_2011})  of  its  heavy ion  program,  \texttt{LHC}  
collided  lead (\ce{^{208}_{82}Pb+}) nuclei  to  study  the  properties  of  hot 
nuclear  matter  at  extreme  energy  densities.  The  second  
stage  of  this  program (2011--2013~\cite{pPb_2013_alt}),  following  the  pattern  at  the Relativistic Heavy Ion Collider (\texttt{RHIC}),  
was  expected  to  be  a  crucial  control  experiment  in  which  the nuclear  matter  is  instead studied  by  colliding  lead  ions  with protons; 
deuterons offer no particular advantage and are not readily available at \texttt{LHC}.   
At  first  sight,  pPb  collisions at \texttt{LHC} thus appear analogous to the successful dAu collisions at \texttt{RHIC}. 
For hybrid collisions to properly occur at IPs though, it is necessary  to  inject  a  proton  beam  with  the  same  bunch  
pattern  as a typical  ion  beam, meaning that the  existing  p  and  Pb injector  chains  have to
to operate  in  tandem efficiently; the  two-in-one  magnet  design   of  \texttt{LHC}  imposes   different  revolution  frequencies  for  the  two  different mass-to-charge-ratio species; 
beam   dynamics  and  the   potential  performance  of  \texttt{LHC} should be reevaluated.

The multistage collimation system~\cite{collimation_2} is designed to intercept protons at large amplitudes relative to primary collimators that scatter them into secondary and tertiary collimators, 
where they should eventually be absorbed. In heavy ion operation, although the stored beam energy is considerably smaller than for protons, the collimation 
is less efficient due to the yield of effectively off-momentum ion fragments in the primary collimators. 
These fragments with different magnetic rigidity can be further subject to fragmentation processes, and hence a large variety of different ion types 
can be produced through the interaction with the collimator material. 
Owing to the large cross sections for the involved fragmentation processes (few hundred Barns~\cite{hermes}), it is crucial to measure and understand ion loss patterns. 

The cleaning performance is described by a local inefficiency (``loss map'') which is the ratio of the number of lost particles at  any  location  
of  the  ring  in  a  given  length  over the total number of lost particles. 
The required local cleaning inefficiency at 7\,\TeV\ is about $10^{-5}$\,$\mathrm{m}^{-1}$ for the nominal intensity~\cite{coll_eff}. 
Figure~\ref{coll_rigid:a} displays exemplary loss maps for \ce{^{208}_{82}Pb+}~\cite{coll_Pb} and p~\cite{coll_p} beams at $3.5$\,$Z$\,\TeV\
having almost identical collimator settings and optics. Both distributions are dominated by losses in the betatron collimators at IR7, 
followed by the momentum collimators in IR3 and the dump protection devices in IR6. 
The different loss patterns in the IR2 region is due to the different optical configuration used in the two measurements. 

\begin{figure}[!ht]
\centering
\subfloat[]{\label{coll_rigid:a}\includegraphics[width=0.35\paperwidth]{figures/apparatus/collimation_eff.png}}
\centering
\subfloat[]{\label{coll_rigid:b}\includegraphics[width=0.35\paperwidth]{figures/apparatus/rigidity_shift.png}}
\caption[Qualification loss maps with proton and lead beams, and the orbit displacement]{\label{fig:coll_rigid}
(a) The representation of the local cleaning inefficiency as a function of the longitudinal position is referred to as a ``loss map.''
All Beam Loss Monitoring signals are cleaned from the background and normalized to the highest signal. The vertical dashed lines mark the \texttt{LHC} octants.
(b) Proton  momentum, $p_{\textrm{p}}$, shift  required  to  equalize p and Pb revolution frequencies during the energy ramp. 
For $p_{\textrm{p}} \lesssim 2.7 $\,\TeV,   the   orbit  displacement exceeds the normally accepted limits at \texttt{LHC}, imposing a lower limit on collision energy for pPb collisions. 
Injection and part of the ramp must be thus performed  with  unlocked RF systems  and  different  revolution  frequencies   for   the   two   beams~\cite{rigidity}.
}
\end{figure}

Following  the  high  integrated  luminosity  accumulated at $\rootsNN=2.76$\,\TeV in the first two PbPb collision runs in 2010~\cite{PbPb_2010} and 2011~\cite{PbPb_2011}, respectively, the 
\texttt{LHC}  heavy ion  physics  community  requested  a  first  run  with proton-nucleus collisions. This almost unprecedented mode of 
collider operation had not been foreseen in the baseline design 
of  the  \texttt{LHC}  whose  two-in-one  magnet  design (see Section~\ref{sec:lhc_layout}) imposes equal rigidity, and hence unequal revolution frequencies per se for asymmetric colliding species. 

Equal-rigidity configuration using deuteron-Gold collisions at \texttt{RHIC} has been considered in the initial injection setup, 
taking advantage of the two independent rings each comprised of single aperture  magnets. 
However, in  such  condition, it has been later realized that unequal frequencies  between  the  two  beams modulate long-range  beam--beam  forces  creating  
untunable  beam  losses  during the  acceleration ramp. 

To equalize the RF  frequencies at \texttt{LHC}, and hence  allow the beams to encounter each other at the  same  position  every  turn, implies  
that  the  lead  ion  has  to  move  to  the inside  of  the  ring  to  compensate  for  being  slower,  and equivalently,  the  proton  beam  
has  to  move  to  the  outside of  the  other ring to  travel  a  larger  distance  to  compensate  for being  faster (Fig.~\ref{coll_rigid:b}).  
The  amount  of  orbit  shift  depends  on  the beam energy, and for instance, at  injection  energy, each  beam  should  have been moved  outside  the  reference  orbit  by about 70\,$\mathrm{mm}$,  
clearly  out  of  reach   given  the  dimensions   of   the   beam   pipe. 
This difficult exercise,  called  ``RF  frequency  lock  and  cogging,'' had therefore to be performed at flat top energies, 
and doubts  were long  extant  as  to  whether \texttt{LHC}  would  ever  deliver collisions with asymmetric species. 

\begin{table}[htbp]
\centering
\caption[Comparison of the \texttt{LHC} design and achieved beam parameters for proton and \ce{^{208}_{82}Pb+} beams]{\label{tab:design_vs_achieved} 
Beam-related parameters in various \texttt{LHC} running periods. Since asymmetric collisions were not included in the \texttt{LHC} design, the physics case~\cite{pPb_opportunities} was based 
on a luminosity  $\mathcal{L}_{\textrm{peak}}=1.15 \times 10^{29}$\,$\mathrm{cm}^{-2}\mathrm{s}^{-1}$ at a beam energy of 7\,Z\,\TeV.
We can thus refer to these values as the ``design'' parameters, similarly to those in pp and PbPb collisions. 
%The 2015 operational period with heavy ions started with a reference proton run at 2.51\,\TeV\ to obtain the same center-of-mass energy as in the pPb run of 2013. 
%For the same reason, the ensuing PbPb operation was carried out at an energy of 6.37\,Z\,\TeV.
Relevant quantities refer to IP1 and 5, while the values of $\mathcal{L}_{\textrm{peak}}$ correspond to conditions of stable  beams.
The final performance values in the 2018 PbPb run are under evaluation.
}
\resizebox{\textwidth}{!}{\begin{tabular}{cc||cc|cc|cc|cc}
\toprule
\multicolumn{2}{c||}{\multirow{2}{*}{Parameter}} & \multicolumn{2}{c|}{Design} & \multicolumn{2}{c|}{PbPb} & \multicolumn{2}{c|}{pPb} & \multicolumn{2}{c}{pp} \\\cline{3-10}
          & & \multicolumn{1}{c}{~~PbPb~~} &\multicolumn{1}{c|}{~~pp~~}  & \multicolumn{1}{c}{~~2011~~} &\multicolumn{1}{c|}{~~2015~~}
 &\multicolumn{1}{c}{~~2013~~} &\multicolumn{1}{c|}{~~2016~~} &\multicolumn{1}{c}{~~2015~~} &\multicolumn{1}{c}{~~2018~~}
\\
\midrule
    $E$ &(TeV) &  7\,$Z$ & 7  &  3.5\,$Z$  & 6.37\,$Z$   &   4\,$Z$ & 6.5\,$Z$  & 6.5 & 6.5     \\
    $n_\textrm{b}$ &  &  592 & 2808   & 358 & 518   &   338 & 540  & 2244 & 2556      \\
    $N_\textrm{\ce{^{208}_{82}Pb+},\,p}$ & ($\times 10^{8},\,\times 10^{11}$) &  0.7  &  1.15 & 1.2 & 1.9   &   1.4 & 2.1  & 1.1 & 1.4      \\
    $\varepsilon_{\textrm{N}}$ & (\mum) &  1.5   &  3.75 & 2.0 & 2.1   &  2.0  & 1.6         & 3.5 & 2.2   \\
    $\beta^{*}$ & (m)     &  0.5 & 0.55  &  0.1 &  0.8   &  0.8 & 0.6 & 0.8 & 0.25 \\
                            $E_{\textrm{S}}$ & (MJ)  &  3.81 & 360  &  1.98 & 8.60 &   2.77 & 9.70 &  277 & 320  \\
    $\mathcal{L}_{\textrm{peak}}$ & ($\times 10^{27},\,\times 10^{34}$\,$\mathrm{cm}^{-2}\mathrm{s}^{-1}$)      &  1   &  1 & 0.5 & 3   &  116 & 850   & 0.5 & 2.0 \\
\bottomrule
\end{tabular}}
\end{table}

After a successful  pilot  physics  Fill  in  2012~\cite{pPb_2012},  \texttt{LHC}  provided its four major experiments with approximately 31\,\invnb of pPb  luminosity  
at an unprecedented nucleon-nucleon center-of-mass energy of  5.02\,\TeV\ in  early  2013~\cite{pPb_2013},  with  several  variations  of  the  operating  conditions.
The gain of about a factor of 25 in collision energy relative to previous asymmetric systems has been one of the largest leaps in the history of particle accelerators. 
Together with a ``reference'' pp run at 2.76\,\TeV, they rendered the very last physics operation  before the long shutdown 1 (Table~\ref{tab:design_vs_achieved}).

For five of the \texttt{LHC} experiments, the second proton-nucleus collision run in 2016 offered the tremendous opportunity to answer a range of crucial 
physics questions~\cite{pPb_opportunities} arising from unexpected discoveries, e.g., collective phenomena in small systems reminiscent to the creation of the \QGP\ state of matter~\cite{cms_ridge_pp7,cms_ridge_pPb5,cms_ridge_pp13}. 
Unlike earlier runs, the requirements for the 2016 proton-nucleus run, and hence operating conditions,  were dissimilar~\cite{pPb_2016_plans} among  
the  different  experiments, in terms of collision energy, 
luminosity, and pileup. Luminosity  sharing  deemed a  critical  issue  to  be  solved since these requests appeared mutually incompatible within the available one month of operation. 
Nevertheless, a plan to satisfy most requirements was implemented successfully exploiting the different beam lifetimes at two nucleon-nucleon center-of-mass energies of 5.02 and 8.16\,\TeV, 
a variety of luminosity sharing and filling schemes, the later further complicated by the separate proton and heavy ion injection, and reversal of beam directions. 

Despite the complex strategy for repeated recommissioning and operation of \texttt{LHC} the longest ever Fill (numbered 5510) was achieved with luminosity leveled for almost 38\,$\mathrm{h}$. 
The peak luminosity also surpassed the ``design'' value by a factor 7.8 (Table~\ref{tab:design_vs_achieved}), and the amount of integrated luminosity substantially exceeded 
the requests of the majority of the experiments. 
With the improvements in the performance of the injector complex and transmission efficiency into \texttt{LHC}~\cite{pPb_inj}, average Pb bunch intensities of  ions 
and normalized transverse emittances of about  $2.1\times 10^{8}$ and 1.6\,\mum, respectively, were achieved at 6.5\,$Z$\,\TeV.

Table~\ref{table:pbLHC_2016} summarizes all primary goals of the 2016 pPb run along with further useful data sets delivered parasitically, 
and Fig.~\ref{fig:pPbLHC} demonstrates the integrated luminosity performance achieved at all heavy ion periods at \texttt{LHC} so far.

\begin{table}[!ht]
\makebox[\textwidth][c]{ 
  \footnotesize
  \begin{tabular}{ l || p{3cm} | p{4cm} | p{4cm}  }
   \toprule
   \rootsNN   & Experiment & Planned luminosity & Delivered luminosity   \\
   \midrule
    \multirow{3}{*}{5.02\,\TeV\ pPb}   & \texttt{ALICE} & $7\times10^8$ min.-bias events & $7.8\times10^8$ min.-bias events\\
                                    & \texttt{ATLAS},\texttt{CMS}   & - & $>0.4$\,\invnb \\
                                    & \texttt{LHCb}   & - & \texttt{SMOG}~\cite{Aaij:2014ida} p\ce{^{4}_{2}He+}, etc.\\
    \midrule
    \multirow{1}{*}{8.16\,\TeV\ Pbp}   & \texttt{ATLAS}, \texttt{CMS} & 100\,\invnb & $>180$\,\invnb\\
    \midrule
    \multirow{2}{*}{8.16\,\TeV\ pPb}   & \texttt{ALICE}, \texttt{LHCb} & 10\,\invnb & 14,13\,\invnb\\
                                    & \texttt{LHCf}   & ($9-12$\,$\mathrm{h}$)$\times 10^{28}$\,$\mathrm{cm}^{-2}\mathrm{s}^{-1}$ & 9.5\,$\mathrm{h}$\\
    \midrule
    \multirow{1}{*}{8.16\,\TeV\ Pbp}   & \texttt{ALICE}, \texttt{LHCb} & 10\,\invnb & 25,19\,\invnb\\
   \bottomrule
  \end{tabular}
}
\caption[Data sets, both primary and additional goals, delivered in the 2016 proton-nucleus runs]{
  Different activities including primary and additional goals were carried out during the various phases of the 2016 proton-nucleus run. 
  More specifically, it involved two nucleon-nucleon center-of-mass energies, i.e., 5.02 and 8.16\,\TeV, and different particle species circulating in the two rings. 
  Protons and \ce{^{208}_{82}Pb+} ions were injected in beam 1 and 2, respectively, for both energies, and beam species were switched once the required integrated luminosity was reached. 
  All these changes in machine configuration took place in about one month, achieving a return to 5.02\,\TeV\ operation to fulfill additional requests.
  The peak luminosity exceeded the ``design'' value by a factor 7.8~\cite{pPb_2016}, and the long-term integrated luminosity goal of 100\,\invnb~\cite{pPb_opportunities} has been substantially surpassed 
  in some \texttt{LHC} experiments.}
\label{table:pbLHC_2016}
\end{table}

\begin{figure}[!ht]
\begin{minipage}{.38\paperwidth}
\centering
\subfloat[]{\label{pPbLHC:a}\includegraphics[scale=0.2]{figures/apparatus/CCnew6_01_17_ext.png}}
\end{minipage}%
\begin{minipage}{.38\paperwidth}
\centering
\subfloat[]{\label{pPbLHC:b}\includegraphics[scale=0.2]{figures/apparatus/PbPb_LHC.png}}
\end{minipage}%
\caption[Integrated luminosity for proton-nucleus and nucleus-nucleus collisions at \texttt{LHC}]{\label{fig:pPbLHC} 
  Integrated luminosity delivered to the four major experiments at \texttt{LHC} using (a) proton-nucleus and (b) nucleus-nucleus collisions during Run 1 and 2.
  Following up on a feasibility test and pilot physics Fill in October 2011 and September 2012, respectively, the first one month-long run took place in January 2013, 
  meaning asymmetric proton-nucleus collisions remain a novel mode of operation at \texttt{LHC}. 
  As in most years, the \texttt{LHC} apparatus was reconfigured for a month-long heavy ion run; 2016 was devoted to colliding beams of protons and \ce{^{208}_{82}Pb+} ions.
  Despite such novelty, the accelerator team succeeded in delivering enormous data sets to all major \texttt{LHC} experiments for the investigation of nuclear matter~\cite{pPb_2016}.
}
\end{figure}

