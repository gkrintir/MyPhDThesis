
\chapter{Projected improvements in the precision of the inclusive \stt measurement}
\minitoc

\InitialCharacter{T}he reference proton-proton run in 2017 consisted of collisions at $\sqrt{s}=5.02$\,\TeV,
the same nucleon-nucleon center-of-mass energy to the heavy ion collisions in 2018, i.e., the last run before the second long shutdown.
The precision of the inclusive \ttbar\ cross section will profit from the larger amount of data (about 0.3\invfb) and the phase-I silicon pixel tracker. 
Owing to refined transverse beam shape models, the systematic uncertainty associated with the knowledge of the absolute luminosity scale is expected to be further decreased.
This has been already demonstrated by a specially conducted run at the end of 2017, in which the beams were adjusted for low pileup configuration. 
In this Chapter, a projection of the \stt measurement is presented based on the result previously attained in 2015, and the expected improvements are used to estimate 
the tighter constraints on parton distribution functions.

The material in the following Chapter, as documented in \hyperref[hellobiblio]{Scientific output} and \hyperref[hellobiblio1]{Internal notes}, relies almost exclusively on a original contribution.
The latter further includes the preliminary effort of luminosity calibration for the pp run at $\sqrt{s}=5.02$\,\TeV\ in 2017.

\clearpage

\section{The largest reference proton-proton data set at 5.02\,\TeV}

The second proton-proton reference data set in Run 2 offers the unprecedented opportunity to establish a precise baseline evaluation relative to heavy ion data, and to parasitically
improve and extend top quark cross section measurements at $\sqrt{s}=5.02$\,\TeV. 
The higher amount of data relative to 2015 (Figs.~\ref{cumulative_2017:a} and~\ref{cumulative_2017:b}) and the inclusion of the phase-I pixel silicon tracker allow even for fine-binned measurements 
in phase space regions---especially at the edge of the tracker acceptance---that are affected in the current \stt\ measurement, e.g., 
the $\lvert \eta \rvert>2.1$ region (Figs.~\ref{cumulative_2017:c} and~\ref{cumulative_2017:d}) because of the \QCD\ multijet background. 
The most significant reduction of uncertainty is thus expected because of the higher event count in data and MC simulation, and reduced uncertainty in the absolute luminosity scale determination. 
Improved jet energy calibration and b jet identification can also be foreseen. 
Since the measurable portion of the phase space will be increased, in turn,
the theory-based uncertainty in the extrapolation to the full phase space will be reduced. 
It can be then demonstrated that the projected inclusive \ttbar\ cross sections a consistently have a sizeable impact on the proton PDFs, with the strongest effect on the gluon distribution.

\begin{figure}[!ht]
\begin{center}
\subfloat[]{\label{cumulative_2017:a}\includegraphics[width=.47\textwidth]{figures/outlook/int_lumi_per_day_cumulative_pp_2017_5TeV_NormtagLumi.pdf}}
\subfloat[]{\label{cumulative_2017:b}\includegraphics[width=.47\textwidth]{figures/outlook/int_lumi_per_day_pp_2017_5TeV_NormtagLumi.pdf}}\\
\subfloat[]{\label{cumulative_2017:c}\includegraphics[width=.45\textwidth]{figures/outlook/ElecEta_ElMu_2jets.pdf}}
\subfloat[]{\label{cumulative_2017:d}\includegraphics[width=.45\textwidth]{figures/outlook/ElecPt_ElMu_2jets.pdf}}\\
\end{center}
     \caption[Cumulative day-by-day integrated luminosity, and predicted and observed kinematic distributions of the electron in the dilepton final state at $\sqrt{s}=5.02$\,\TeV in 2017]
      {
        (a,\,b) Cumulative luminosity as a function of time delivered by \texttt{LHC} and recorded by \texttt{CMS} during the 5.02\,\TeV pp reference run in 2017~\cite{pp5TeV2017_lumi}. 
        The plot makes use of the offline preliminary calibration that amounts to $316.31 \pm 11.07$\,\invpb~\citeAN{AN-18-186} under the condition of ``stable beams'' 
        and with no requirement on the data quality in \texttt{CMS}.
        (c,\,d) Predicted and observed distributions of the electron $\eta$ and \pt for events passing the dilepton criteria (Table~\ref{tab:cutFlow}), 
        and after requiring at least two jets, in the \empm final state (image courtesy of J. Gonz\'{a}lez, Oviedo).
      }
      \label{fig:cumulative_2017}
\end{figure}


\section{Incremental improvements in luminosity determination}
\label{sec:lum_impr}

\subsection{Refined impact of nonfactorizable beam shape}

The vdM scan method assumes that the bunch proton density function is factorizable into independent $x$- and $y$-dependent terms.
 However, this assumption is not strictly valid, and leads to a biased estimate of the beam overlap area. 
The two-dimensional vertex distributions accumulated during the beam imaging scans in \myDate are used to measure the bunch proton densities, 
extending the methodology described in Section~\ref{sec:xyCor}.

The simplest model for the bunch proton density that has a correlated spatial dependence is a Gaussian distribution (Eq.~\ref{eq:sgmodel1}) 
with $x$--$y$ correlations parametrized by a correlation parameter $r$. For the refined analysis, various parametrizations are considered to determine 
the optimal fit function to the vertex distributions. In the end, the fit is performed using a model that describes the proton density function
$\rho(x,y)$ with a main Gaussian component $g_{\textrm{M}}$ (following the form of Eq.~\ref{eq:sgmodel1}) with a large
weight $w_{\textrm{M}}$, a wide component $g_{\textrm{W}}$ with a small weight to model wide tails, and a narrow component $g_{\textrm{N}}$
with a small but negative coefficient $-w_{\textrm{N}}$ to model a flattened central part, i.e,
\begin{equation}\begin{aligned}
    \rho(x,y) &= - w_{\textrm{N}} g_{\textrm{N}}(x,y) + w_{\textrm{M}} g_{\textrm{M}}(x,y) + (1+w_{\textrm{N}}-w_{\textrm{M}}) g_{\textrm{W}}(x,y).
    \label{eq:supdg}
\end{aligned}\end{equation}
The reconstruction of bunch proton densities using a fit model with the weighted sum of three Gaussian distributions is also considered.

To derive a correction for the measured cross section from the vdM scan the fitted bunch proton
densities are used to simulate vdM scans. The product $\Sigma_x\Sigma_y$ from the MC simulation of
the vdM scan method is then compared to the value from direct integration of the nonfactorized bunch
proton densities, yielding an estimate of the inaccuracy introduced by using the beam overlap area $\propto \Sigma_x\Sigma_y$, 
which does not account for the $x$--$y$ correlations of the bunch proton densities.
Multiple pseudo-experiments are performed to derive the central value of the correction and its statistical uncertainty. 
%Additionally, a Monte Carlo simulation of beam imaging scans is performed to estimate any
%possible bias from the fit method, and the resulting uncertainty is added to the statistical one.


\begin{figure}[!ht]
\begin{center}
\subfloat[]{\includegraphics[width=.45\textwidth]{figures/outlook/Fill4634_corr_central.pdf}}
\subfloat[]{\includegraphics[width=.45\textwidth]{figures/outlook/Fill4634_chisq_central.pdf}}\\
\subfloat[]{\includegraphics[width=.45\textwidth]{figures/outlook/Fill4634_comb_central_SupDG_bcid644_X2.pdf}}
\subfloat[]{\includegraphics[width=.45\textwidth]{figures/outlook/Fill4634_comb_central_TG_bcid644_X2.pdf}}
\end{center}
\caption[Primary vertex transverse and longitudinal resolution as a function of the track multiplicity]{
  (a) Difference between the beam overlap area from the simulated vdM scans and the integrals incorporating genuine nonfactorizabilities as a function 
  of the bunch crossing identification number in \myDate scan program.
  (b) The goodness-of-fit $\chi^2$ divided by the number of degrees of freedom in the modeling of the vertex distributions accumulated during beam imaging scans in \myDate.
  (c,\,d)  Projections to radial and angular coordinates from the two-dimensional pull distributions of proton density models built 
   based on (c) Eq.~\ref{eq:supdg} and (d) a combination of three Gaussian pdf~\citeAN{AN-17-240}.
}
\label{fig:xycorr_updated}
\end{figure}

Figure~\ref{fig:xycorr_updated} shows the calculated corrections and the goodness-of-fit $\chi^2$ divided by the number of degrees of freedom as a function of the bunch crossing identification number
from different fits to the reconstructed vertex distributions. In general, the fit results are good, although there is some remaining mismodeling for BCID 644. 
A correction factor on the visible cross section of 1.5\% is obtained taking the bunch-averaged correction and using the weighted sum of three Gaussian distributions. 
The uncertainty due to residual beam-shape effects in the bunch proton densities is estimated to be 0.2\%, covering the range of correction factors obtained by the alternative 
model of Eq.~\ref{eq:supdg} per BCID.

\subsection{Concept and formalism of variable separation scan}

To displace the beams at the IP the orbit is modified with a four-magnet closed orbit bump (Fig.~\ref{fig:trims}) that allows establishing an orbit deformation with well-defined position and slope at any given point $m$, with $m$ being located between the second and the third magnet. 
Considering the kick $k_{i},\,(i=1-4)$ instantaneous values in the field generated by a corrector $i$, the requirement for the bump to be ``closed,'' i.e., 
an unchanged orbit at the first and last corrector can be expressed as a system of linear equations
\begin{equation}
\begin{pmatrix}
x_{m}  \\
\beta_{m}x'_{m}+\alpha_{m}x_{m}\\ 
x_{1}\\
x'_{1}
\end{pmatrix} = 
\begin{pmatrix}
A\\
\beta_{m}A'+\alpha_{m}A\\ 
0\\
0
\end{pmatrix} = 
T\times C
\begin{pmatrix}
\sqrt{\beta_{1}}k_{1}\\
\sqrt{\beta_{2}}k_{2}\\
\sqrt{\beta_{3}}k_{3}\\
\sqrt{\beta_{4}}k_{4}
\end{pmatrix}
\, ,
  \label{eq:myeqn}
\end{equation}
where the transformation matrix $T$ is given in and the constant $C$ depends on the $\beta_{m}$ function and the overall machine tune (ref); 
$A$ and $A'$ are the bump amplitude and slope, respectively. For the specific case of the variable separation scans each beam is displaced at the IP 
with both $\alpha$---the slope of the $\beta_{m}$ function---and $x'_{m}$ equal to zero allowing only for a parallel separation.

A bump nonclosure would result in a scale factor error in the beam displacement which would directly
modify the measured beam size. The origin of the nonclosure could be the combined effect of the ``hysteresis,'' i.e., distance (from the IP) dependent distortions 
in $x_{m}$ and $x'_{m}$, and lattice imperfections. To minimize the hysteresis effect during the scan the translation is always performed in the same
direction. In addition, for each scan an acquisition at zero separation is performed at the beginning, middle and the end of the scan;
given the middle point lies on a different hysteresis branch as compared to the two other points, 
an indication of the hysteresis effect translates to a reduction of rate assuming the beam parameters do not significantly vary, e.g., beam emittance, intensity, etc.


\begin{figure}[!ht]
\begin{center}
\subfloat[]{\includegraphics[width=0.5\textwidth]{{figures/outlook/Figure_004-a.pdf}}}
\subfloat[]{\includegraphics[width=0.5\textwidth]{{figures/outlook/Figure_004-b.pdf}}}
\end{center}
  \caption[Length scale calibration scan using the variable separation procedure]{
    Length scale calibration scan, using the variable separation procedure, for the $x$ (a) and $y$ (b) direction of beam 1 (red) and 2 (blue), respectively. 
    Shown is the measured displacement of the luminous centroid as a function of the expected displacement based on the corrector bump amplitude. 
    The line is a linear fit to the data. Errors are of statistical nature~\citeTH{CMS-PAS-LUM-17-004}.}
  \label{fig:LSCATLAS}
\end{figure}


The variable separation scan for the length scale calibration was introduced in the July 2017 scan program using pp collisions at $\sqrt{s}=13$\,\TeV. 
Similar to the constant separation scan we make use of the \texttt{CMS} tracker to reconstruct the displacement of the luminous region. 
However, the variable separation scan is designed to measure the calibration constant for each of the two beams in each of the two directions
independently.  The calibration data for both horizontal and vertical bumps of beam 1 and  2 are shown
in Fig.~\ref{fig:LSCATLAS}. The scale factor which relates the nominal beam displacement to the measured
displacement of the luminous centroid is given by the slope of the fitted straight line; the intercept is
irrelevant. Because regular vdM scans are performed by displacing the two beams symmetrically in opposite
directions, the relevant scale factor in the determination of the beam overlap is the average of the scale
factors for beam 1 and 2 in each plane, i.e., $(0.9937+0.9947)/2 \approx 0.994$ and $(0.9978+0.9965)/2 \approx 0.997$ in $x$ and $y$, respectively. 
This represented an excellent agreement with the scale factors obtained based on the constant separation scan, though with much better precision~\citeTH{CMS-PAS-LUM-17-004}.


\section{Cross section measurements}

\subsection{Statistical and systematic uncertainty treatment}

Based on the measurements of the inclusive \ttbar\ cross sections at $\sqrt{s}=5.02$\,\TeV~\citeTH{topobs_CMS_jhep}, their performance can be conjectured 
using the data set collected in 2017 at the same center-of-mass energy. 
Although the higher instantaneous luminosity resulted in about three pp interactions per bunch crossing on average, pileup mitigation techniques are not crucial
for a good performance of the \ttbar\ reconstruction. 
Because of the inclusion of the phase-I silicon tracker, the requirement of $\lvert \eta \rvert < 2.1$ for electrons can be relaxed. 
At least two of the jets have to be identified as b jets, i.e., fulfilling a requirement of the CSVv2 algorithm, whereas alternative b tagging techniques can be used 
with a higher b-jet selection efficiency and rejection power for other jets in \ttbar\ events (Table~\ref{tab:OP}).

Theoretical and modeling uncertainties make a significant contribution to the overall uncertainty. 
Since it is speculative to estimate a possible reduction of these uncertainties we consider the current ones as a conservative estimate. 
Improvements of the theoretical predictions can be expected, while further measurements at $\sqrt{s}=13$\,\TeV\ could reduce the modeling uncertainty. 

The individual sources of uncertainty, as well as the assumed correlations, that are used as input to illustrate the impact of the conjectured \stt\ measurements 
at $\sqrt{s}=5.02$\,\TeV\ on the knowledge of the proton PDFs are summarized in Table~\ref{tab:inputs_proj}. 
The setup of the \QCD\ analysis at NNLO is identical to the one used in Ref.~\citeTH{topobs_CMS_jhep}, i.e., 
the total uncertainty in the measurements are propagated to the extracted \QCD\ fit parameters using the MC method.

\begin{table*}[htb]
\scriptsize
\begin{center}
\caption{Conjectured inputs to the updated \QCD\ analysis at NNLO using the data set collected in 2017 at $\qrt{s}=5.02$\,\TeV. 
Entries marked with one (two) ``*'' delineate foreseen but difficult to quantify improvements over experimental (theoretical) point of view.
\label{tab:inputs_proj}}
\resizebox{\textwidth}{!}{\begin{tabular}{lccccc}
\toprule
Final state & \empm & \mmpm & $\ell+$jets & & \\
\midrule
\midrule
Central value (pb) & 68.9 & 68.9 & 68.9 & &\\
\midrule
\midrule
Uncertainty (\%) &  &  &  & Correlation & Note\\
\midrule
Data sample event count          & $25/\sqrt{10}\sim 8$  & $48/\sqrt{10} \sim 16 $ & $9.5/\sqrt{10} \sim 3.0$   & 0 & $\times 10$ expected yield \\
MC sample event count         & $1.4/\sqrt{2} \sim 1.0 $   & $2.4/\sqrt{2} \sim 1.7$  & $0.1/\sqrt{2} < 0.1 $   & 0 & $\times 2$ MC simulated yield\\
b tagging  efficiency           & -     & -    & 3.4 (*)   & - & New algorithms; performance to be validated\\
Electron efficiency   & 1.4 (*) & -    & 1.1 (*)   & 1 & Some improvement upon endcap electrons\\
Muon efficiency         & $3.0 \rightarrow 2.5 $   & $6.1 \rightarrow 5.0 $  & 1.7   & 1 & Less conservative approach\\
\ptmiss             & -     & 0.7  & -     & - &\\
Jet energy scale      & 1.3   & 1.3  & 3.0   & 1 &\\
Jet energy resolution & $<$0.1  & $<$0.1 & 0.6   & 1 &\\
\QCD\ multijet background        & -     & -    & 2.4 (*)   & - & Some improvement upon endcap electrons\\
W+jets background     & $2.5/\sqrt{10} \sim 0.8 $   & $0.7/\sqrt{5} \sim 0.3$  & $3.5/\sqrt{5} \sim 1.6 $   & 1 & $\times 10\,(5)$ due to data-\,(MC-) based estimation\\
tW background         & 1.4   & 1.6  & 1.3   & 1 &\\
WV background         & 0.7   & 0.9  & -     & 1 &\\
Z/$\gamma^{*}$ background  & $2.7\rightarrow \sim 2.0$   & $15.4/\sqrt{10} \sim 5.0$ & -     & 1 & $\times 10$ due to data-based estimation; in \empm partly  \\
$\mu_\textrm{R},\mu_\textrm{F}$ scales of \ttbar\ signal (PS)  & 1.2   & 1.7  & 4.4   & 1 &\\
$\mu_\textrm{R},\mu_\textrm{F}$ scales of \ttbar\ signal (ME) & $<$0.1  & 1.1  & $<$0.1  & 1 &\\
Hadronization model of \ttbar\ signal  & $1.2$ (**)   & $5.2$ (**)   & $2.8$ (**)   & 1 & \\
PDF                   & 0.5   & 0.4  & $<$0.1  & 1 &\\
Integrated luminosity            & $2.3\rightarrow 1.7$   & $2.3\rightarrow 1.7$  & $2.3\rightarrow 1.7$   & 1 & see Section~\ref{sec:lum_impr}\\
\bottomrule
\end{tabular}}
\end{center}
\end{table*}


\subsection{PDF constraints including the inclusive \stt}


\begin{figure}[!ht]
\begin{center}
\subfloat[]{ \label{projections:a}\includegraphics[width=0.415\textwidth]{{figures/outlook/gluon}}}
\subfloat[]{ \label{projections:b}\includegraphics[width=0.39\textwidth]{{figures/outlook/gluon_proj}}}\\
\subfloat[]{ \label{projections:c}\includegraphics[width=0.4\textwidth]{{figures/outlook/u_valence}}}
\subfloat[]{ \label{projections:d}\includegraphics[width=0.4\textwidth]{{figures/outlook/d_valence}}}\\
\end{center}
  \caption[Relative uncertainties in the gluon and valence quark PDFs of the proton as a function of $x$ at $\mu^2_{\textrm{F}}=10^5\,\mathrm{GeV}^2$ including \stt data at $\sqrt{s}=5.02$\,\TeV]{
    (a) The relative uncertainties in the gluon distribution functions of the proton as a function of $x$ at $\mu^2_{\textrm{F}}=10^5\,\mathrm{GeV}^2$ from a 
    \QCD\ analysis using the \texttt{HERA} DIS and \texttt{CMS} muon charge asymmetry measurements, and also including the \texttt{CMS} \stt results at $\sqrts = 5.02$\,\TeV.
    (b) Same as (a) but corresponding to the conjectured \stt results.
    (c,\,d) Same as (b) but corresponding to the relative uncertainties in the valence quark distribution functions. 
    The latter needs to be compared with Figs.~\ref{PDFs_light_and_gluon:d} and~\ref{PDFs_light_and_gluon:d}, respectively.
}
  \label{fig:projections}
\end{figure}

For each replica, the PDF fit is performed, and the uncertainty is estimated as the RMS around the central value. 
The relative uncertainties in the gluon distributions, as obtained in the 
\QCD\ analyses with and without the measured and conjectured values for \stt at $\mu^2_{\textrm{F}}=10^5\,\mathrm{GeV}^2$, are shown in Figs.~\ref{projections:a} and~\ref{projections:a}, respectively. 
A further reduction of the uncertainty in the gluon distribution at $x \gtrsim 0.1$ is observed, once the conjectured values for \stt are included in the fit. 
Although they remained unaffected in Ref.~\citeTH{topobs_CMS_jhep}, a small improvement to the uncertainties in the valence quark distributions 
(Figs.~\ref{projections:c} and~\ref{projections:d}) is also observed, consistent with the outcome of Refs.~\cite{TOP-18-004} and~\cite{FTR-18-015}.


It is thus demonstrated that the projected inclusive \ttbar\ cross sections have a stronger impact on the gluon distribution in the proton. 
Overall, this measurement is expected to profit from the higher amount of data collected in 2017, the improved phase-I silicon tracker detector, 
and the better knowledge of the absolute luminosity scale.

