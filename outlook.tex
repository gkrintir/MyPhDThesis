\begin{outlook}
  
  In November 2015, \texttt{LHC} delivered proton-proton (pp) collisions at a center-of-mass energy ($\sqrt{s}$) of 5.02\,\TeV,
  corresponding to an integrated luminosity of $27.4\pm0.6$\,\invpb in the \texttt{CMS} experiment.
  Measurements of the top quark pair production cross section (\stt) at various $\sqrt{s}$ probe different values of the fraction $x$ of the proton longitudinal momentum,
  and thus provide complementary information on the gluon content of the proton.
  Using \ttbar\ candidate events with $\ell$+jets, where leptons are either electrons ($\ell=\mathrm{e}$) or muons ($\ell=\mu$), and dilepton (\empm\ or \mmpm) final states,
  the result is $\sigma_{\mathrm{t\overline{t}}} =  69.5 \pm 6.1\,(\textrm{stat}) \pm 5.6\,(\textrm{syst}) \pm 1.6\,(\textrm{lumi})\,\pb$ with a total relative uncertainty of 12\%,
  that represents a remarkable achievement and a significant improvement relative to the first observation based on the \empm\ final state alone.

  The correlation in phase space of the jets from the W boson hadronic decay (``light jets'') carries a distinctive hallmark with respect to the main backgrounds that are
  controlled by counting the number of jets coming from the hadronization of the b quark (``b jets'') in the selected $\ell$+jets events.
  The signal extraction is then performed maximizing a profile likelihood fit to the distribution of a kinematic variable, sensitive to the resonant
  behavior of the light jets, for different categories of lepton flavor and jet multiplicity.
  Similarly to the most recent \texttt{LHC} studies of the inclusive \stt\, the measurement is first performed in a fiducial phase space---a restricted region
  that closely resembles the detector acceptance in \pt\ and $\eta$ of leptons and jets---and it is then extrapolated to the full phase space based on MC simulation.
  The individual and combined results are compared to the predictions from the ABMP16, CT14, MMHT14, and NNPDF3.0 parton distribution functions (PDFs).
  Theoretical predictions from different PDFs have comparable values and uncertainties, once consistent values of $\alpha_{\rm{s}}$ and $m_{\rm{top}}$ are associated with the respective PDF set.
  The limited-precision measurement can be complemented with the significantly larger pp data set recorded in 2017, equivalent to almost $0.3$\,\invfb.

  Asymmetric collisions of lead (\ce{^{208}_{82}Pb^{82+}}) nuclei with protons had not been included in the initial \texttt{LHC} design.
  However, unexpected discoveries in small collision systems, reminiscent of flow-like collective phenomena, engaged further investigations.
  After the short, yet remarkable, pilot physics run in 2012, and the first full one-month run in early 2013, 
  the second full proton-nucleus run took place in late 2016 delivering collisions primarily at a nucleon-nucleon center-of-mass energy (\rootsNN) of 8.16\,\TeV, for each direction of the beams. 
  Apart from complex bunch filling schemes due to the generation of the beams from two separate injection paths, 
  the distinct feature of operation with asymmetric collisions at \texttt{LHC} is the difference in revolution frequencies.
  Given that the colliding bunches have significantly different size and charge, both beams are displaced transversely, 
  onto opposite-sign off-momentum orbits, and longitudinally, to restore collisions at the proper interaction points.
  The achieved performance surpassed though almost eight times the designed instantaneous luminosity.
  The long-term integrated luminosity goal of 100\,\invnb has been surpassed, rendering the 2016 pPb run the baseline for several years.
  
  Until recently, top quark measurements therefore remained out of reach in nuclear collisions due to the reduced amount of integrated luminosity produced during the first period at the \texttt{LHC},
  and the relatively low \rootsNN values available at the \texttt{BNL} \texttt{RHIC}.
  Novel studies of top quark cross sections have finally become feasible with the 2016 pPb run at $\rootsNN=8.16$\,\TeV.
  The top pair production cross section has been measured for the very first time in nuclear collisions, using a data set of $174\pm6$\,\invnb in the \texttt{CMS} experiment.
  The measurement is performed analyzing events with exactly one isolated lepton and at least four jets,
  and minimally relies on assumptions derived by simulating signal and background processes.
  The resonant nature of the invariant mass of the two light jets, \mjj, provides a distinctive feature of the signal with respect to the main backgrounds, i.e.,
  from \QCD\ multijet and W+jets processes.
  The significance of the \ttbar\ signal against the background-only hypothesis is above five standard deviations.
  The measured cross section is $45\pm 8\,(\textrm{total})$\,$\mathrm{nb}$, consistent with perturbative quantum chromodynamics calculations (using the CT14 proton PDF and the EPPS16 nuclear PDF for the lead ions) as well as the expectations from scaled pp data.
  To further support the hypothesis that the selected data are consistent with the production of top quarks,
  a ``proxy'' of the top quark mass is constructed as the invariant mass of candidates formed by pairing the W candidate with a b jet.
  This first study clearly paves the way for further detailed investigations of the top quark production in nuclear interactions~\cite{FTR-18-027},
  providing in particular a new tool for studies of the hot and dense matter created in nucleus-nucleus collisions~\cite{HL-HE-LHC_Report,HL-LHC_Report,HE-LHC_Report,WG5}.

  Measurements of production cross sections provide fundamental tests of theoretical predictions.
  Increasingly higher precision both in the experimental measurements and the theoretical predictions is required to determine fundamental parameters of the standard model.
  At \texttt{LHC}, cross section measurements are often limited by the uncertainty in the integrated luminosity that is currently known with a precision of $\mathcal{O}$(2--4\%),
  depending on the collision system.
  The luminosity calibration is based on the van der Meer scan technique, a purely experimental method. In dedicated sessions,
  the beam axes are moved in the transverse plane across each other such that the ``beam overlap integral'' can be determined.
  From the extracted integral, and the measured beam currents, the scale of the instantaneous luminosity is determined.
  To this end, several observables are used, each one corresponding to a cross section (\sigmaVis)  in the 
  visible phase space region. The integrated luminosity for an arbitrary period of data
  taking is obtained from the accumulated counts of calibrated \sigmaVis. Relative nonlinearity and
  long-term stability in the response of the detector-based algorithms account for residual dependencies of the \sigmaVis ratios on conditions typical for normal physics operations.

  Significant improvements in the luminosity measurement are being planned, and a target uncertainty of 1\% has been set for the High-Luminosity \texttt{LHC}~\cite{HL-HE-LHC_Report,HL-LHC_Report,WG1}.
  Such improvement is expected to be achieved by combination of improved luminosity detector instrumentation---currently in the design phase---and
  refined analysis techniques, rapidly developing during the analysis of Run 2 data.  
  
  

  %Second, the observation of quantitatively-large quenching phenomena in essentially all measured
  %hard hadronic observables in AA collisions has established the feasibility of testing the produced QCD
  %matter with a broad set of probes whose production rates are controlled with good precision with pp ref erence measurements and perturbative QCD calculations.
 
  % \begin{figure}
  %   \begin{minipage}{.5\linewidth}
  %     \centering
  %     \subfloat[]{\label{pPb:a}\includegraphics[scale=0.25]{pPb_Jowette.png}}
  %   \end{minipage}%
  %   \begin{minipage}{.5\linewidth}
  %     \centering
  %     \subfloat[]{\label{pPb:b}\includegraphics[scale=0.215]{gluonPDF_epps16.png}}
  %   \end{minipage}\par\medskip
  
  %   \caption{
  %   (a) Accumulation of integrated luminosity in each LHC experiment in the pPb runs for years 2013 and 2016~\cite{Jowett:IPAC2017}.
  %   (b) The EPPS16 nuclear modifications for the bound gluon PDF in Pb nucleus at the parametrization scale $Q^2=10^4$~$\rm{GeV^2}$.
  %   The thick black curves correspond to the central fit values, the dotted curves to the individual error sets, and 
  %   the total uncertainties are shown as blue bands~\cite{Eskola}.}
  %   \label{fig:pPb}
  % \end{figure}
  
 \end{outlook}