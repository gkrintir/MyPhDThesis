\section{Cross-detector stability}
\label{sec:stability}

Given that the absolute calibration of the applied algorithms is carried out under specially tailored beam conditions, 
the comparison of the relative response of several independent luminometers during routine physics conditions reveals possible time- and rate-dependent effects.  
The stability of the PCC response has been investigated based on the relative contributions of the different layers of the pixel detector to the total PCC response over the \myDate data-taking period. 
Excellent stability and independence to various data-taking conditions of the different layers relative to
the total response within 0.5\% are found as shown in Fig.~\ref{PCC_RelContribution:a}. 
The long-term PCC response has been independently monitored and compared to the rate of the \texttt{DT} luminometer over the entire 2015 data-taking
period at \highEnergy. The RMS of 1\% for the per luminosity section ratio is conservatively assigned as systematic uncertainty 
due to the stability of the PCC rate in the \myDate period~\cite{CMS-PAS-LUM-15-001}. 
If the luminous region moves significantly, then the acceptance of the pixel detector slight changes, and the effect appears as a change in the visible cross section. 
In 2015, the luminous region has been always within $\approx \pm 4$\,$\mathrm{cm}$ from the origin in the longitudinal direction (Fig.~\ref{fig:bs_coordinates_2015}). 
Within this range, changes in acceptance are negligible~\cite{CMS-PAS-LUM-13-001,CMS-PAS-LUM-15-001}.


For the \mypPbDate period, the RMS of 0.9 (0.7)\% of the \texttt{PLT} over \texttt{HF} ratios as a function of the single-bunch instantaneous luminosity is considered 
as the systematic uncertainty related to the stability of the luminometer rate in the Pbp (pPb) period, as displaced in Fig.~\ref{fig:PLTLumiOverHFSBILOffline}. 
The stability has been independently investigated using the silicon pixel detector; the relative contributions of the different layers of the pixel detector to the total response is
shown in Fig.~\ref{PCC_RelContribution:b}. Excellent stability of the layers and independence to various data-taking conditions is indeed found.

\begin{figure*}[tbh]
\begin{minipage}{.5\linewidth}
\centering
\subfloat[]{\label{PCC_RelContribution:a}
  \includegraphics[width=0.35\paperwidth]{figures/luminosity/Rel_Contribution_Whole_5TeV_JSON.png}}
\end{minipage}%
\begin{minipage}{.5\linewidth}
\centering
\subfloat[]{\label{PCC_RelContribution:b}\includegraphics[width=0.32\paperwidth]{figures/luminosity/PCC_Stability_Pbp.png}}
\end{minipage}\par\medskip
 \caption[The relative contributions of the pixel layers during pp and pPb collisions at $\sqrt{s}=5.02$ and $\rootsNN=8.16$\,\TeV]{\label{fig:PCC_RelContribution}
    The relative contributions in percentage of the different pixel layers for the entire (a) \myDate~\citeTH{CMS-PAS-LUM-16-001}  and (b) the first part (Pbp) of \mypPbDate~\citeTH{CMS-PAS-LUM-17-002} data-taking periods.
    The innermost pixel layer is excluded from the PCC rate measurements. Only periods where the \texttt{CMS} detector is fully operational are considered.
    The last part of the Pbp period, corresponding to \texttt{LHC} Fill 5538 with active leveling at IP5 and contributing negligibly to the total luminosity, has also been excluded.
}
\end{figure*}

\begin{figure}
\centering

 \subfloat[][]{\includegraphics[width=0.48\textwidth]{figures/luminosity/PLT_Over_HF_SBIL_Calibrated_Pbp.pdf}}
 \subfloat[][]{\includegraphics[width=0.48\textwidth]{figures/luminosity/PLT_Over_HF_SBIL_Calibrated_pPb.pdf}}

\caption[Ratio of \texttt{PLT} over \texttt{HF} luminosity estimations for Fills 5527 and 5563]{
  Comparison of the \texttt{PLT} over \texttt{HF} luminosity ratio against the single-bunch instantaneous (SBIL) luminosity, 
  the latter estimated based on \texttt{PLT} for the Pbp (a) and pPb (b) periods~\citeTH{CMS-PAS-LUM-17-002}.  
  The last part of the Pbp period, corresponding to \texttt{LHC} Fill 5538 with active levelling at IP5, has been excluded, while the rising trend for SBIL 
  higher than about 0.035\,$\mathrm{Hz}$/$\mu\mathrm{B}$ includes less than 2\% of the integrated luminosity.  
}
\label{fig:PLTLumiOverHFSBILOffline}
\end{figure}
\section{Conclusions}
\label{sec:conclusions}

Table~\ref{tab:systematics} summarizes the derived corrections along with their associated systematic uncertainty. The various sources are grouped into i) ``normalization'' 
uncertainty; that is, the luminosity scale calibration from the vdM scan procedure, and ii) ``integration'' uncertainty; that is, the uncertainty associated with the extrapolation of \sigmaVis 
to luminometer rate measurements under usual data-taking conditions.
The uncertainties in the integration and normalization are treated as uncorrelated, and  are summed  in  quadrature.
An uncertainty of 0.5\% associated to the deadtime estimate from the \texttt{CMS} \DAQ\ system is additionally included, affecting the recorded integrated luminosity. 
The dominant uncertainty contributing to the luminosity scale calibration is associated to the nonfactorizability of the colliding beam bunch densities.

The weighted averages of \sigmaVis for all BCIDs measured over the considered scans in the Pbp relative to the pPb period, 
corrected for effects impacting the accuracy of the absolute and relative bunch populations, and for beam--beam interactions, is found to be in excellent agreement 
for the symmetric luminometers.  A combined result is then determined by computing a weighted average from the \sigmaVis calibrations in the two periods.  
To avoid underestimating the uncertainty in the combined result it is assumed that most sources of systematic uncertainty are fully correlated except for 
the uncertainty in bunch-to-bunch and scan-to-scan variations, and cross-detector stability.  

In summary, the total uncertainty in the \texttt{CMS} luminosity measurement in \myDate is 2.3\% using proton-proton collisions at \myEnergy, and is estimated to be 3.5\% for 
data recorded with proton-nucleus collisions at \mypPbEnergy in \mypPbDate. The results can be compared to those obtained from the \texttt{ATLAS}~\cite{atlas_pp_lumi} (proton-proton), and \texttt{ALICE}~\cite{alice_pPb_lumi} and \texttt{LHCb}~\cite{lhcb_pPb_lumi}
(proton-nucleus) Collaborations.


\begin{table*}[htb]
\caption[Summary of the systematic uncertainty in the luminosity measurements at $\sqrt{s}=5.02$ and $\rootsNN=8.16$\,\TeV]{
  \label{tab:systematics}
  Summary of the systematic uncertainty in the \texttt{CMS} luminosity measurement using proton-proton and proton-nucleus collisions at $\sqrt{s}=5.02$ and $\rootsNN=8.16$\,\TeV, respectively.
  When applicable, the percentage correction is shown.}
\begin{center}
\begin{tabular}{|c|c|ccc|ccc|}
\hline
& \multirow{2}*{Source}  & \multicolumn{3}{c|}{ \multirow{2}*{Correction ($\%$)}} & \multicolumn{3}{c|}{ \multirow{2}*{uncertainty ($\%$)}}\\[+5pt] 
 & &pp&Pbp&pPb&pp&Pbp&pPb\\
\hline
\hline
\multirow{8}{*}{Normalization} & Beam nonfactorizability  &  - & - & - & 1.4 & 2.3 & 2.9 \\ \cline{2-8}
&Bunch-to-bunch variation             &  -& -&-  & - & 1.4 & 1.5 \\ \cline{2-8}
&Scan-to-scan variation               &  -& -&-  & - & 0.6 & 1.0 \\ \cline{2-8}
&Length scale                         & 1.0 & -&-  & 0.2 & 0.7 & 0.7 \\ \cline{2-8}
&Ghosts and satellites                &  1.8 & 1.3 & 1.2  &  0.2 & 0.7 & 0.6 \\ \cline{2-8}
&Orbit Drift                          &  - & - & -  &  0.4 & 0.5 & 0.5 \\ \cline{2-8}
&$\Sigma_{x,y}$ compatibility          &  - & - & - & - & 0.5 & 0.2 \\ \cline{2-8}
&Dynamic-$\beta$ effect               &  - & - & - & 0.5 & 0.5 &0.5  \\
&Beam--beam deflection                &  1 & 0.3 & 0.3 & 0.2 & 0.3 &0.3 \\ \cline{2-8}
&Beam current calibration             &  - & - & - & 0.3 & 0.3 &0.3 \\ \cline{2-8}
\hline
\hline
\multirow{2}{*}{Integration} & Cross-detector stability &  - & - & - & 1 & 0.9 & 0.7 \\ \cline{2-8}
							& Type 1/2 &  7 & - & - & 1 & - & - \\ \cline{2-8}
                         & \DAQ deadtime &  - & - & - & 0.5 & 0.5 &0.5  \\ \cline{2-8}
                          & Dynamic inefficiency &  - & - & - & 0.4 & - &-  \\ \cline{2-8}

\hline
\hline
\hline & Total & & & & 2.3 & 3.2 & 3.7\\
\hline
\end{tabular}
\end{center}
\end{table*}
