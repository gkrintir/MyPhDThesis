\section{The concept of luminosity}
\label{sec:intro}

Luminosity has been used in astronomy (cosmology) indicating the amount of electromagnetic energy an astronomical (cosmological) object radiates per unit time.
The term has been introduced in particle physics in the early 1960's in the context of the first matter-antimatter collider, Anello di Accumulazione,  at the Frascati laboratory
accelerating electron ($\mathrm{e^-}$) against positron ($\mathrm{e^+}$) beams at $\sqrt{s}=250$~\MeV~\cite{ada}.
The analogy between the definitions in accelerator physics and astronomy, that is driven by a characteristic ``source factor'' in both cases,  rendered luminosity as the proportionality coefficient 
between the event accumulation rate in a particle collider and the cross section.
Luminosity thus quantifies the potential of the collider for delivering a statistically significant sample of any class of events.

Storage-ring beam dynamics suggest operating particle colliders in ``bunched mode,'' i.e., each of the two beams consists of a
string of bunches typically unevenly distributed around the collider ring and numbering a few ten to a few thousand (see Section~\ref{sec:cycle}).
To determine the cross section of any given subatomic process at high energy colliding-beam experiments 
a measurement of the colliding-bunch luminosity must hence be performed. 
The single-bunch instantaneous luminosity $\mathcal{L_{\textrm{b}}}$, i.e., produced by a single pair of colliding bunches, can be written as
\begin{equation}
\label{eq:lumibunch}
\mathcal{L_{\textrm{b}}} = \frac{R}{\sigma_{\textrm{ref}}}\ ,%(t,\mu,n_b,...)}.
\end{equation}
where the interaction rate $R=\mu f_{\textrm{r}}$ for any reference process is linearly dependent on the average number of interactions per bunch crossing ($\mu$)
and the bunch revolution frequency ($f_{\textrm{r}}$). In principle, the reference process can be arbitrarily selected, and referring to inelastic collisions as a typical example, 
the total instantaneous luminosity for inelastic interactions known with an absolute scale $\sigma_{\textrm{inel}}$ is given by
\begin{equation}
\label{eq:totlumibunch}
\mathcal{L} = \sum_{\textrm{b}=1}^{n_{\textrm{b}}}\mathcal{L_{\textrm{b}}}=n_{\textrm{b}}\frac{\langle \mu \rangle f_{\textrm{r}}}{\sigma_{\textrm{inel}}}\ .%(t,\mu,n_b,...)}.
\end{equation}
In Eq.\,(\ref{eq:totlumibunch}) the sum runs over the bunch pairs $n_{\textrm{b}}$ colliding at the interaction point (IP), and the mean bunch luminosity is regulated by 
the bunch-averaged pile-up parameter $\langle \mu \rangle$. Therefore the instantaneous luminosity can be determined using any per-bunch-granularity method that measures the 
ratio $\mu / \sigma_{\textrm{ref}}$, or respectively  $\langle \mu \rangle/\sigma_{\textrm{ref}}$ for the total instantaneous luminosity.
%motivate a precision of order 1–2\% for the luminosity calibration
Although luminosity is a macroscopic indicator of the global performance of a collider, 
the observed bunch-to-bunch intensity and emittance variations in hadron colliders result in a large spread in $\mathcal{L_{\textrm{b}}}$, hence make
impractical any bunch-averaged luminosity measurements.

\subsection{Interaction rate determination}
\label{sec:rate}

Methods for absolute luminosity determination can be classified as being either direct or indirect.  
For instance, indirect methods make use of the optical theorem in a simultaneous
measurement  of  the  elastic  and  total  cross sections~\cite{totem} or perform a comparison  with 
\QED\ processes for which the absolute cross section is well known from theory~\cite{opal,ADAMCZYK201480}.  
Direct methods derive the luminosity from the measurement of the colliding-beam parameters. 
The analysis described in this thesis relies on two direct methods to determine the absolute luminosity calibration, i.e., 
the ``van der Meer'' (vdM) and the ``beam-imaging'' (BI) scan methods. 
With the notable exception of the ``first'' and ``second'' generation experiments at \texttt{CERN} Intersecting Storage Rings (\texttt{ISR}) facility---including the Louvain-Northwester group~\cite{ISR_UCLouvain}---the achieved precision in the luminosity determination at hadron colliders typically ranges from 1 to 15\%~\cite{Witold_overview}. 
The 1\% ``precision frontier'' is not uniquely linked to a fundamental limitation, rather it stems from a complex mix of sources of systematic uncertainty. 

The vdM technique exploits the ability to control the beam separation in both transverse coordinates with high precision (see Section~\ref{sec:lhc_layout}), and hence to scan the
overlap integral of the colliding beams at different relative beam positions, while measuring the interaction rate.  
This method---first applied at the \texttt{CERN} \texttt{ISR}~\cite{vanderMeer:296752}---has been widely used by all major \texttt{LHC} experiments 
during Run 1~\cite{Abelev:2014epa,ALICE-PUBLIC-2017-002,Aad:2013ucp,Aaij:2014ida,CMS-PAS-LUM-13-001} and more recently at 13\,\TeV~\cite{ALICE-PUBLIC-2016-002,ALICE-PUBLIC-2016-005,CMS-PAS-LUM-15-001}\,~\citeTH{CMS-PAS-LUM-16-001,CMS-PAS-LUM-17-001}. The BI method~\cite{BALAGURA2011634} is based on reconstructing primary vertices 
from interactions between one beam fixed in the rest-frame of the detector and the other one consecutively moving in $x$ and $y$. 
The shapes obtained by the distribution of vertices can be analytically convolved with vertex position resolution models, 
and can be used to determine the overlap integral accounting for genuine nonfactorizabilities. 

In both methods, data recorded by \texttt{CMS} are used in conjunction with input from the \texttt{LHC} beam instrumentation. 
While beam-current monitoring currently achieves sub-percent level precision, single-beam profile measurements, e.g, Ref.~\cite{fanouria_overview}, 
are challenging because of instrumental resolution and limitations on optical models of the collider lattice. 

Lately, \texttt{LHC} measurements, that resorted to synchrotron-light telescopes as transverse and longitudinal beam-profile monitors, 
achieved a remarkably good ($\mathcal{O}$(5\%)) agreement with the absolute luminosity scales of both \texttt{ATLAS} and \texttt{CMS}; discrepancies though appeared indicating the impact driven by instrumental calibration. 
Figure~\ref{fig:fanouria} shows an example comparison between the peak luminosity values calculated (crosses) using the measured bunch parameters, i.e., 
transverse emittance, bunch intensity and length, and the average measured peak luminosity provided by \texttt{ATLAS} (blue circles) and \texttt{CMS} (red circles) during 2016~\cite{fanouria}. 
The bottom plot shows the luminosity imbalance between the two experiments using the same marker convention. 
During the first part of the year, before the transition to the compression merging and splitting (BCMS) production scheme, 
very good agreement between the calculated and measured peak luminosity is observed. 
After the transition to BCMS, even though the calculated and measured imbalance agrees well, the absolute values start to diverge. 
In the third part, after the crossing angle reduction, a disagreement is observed both in absolute values and in imbalance. 
An update in the calibration of the beam synchrotron radiation telescope (\texttt{BSRT}) system was performed both before the transition to BCMS and the crossing angle change. 

\begin{figure*}[!tbh]
  \centering
  \includegraphics[width=0.9\textwidth]{figures/luminosity/fanouria.png}
  \caption[Calculated and measured peak instantaneous luminosity follow-up throughout the 2016 year]{\label{fig:fanouria}
     Using the measured bunch parameters, i.e., transverse emittances, bunch intensity and length, the calculated peak luminosity values (crosses)
     at the beginning of stable beams are compared to the average measured peak luminosity provided by the \texttt{ATLAS} (blue circles) and \texttt{CMS} (red circles) during 2016~\cite{fanouria}.
     The drop in peak luminosity for Fills 5219, 5222, 5223, and 5433 corresponds to machine development conditions~\cite{CERN-ACC-NOTE-2017-0029,ecloud} 
     for studying electron cloud effects on the \texttt{LHC} performance.
     %https://cds.cern.ch/record/2260999/files/CERN-ACC-NOTE-2017-0029.pdf
     %https://cds.cern.ch/record/2293524?ln=en
    }
\end{figure*}
