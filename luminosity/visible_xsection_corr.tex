\subsection{Extraction of the visible cross section}
\label{sec:PCCxsection}

The size of the beam overlap is measured by fitting the luminometer rate measurements, normalized by the bunch current product, as a function of beam--beam separation.  
The scan curves are fitted with a double-Gaussian model with an additional constant term, whose purpose is to subtract pedestals from the background rates.
A more accurate estimate of the constant term is obtained measuring the beam overlap by simultaneously fitting the PCC and reconstructed PV (``vertex counting'') rate measurements;
PCC data are available only during the \myDate vdM scan program.
The effective horizontal and vertical widths of the beam ($\Sigma_x$ and $\Sigma_y$) as well as the normalized rates ($R_x$, $R_y$) are obtained per scan per BCID. 
For each BCID the visible cross sections are then measured using
\begin{equation}
\sigmaVis = 2\pi \Sigma_x \Sigma_y \mu_{\textrm{vis}}\, ,
\end{equation}
where
\begin{equation}
\mu_{\textrm{vis}} = \frac{1}{2}(R_x + R_y)\, ,
\end{equation}
with $R_{x,y}$ denoting the amplitudes of the fitted scan curves, i.e.,  $R_{x} \equiv R(\Delta x=0, \Delta y_0)$ and $R_{y} \equiv R(\Delta x_0, \Delta y=0)$.

Example fits for BCID 1215 in $\mathrm{Y}_3$ and $\mathrm{X}_3$ are shown in Figs.~\ref{fig:pccXYFits} and \ref{fig:vtxXYFits} for PCC and vertex counting, respectively, during the \myDate vdM program. For vertex selection in the vertex counting method the standard \texttt{CMS} definition of good vertices (see Section~\ref{sec:tracking}) is applied. 
The fitted $\Sigma_x$ and $\Sigma_y$ parameters are shown in Fig.~\ref{fig:SimFit_Sigma}, while the resulting \sigmaVis are illustrated in Fig.~\ref{fig:SimFit_PCCResultsUnCorr}. 
For the \myDate vdM program all luminometers produce compatible estimates: the percent differences between any two luminometers are statistically consistent with zero. 
For the \mypPbDate vdM program the percent differences between any two luminometers is consistent with zero at the level of 0.5 and 0.2\% in the Pbp and pPb period, respectively, 
while a scan-to-scan variation is also assigned based on the root-mean-square of the measured \sigmaVis, which is found to be 0.6 (1.0)\% in the Pbp (pPb) period, respectively.


\begin{figure*}[tbh]
\begin{minipage}{.5\linewidth}
\centering
 \subfloat[]{\label{pccXYFits:a}\includegraphics[scale=0.3]{figures/luminosity/SimCapSigma_PCC_PCCAndVtx_FittedGraphs_13.pdf}}
\end{minipage}%
\begin{minipage}{.5\linewidth}
\centering
\subfloat[]{\label{pccXYFits:b}\includegraphics[scale=0.3]{figures/luminosity/SimCapSigma_PCC_PCCAndVtx_FittedGraphs_19.pdf}}
\end{minipage}\par\medskip
 \caption[Examples of the fitted scan curves based on the pixel-cluster counting method in the \myDate vdM period]{
   Examples of fitted scan curves, i.e., normalized rates recorded based on the PCC method as a function of the beam separation ($\Delta$) in (a) $\mathrm{Y}_3$ and (b) $\mathrm{X}_3$ scans
   in the \myDate vdM period~\citeTH{CMS-PAS-LUM-16-001}.  
   The applied fit model is represented as the green and red, blue, and black curves corresponding to the two Gaussian components, the constant term, and their sum, respectively. 
   The values and the statistical uncertainty in the fitted parameters are shown on the legend, along with the reduced $\chi^2$. 
   The bottom panels include the residuals, i.e., the difference between the measured and fitted values divided by the statistical uncertainty.
   The fit is performed simultaneously with the rates of Fig.~\ref{fig:vtxXYFits}.
}
\label{fig:pccXYFits}
\end{figure*}

\begin{figure*}[tbh]
\begin{minipage}{.5\linewidth}
\centering
\subfloat[]{\label{vtxXYFits:a}\includegraphics[scale=0.3]{figures/luminosity/SimCapSigma_TrkVtx_PCCAndVtx_FittedGraphs_13.pdf}}
\end{minipage}%
\begin{minipage}{.5\linewidth}
\centering
\subfloat[]{\label{vtxXYFits:b}\includegraphics[scale=0.3]{figures/luminosity/SimCapSigma_TrkVtx_PCCAndVtx_FittedGraphs_19.pdf}}
\end{minipage}\par\medskip
\caption[Examples of the fitted scan curves based on the vertex counting method in the \myDate vdM period]{
  Examples of fitted scan curves, i.e., normalized rates recorded based on the vertex counting method as a function of the beam separation ($\Delta$) in (a) $\mathrm{Y}_3$  and  (b) $\mathrm{X}_3$
  scans in the \myDate vdM period~\citeTH{CMS-PAS-LUM-16-001}.
  The applied fit model is represented as the green and red, blue, and black curves corresponding to the two Gaussian components, the constant term, and their sum, respectively. 
  The values and the statistical uncertainty in the fitted parameters are shown on the legend, along with the reduced $\chi^2$. 
  The bottom panels include the residuals, i.e., the difference between the measured and fitted values divided by the statistical uncertainty.
  The fit is performed simultaneously with the rates of Fig.~\ref{fig:pccXYFits}.
}
\label{fig:vtxXYFits}
\end{figure*}


\begin{figure}
\begin{minipage}{.5\linewidth}
\centering
\subfloat[]{\label{SimFit_Sigma_Chi2Norm:a}\includegraphics[scale=0.4]{figures/luminosity/plots_SimCapSigma_PCCAndVtx_4634_14.pdf}}
\end{minipage}%
\begin{minipage}{.5\linewidth}
\centering
\subfloat[]{\label{SimFit_Sigma_Chi2Norm:b}\includegraphics[scale=0.4]{figures/luminosity/plots_SimCapSigma_PCCAndVtx_4634_13.pdf}}
\end{minipage}\par\medskip

\caption[The measured convolved beam sizes from the PCC method in the \myDate vdM period]{
  The extracted $\Sigma_{x/y}$ (a,b) from the PCC method as a function of BCID (a) and scan sequence (b) in the \myDate vdM period~\citeTH{CMS-PAS-LUM-16-001}.
  The sequential naming convention adheres to the horizontal and vertical scan sequence as illustrated in Fig.~\ref{fig:BeamPos}.
}
\label{fig:SimFit_Sigma}
\end{figure}


\begin{figure}
\begin{minipage}{.5\linewidth}
\centering
\subfloat[]{\label{SimFit_PCCResultsUnCorr:a}\includegraphics[width=\textwidth]{figures/luminosity/xsecsPerBCID_PCC.pdf}}
\end{minipage}%
\begin{minipage}{.5\linewidth}
\centering
\subfloat[]{\label{SimFit_PCCResultsUnCorr:b}\includegraphics[width=\textwidth]{figures/luminosity/xsecsPerScan_PCC.pdf}}
\end{minipage}\par\medskip
\centering
\subfloat[]{\label{SimFit_PCCResultsUnCorr:c}\includegraphics[width=0.55\textwidth]{figures/luminosity/xsecs_PCC.pdf}}
\caption[The measured visible cross section from the PCC method in the \myDate vdM period]{
  The weighted average of PCC \sigmaVis per BCID (a) and scan (b), and separately in each BCID and scan (c) in the \myDate vdM period~\citeTH{CMS-PAS-LUM-16-001}.
  Errors are statistical only.}
\label{fig:SimFit_PCCResultsUnCorr}
\end{figure}