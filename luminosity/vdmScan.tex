
\subsection{Formalism of absolute luminosity from separating the beams}
\label{sec:vdmScan}

The bunch luminosity $\mathcal{L_{\textrm{b}}}$ produced by one colliding bunch pair, with time- and position-dependent density functions,
$\rho_1(x,y,z,t)$ and $\rho_2(x,y,z,t)$, is given by~\cite{BALAGURA2011634}
\begin{equation}
\label{eq:lumidef}
\mathcal{L_{\textrm{b}}} = K N_1 N_2 f_{\textrm{r}}\int_{-\infty}^{\infty}\rho_1(x,y,z,t)\rho_2(x,y,z,t) \,\mathrm{d}x\,\mathrm{d}y\,\mathrm{d}z\,\mathrm{d}t\, ,
\end{equation}
where $N_1$ and $N_2$ are the numbers of protons in the two colliding bunches respectively, \lhcOrbit is the bunch 
orbit frequency around the \texttt{LHC} ring, and $\rho_{1,2}$ are the bunch proton densities normalized to unity at any time $t$. 
While the bunch populations can be measured to good precision directly, a precise measurement of $\rho_{1,2}$ is difficult.
\begin{figure*}[!tbh]
  \centering
  \includegraphics[width=0.9\textwidth]{figures/luminosity/coordinates.png}
  \caption[Definition of coordinates and beam trajectories typically used in luminosity measurement]{\label{fig:coordinates}
    Definition of coordinates and beam trajectories~\cite{BALAGURA2011634}; 
    the laboratory frame $xyz$ is defined such that the $x$ axis points in the direction of $\vec{v}_1+\vec{v}_2$,
    the $y$ axis in that of $\vec{v}_1 \times \vec{v}_2$, and the $z$ axis in that of $\vec{v}_1 - \vec{v}_2$, where $\vec{v}_1=c\,\hat{v}_1(sina,0,cosa)$ and $\vec{v}_2=c\,\hat{v}_2(sina,0,-cosa)$.
    The points $(x_1,z_1)$ and $(x_2,z_2)$ are the positions of the bunch centers at time $t=0$. 
    For simplicity, the former is on the intersect of the two beam trajectories in the $x$--$z$ plane. 
    The third dimension ($y$) is here suppressed.
    }
\end{figure*}
%\begin{figure*}[!tbh]
%  \centering
%  \includegraphics[width=0.9\textwidth]{figures/luminosity/overlap_v2.png}
%  \caption[Illustration of a bunch crossing at the interaction point]{\label{fig:overlap}
%    Illustration of a bunch crossing; 
%    in principle, the bunch luminosity can be inferred from the bunch populations and the transverse beam sizes at the IP.
%    The beam-overlap or ``effective'' area is extracted during the well-established technique of beam-separation scans.
%    }
%\end{figure*}
The kinematic factor $K$ 
\begin{equation}
\label{eq:k_factor}
K = \sqrt{\lvert \vec{v}_1-\vec{v}_2 \rvert ^2-\frac{\vec{v}_1 \times \vec{v}_2}{c^2}}\, ,
\end{equation}
can be readily attained in case of equal and relativistic beam velocities, i.e., $\lvert \vec{v}_1 \rvert=\lvert \vec{v}_2 \rvert =c$, and Eq.\,(\ref{eq:lumidef}) can be thus simplified to
\begin{equation}
\label{eq:lumidef_simpl}
\mathcal{L_{\textrm{b}}} = (2\mathrm{cos}^2\theta_\textrm{C}) N_1 N_2 f_{\textrm{r}}\int_{-\infty}^{\infty}\rho_1(x,y,z,t)\rho_2(x,y,z,t) \,\mathrm{d}x\,\mathrm{d}y\,\mathrm{d}z\,\mathrm{d}t\, ,
\end{equation}
in terms of a half-crossing angle $\theta_\textrm{C}$ defined by the two beam trajectories without loss of generality in the $x$--$z$ plane (Fig.~\ref{fig:coordinates}). 
The time-dependent beam overlap $f=\rho_1\rho_2$ equals simply to the product of the normalized particle-density distributions, and it can be approximated as
\begin{equation}
\label{eq:densities}
f(x,y,z,t) = \frac{1}{(2\pi)^3\sigma_{xi}\sigma_{yi}\sigma_{zi}}\mathrm{exp}\left[-\left(\frac{(x-x_i)^2}{2\sigma_{xi}^2} + \frac{(y-y_i)^2}{2\sigma_{yi}^2} + \frac{(z-ct)^2}{2\sigma_{zi}^2}\right)\right]\, ,
\end{equation}
assuming perfectly Gaussian bunch profiles. The values $\sigma_{xi}$, $\sigma_{yi}$, and $\sigma_{zi}$ are the transverse and longitudinal beam sizes ($i=1,2$) in the frame as defined in Fig.~\ref{fig:coordinates}, whereas $x_i$ and $y_i$ correspond to the transverse positions of the bunch centroids at the nominal collision point ($t=0$).
The time-integrated beam overlap distribution reveals cross terms implying that the resulting distribution is not exactly factorizable in a $x$- and $z$-dependent Gaussian profile.
For small crossing angles and $\sigma_{zi}\gg\sigma_{xi}$---two conditions typically valid at \texttt{LHC}---the longitudinal dependence of the transverse beam size is raised, and
one can always find a rotated reference in the crossing plane in which the beam overlap is the product of two profiles each depending only on one position variable.
For beams colliding with a half crossing-angle $a$ in the $x$--$z$ plane, and with a relative transverse offsets $\Delta x$ ($\Delta y$) in the $x$ ($y$) direction, 
it can be therefore shown that integrating Eq.\,(\ref{eq:densities}) leads to
\begin{equation}
\label{eq:lumidefgen}
\mathcal{L_{\textrm{b}}} = 2\mathrm{cos}^2\theta_\textrm{C} N_1 N_2 f_{\textrm{r}}\int_{-\infty}^{\infty}f(x,y,z)\mathrm{d}y\,\mathrm{d}z\ = \frac{\mathrm{cos}^2\theta_\textrm{C} N_1 N_2 f_{\textrm{r}}}{2\pi\Sigma_{x}\Sigma_{y}}\mathrm{exp}\left[-\left(\frac{\Delta x^2}{2\Sigma_{x}^2} + \frac{\Delta y^2}{2\Sigma_{y}^2} \right)\right]\, .
\end{equation}
More specifically, the variances $\sigma_{xi}^2$ and $\sigma_{yi}^2$ have been convolved to 
\begin{equation}
\label{eq:capsigmaX}
\Sigma_x = \sqrt{(\sigma_{x1}^2+\sigma_{x2}^2)cos^2\theta_\textrm{C} + (\sigma_{z1}^2+\sigma_{z2}^2)\mathrm{sin}^2\theta_\textrm{C}}\, ,
\end{equation}
in the crossing plane, and
\begin{equation}
\label{eq:capsigmaY}
\Sigma_y = \sqrt{(\sigma_{y1}^2+\sigma_{y2}^2)}\, ,
\end{equation}
otherwise. 


The vdM scan method allows to measure the beam overlap in Eq.\,(\ref{eq:densities}) assuming that the two bunch densities factorize in $x$ and $y$, meaning
%For the \texttt{vdM} scan method to be valid, it is assumed that the two bunch proton densities factorize in $x$ and $y$, i.e.
\begin{equation}
\label{eq:fact}
\int_{-\infty}^{\infty}\rho_1(x,y)\rho_2(x+\Delta x,y+\Delta y) \,\mathrm{d}x\,\mathrm{d}y  = \int_{-\infty}^{\infty} \rho_1(x)\rho_2(x+\Delta x) \,\mathrm{d}x \,  \int_{-\infty}^{\infty} \rho_1(y)\rho_2(y+\Delta y) \,\mathrm{d}y\, .
\end{equation}
The estimate of the possible bias introduced by this assumption is calculated in Section~\ref{sec:xyCor}. Both sides of 
Eq.\,(\ref{eq:lumidef}) can then be integrated independently in $\Delta x$ and $\Delta y$, while the separation in the
other direction is kept fixed at arbitrary values $\Delta y_0$ and $\Delta x_0$, respectively. For $\theta_\textrm{C}=0$
\begin{equation}
\label{eq:evaldense}
N_1 N_2 f_{\textrm{r}} \int_{-\infty}^{\infty} \rho_1(y)\rho_2(y+\Delta y_0) \,\mathrm{d}y = \int_{-\infty}^{\infty} \mathcal{L_{\textrm{b}}}(\Delta x, \Delta y_0) \,\mathrm{d}(\Delta x)\, ,
\end{equation}
and therefore
\begin{equation}
\label{eq:evaldense2}
\int_{-\infty}^{\infty} \rho_1(x)\rho_2(x+\Delta x_0) \,\mathrm{d}x = \frac{\mathcal{L_{\textrm{b}}} (\Delta x_0, \Delta y_0)}{\int_{-\infty}^{\infty} \mathcal{L_{\textrm{b}}}(\Delta x, \Delta y_0) \,\mathrm{d}(\Delta x)}\, .
\end{equation}
Likewise for $y$. Experimentally the integration over $\Delta x$ and $\Delta y$ is implemented by scanning the two 
beams against each other and the integral in the denominator of Eq.\,(\ref{eq:evaldense2}) is evaluated by measuring 
the detector rate as a function of the beam-beam separation, the so-called ``scan curves.'' After replacing the factorized in $x$ and $y$ beam overlap integral
according to Eq.\,(\ref{eq:evaldense2}), Eq.\,(\ref{eq:lumidefgen}) becomes for any head-on collision: 
\begin{equation}
\label{eq:lumidefVdM}
\mathcal{L_{\textrm{b}}} (\Delta x=0, \Delta y=0) = N_1 N_2 f_{\textrm{r}}  \frac{R (\Delta x=0, \Delta y_0) R (\Delta x_0, \Delta y=0)}{\int_{-\infty}^{\infty} R (\Delta x, \Delta y_0) \,\mathrm{d}(\Delta x) \int_{-\infty}^{\infty} R(\Delta x_0, \Delta y) \,\mathrm{d}(\Delta y)}\, ,
\end{equation}
where the luminosity is expressed in terms of the rate $R(\Delta x, \Delta y)$ measured when the two beams are separated by values
$\Delta x$ and $\Delta y$, respectively. The integrals of the scan curves can be obtained from the convolved beam widths $\Sigma_x$ and $\Sigma_y$ 
\begin{equation}
\label{eq:capsigmaX_vdm}
\Sigma_x = \frac{1}{\sqrt{2\pi}}\frac{\int_{-\infty}^{\infty} R (\Delta x, \Delta y_0) \,\mathrm{d}(\Delta x)}{R (\Delta x=0, \Delta y_0)}\, .
\end{equation}
Likewise for $y$. 
In the case of Gaussian luminosity curves, the convolved beam width coincides  with  the  standard  deviation  of  that  distribution. Equation\,(\ref{eq:capsigmaX_vdm}) is generic though
meaning $\Sigma_x$ and $\Sigma_y$ depend only upon the area under the luminosity curve.
% and make no assumption as to the shape of that curve.
 The appealing feature of the vdM method is therefore no assumption about the shape of the scan curve is made.
%; $\Sigma$ can be interpreted as the standard deviation of a Gaussian-like scan curve. 
 The bunch luminosity at zero separation is extracted from machine parameters by performing a pair of beam-separations scans 
\begin{equation}
\label{eq:lumidefVdMcSig}
\mathcal{L_{\textrm{b}}} (\Delta x=0, \Delta y=0) =\frac{N_1 N_2 f_{\textrm{r}}}{2\pi\Sigma_x \Sigma_y}\, ,
\end{equation}
such that the final formula used to measure the so-called ``visible cross section'' is
\begin{equation}
\label{eq:sigVisX_vdm}
 \sigmaVis=\frac{2\pi\Sigma_x \Sigma_y R(\Delta x=0, \Delta y=0)}{N_1 N_2 f_{\textrm{r}}}\, .
\end{equation}
%retrieving Eq.~\ref{eq:sigVisX} for $\mu_{\textrm{vis}} \equiv R(\Delta x=0, \Delta y=0)$. 
In contrast to beam-profile measurements, $\Sigma_x$ and $\Sigma_y$ are directly determined from fits of the scan curves based on detector rate measurements during the vdM scans. 
While the convolved beam widths are the same for all detectors, the peaks of the corresponding scan curves depend on the detector, meaning the exercise is repeated per detector. 
The comparison of the $\Sigma_x$ and $\Sigma_y$ as estimated from the scan curves of different detectors represents a crucial cross-check.
