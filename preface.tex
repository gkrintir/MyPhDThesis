%%%% PUNCTUATION:   
\newcommand{\NA}{\ensuremath{\text{---}}}
\newcommand{\x}{\ensuremath{\phantom{0}}}

%%%%VARIABLES/NOTATIONS

\newcommand{\etalab}{\ensuremath{\eta_{\textrm{lab}}}\xspace}
\newcommand{\etaCM}{\ensuremath{\eta_{\textrm{CM}}}\xspace}
\newcommand{\etaMuref}{\ensuremath{\eta^{\mu}_{\textrm{ref}}}\xspace}
\newcommand{\etaMulab}{\ensuremath{\eta^{\mu}_{\textrm{lab}}}\xspace}
\newcommand{\etaMuCM}{\ensuremath{\eta^{\mu}_{\textrm{CM}}}\xspace}
\newcommand{\sigmaVis}{\ensuremath{\sigma_{\textrm{vis}}}\xspace}
\newcommand{\sigmaVisPCC}{\ensuremath{\sigma_{\textrm{vis}}^{\textrm{PCC}}}\xspace}
\newcommand{\rateDXDY}{\ensuremath{R(\Delta x_0, \Delta y_0)}\xspace}
\newcommand{\stt}{\ensuremath{\sigma_{\ttbar}}\xspace}
\newcommand{\sttfig}{\ensuremath{\sigmat_{\ttbarfig}}\xspace}
%\newcommand{\jj}{\ensuremath{ \mathrm{j} \mathrm{j}^{\prime}}\xspace}
\newcommand{\jj}{\ensuremath{ jj^{\prime}}\xspace}
\newcommand{\mjj}{\ensuremath{m_{\jj}}\xspace}
\newcommand{\mtop}{\ensuremath{m_\text{top}}\xspace}
\newcommand{\et}{\ensuremath{E_{\textrm{T}}}\xspace}
\newcommand{\eT}{\ensuremath{E_{\textrm{T}}}\xspace}
\newcommand{\ET}{\ensuremath{E_{\textrm{T}}}\xspace}
\newcommand{\kt}{\ensuremath{k_{\textrm{T}}}\xspace}
\newcommand{\pt}{\ensuremath{p_{\textrm{T}}}\xspace}
\newcommand{\vecpt}{\ensuremath{\vec{p}_\textrm{T}}\xspace}
\newcommand{\ptmiss}{\ensuremath{p_{\textrm{T}}^{\textrm{miss}}}\xspace}
\newcommand{\ptmissfig}{\ensuremath{p_{\mathit{T}}^{\mathit{miss}}}\xspace}
\newcommand{\ptmisssquared}{\ensuremath{p_{\textrm{T}}^{\textrm{miss},2}}\xspace}
\newcommand{\ptmisssquaredfig}{\ensuremath{p_{\mathit{T}}^{\mathit{miss},2}}\xspace}
\newcommand{\vecptmiss}{\ensuremath{\vec{p}_\textrm{T}^{\kern1pt\textrm{miss}}}\xspace}
\newcommand{\vecptmissfig}{\ensuremath{\vec{p}_\mathit{T}^{\kern1pt\textrm{miss}}}\xspace}
\newcommand{\met}{\ensuremath{p_{\textrm{T}}^{\textrm{miss}}}\xspace}
\newcommand{\qt}{\ensuremath{{q}_{\textrm{T}}}\xspace}
\newcommand{\vqt}{\ensuremath{\vec{q}_{\textrm{T}}}\xspace}
\newcommand{\vut}{\ensuremath{\vec{u}_{\textrm{T}}}\xspace}
\newcommand{\upar}{\ensuremath{u_{\parallel}}\xspace}
\newcommand{\uperp}{\ensuremath{u_{\bot}}\xspace}
\newcommand{\redupara}{\ensuremath{u_{\parallel}+\qt}\xspace}
\newcommand{\mt}{\ensuremath{m_{\ell\nu \textrm{b}}}\xspace}
\newcommand{\mtw}{\ensuremath{m_{\textrm{T}}}\xspace}
\newcommand{\mTW}{\ensuremath{m_{\textrm{T}}}\xspace}
\newcommand{\mW}{\ensuremath{m_{\mathrm{W}}}\xspace}
\newcommand{\mT}{\ensuremath{m_{\textrm{T}}}\xspace}
\newcommand{\PFrelIso}{\ensuremath{I_{\textrm{rel}}}\xspace}
\newcommand{\PFrelIsoRho}{\ensuremath{I_{\textrm{rel}^{\rho, \textrm{corr}}}}\xspace}
\newcommand{\PFrelIsoDb}{\ensuremath{I_{\textrm{rel}^{\delta\beta, \textrm{corr}}}}\xspace}
\newcommand{\pfPhotonIso}{\ensuremath{I^{\gamma}}\xspace}
\newcommand{\pfChargedHadronIso}{\ensuremath{I^{\textrm{ch.\,h}}}\xspace}
\newcommand{\pfNeutralHadronIso}{\ensuremath{I^{\textrm{n.\,h}}}\xspace}
\newcommand{\pfPU}{\ensuremath{I^{\textrm{PU}}\xspace}}
\newcommand{\sumPUPT}{\ensuremath{\sum p_{\textrm{T}}^{\textrm{PU}}}\xspace}
\newcommand{\sumPUPt}{\sumPUPT}
\newcommand{\rhoEnergy}{\ensuremath{\rho \times A}}
\newcommand{\murelIso}{\ensuremath{I_{\textrm{rel}}^{\mu}\xspace}}
\providecommand{\PN}{\ensuremath{\mathrm{N}}\xspace}

%%%% GENERAL
\newcommand{\QCD}{\ensuremath{\mathrm{QCD}\xspace}}
\newcommand{\QED}{\ensuremath{\mathrm{QED}\xspace}}
\newcommand{\QGP}{\ensuremath{\mathrm{QGP}\xspace}}
\newcommand{\DAQ}{\ensuremath{\mathrm{DAQ}\xspace}}
\newcommand{\DQM}{\ensuremath{\mathrm{DQM}\xspace}}
\newcommand{\mrad}{\ensuremath{\,\text{mrad}}\xspace}
\newcommand{\Order}{$\mathcal{O}$}
\newcommand{\MeV}{\ensuremath{\mathrm{MeV}\xspace}}
\newcommand{\GeV}{\ensuremath{\mathrm{GeV}\xspace}}
\newcommand{\TeV}{\ensuremath{\mathrm{TeV}\xspace}}
\newcommand{\mum}{\ensuremath{\mu\mathrm{m}\xspace}}
\newcommand{\mumsq}{\ensuremath{\mu\mathrm{m}^{2}}\xspace}
\newcommand{\ppbar}{\ensuremath{\mathrm{p\overline{p}}\xspace}}
\newcommand{\ttbar}{\ensuremath{\mathrm{t\overline{t}}\xspace}}
\newcommand{\ttbarfig}{\ensuremath{\mathit{t\overline{t}}\xspace}}
%\newcommand{\tt}{\ensuremath{\mathrm{t\bar{t}}\xspace}}
\newcommand{\rootsNN}{\ensuremath{\sqrt{\smash[b]{s_{_{\mathrm{NN}}}}}}\xspace}
\newcommand{\rootsNNfig}{\ensuremath{\sqrt{\smash[b]{s_{_{\mathit{NN}}}}}}\xspace}
%\newcommand{\rootsNN}{\ensuremath{\sqrt{s_{_{\mathrm{NN}}}}}\xspace}
\newcommand{\roots}{\ensuremath{\sqrt{\smash[b]{s}}}\xspace}
\newcommand{\sqrts}{\ensuremath{\sqrt{\smash[b]{s}}}\xspace}
\newcommand{\pPb}{\ensuremath{\textrm{p}\textrm{Pb}}\xspace}
\newcommand{\pp}{\ensuremath{\textrm{p}\textrm{p}}\xspace}
\newcommand{\pn}{\ensuremath{\textrm{p}\textrm{n}}\xspace}
\newcommand{\PYTHIA}{\textsc{pythia}\xspace}
\newcommand{\HERWIGpp}{\textsc{herwig++}\xspace}
\newcommand{\HERWIG}{\textsc{herwig}\xspace}
\newcommand{\mcfm}{\textsc{mcfm}\xspace}
\newcommand{\EPOS}{\textsc{epos}\xspace}
\newcommand{\EPOSLHC}{\textsc{epos-lhc}\xspace}
\newcommand{\madgraph}{\textsc{MadGraph}\xspace}
\newcommand{\amcatnlo}{\textsc{MadGraph5\_\textup{a}MC@NLO}\xspace}%\newcommand{\amcpy}{\textsc{MG5}\_a\textsc{MC}}
%\newcommand{\amcatnlo}{\sc MadGraph5_aMC@NLO}
\newcommand{\POWHEG}{\textsc{powheg}\xspace}
\newcommand{\FEWZ}{\textsc{fewz}\xspace}
\newcommand{\GEANTfour}{\textsc{geant4}\xspace}
\newcommand{\NNPDFzero}{NNPDFf3.0\xspace}
\newcommand{\NNPDFone}{NNPDF3.1\xspace}
\newcommand{\ABMP}{ABMP16\xspace}
\newcommand{\MMHT}{MMHT14\xspace}
\newcommand{\CT}{CT14\xspace}
\newcommand{\CTten}{CT10\xspace}
\newcommand{\HERAPDF}{HERAPDF2.0\xspace}
\newcommand{\JR}{JR14\xspace}
\newcommand{\EPPS}{EPPS16\xspace}
\newcommand{\EPS}{EPS09\xspace}
\newcommand{\nCTEQ}{nCTEQ15\xspace}
\newcommand{\empm}{\ensuremath{\mathrm{e}^\pm \mu^\mp}\xspace}
\newcommand{\empmfig}{\ensuremath{\mathit{e}^\pm \mu^\mp}\xspace}
\newcommand{\mmpm}{\ensuremath{\mu^\pm \mu^\mp}\xspace}
\newcommand{\empe}{\ensuremath{\mathrm{e}^\pm \mathrm{e}^\mp}\xspace}
\newcommand{\Pep}{\ensuremath{\mathrm{e}^{+}}\xspace}
\newcommand{\Pem}{\ensuremath{\mathrm{e}^{-}}\xspace}
\newcommand{\PZ}{\ensuremath{\mathrm{Z}}}
\newcommand{\cPZ}{\ensuremath{\mathrm{Z}}}
\newcommand{\Zee}{\ensuremath{\cPZ \rightarrow \Pe \Pe}}
\newcommand{\Zmumu}{\ensuremath{\cPZ \rightarrow \mu \mu}}
\newcommand{\mee}{\ensuremath{m_{\Pe \Pe}}}
\newcommand{\mmumu}{\ensuremath{m_{\mu \mu}}}
\newcommand{\PW}{\ensuremath{\mathrm{W}}}
\newcommand{\Pp}{\ensuremath{\mathrm{p}}\xspace}
\newcommand{\Pe}{\ensuremath{\mathrm{e}}\xspace}
\newcommand{\cPg}{\ensuremath{\mathrm{g}}\xspace}
\newcommand{\Pg}{\ensuremath{\mathrm{g}}\xspace}
\newcommand{\cPq}{\ensuremath{\mathrm{q}}\xspace}
\newcommand{\cPqbar}{\ensuremath{\mathrm{\overline{q}'}}\xspace}
\newcommand{\cPb}{\ensuremath{\mathrm{b}}\xspace}
\newcommand{\cPqu}{\ensuremath{\mathrm{u}}\xspace}
\newcommand{\cPqd}{\ensuremath{\mathrm{d}}\xspace}
\newcommand{\cPqs}{\ensuremath{\mathrm{s}}\xspace}
\newcommand{\cPqc}{\ensuremath{\mathrm{c}}\xspace}
\newcommand{\cPqb}{\ensuremath{\mathrm{b}}\xspace}
\newcommand{\cPqt}{\ensuremath{\mathrm{t}}\xspace}
\newcommand{\cPaq}{\ensuremath{\mathrm{\overline{q}}}\xspace}
\newcommand{\cPaqu}{\ensuremath{\mathrm{\overline{u}}}\xspace}
\newcommand{\cPaqd}{\ensuremath{\mathrm{\overline{d}}}\xspace}
\newcommand{\cPaqs}{\ensuremath{\mathrm{\overline{s}}}\xspace}
\newcommand{\mll}{\ensuremath{M_{\ell^{\pm}\ell^{\mp}}}\xspace}
\newcommand{\WV}{\ensuremath{\mathrm{W}\mathrm{V}}\xspace}
\newcommand{\WW}{\ensuremath{\mathrm{W}\mathrm{W}}\xspace}
\newcommand{\WZ}{\ensuremath{\mathrm{W}\mathrm{Z}}\xspace}
\newcommand{\ZZ}{\ensuremath{\mathrm{Z}\mathrm{Z}}\xspace}
\newcommand{\tW}{\ensuremath{\mathrm{t}\mathrm{W}}\xspace}
\newcommand{\nonW}{\ensuremath{{\textrm{non-W/Z}{}}}\xspace}
\newcommand{\minDeltaR}{\ensuremath{\Delta R_{\text{min}}(j,j')}\xspace}
%%%% Electron Variables
\newcommand{\sigmaIetaIeta}{\ensuremath{\sigma_{\eta\eta}}}
\newcommand{\dEtaIn}{\ensuremath{\Delta \eta_{in}}}
\newcommand{\dPhiIn}{\ensuremath{\Delta \phi_{in}}}
\newcommand{\hOverE}{\ensuremath{H/E}}
\newcommand{\relIso}{relIsoWithEA}
\newcommand{\ooEmooP}{\ensuremath{\lvert 1/E-1/p \rvert}}
\newcommand{\dxy}{\ensuremath{\lvert d_{0} \rvert}}
\newcommand{\dz}{\ensuremath{\lvert d_{z} \rvert}}

%\newcommand{\mhits}{expectedMissingInnerHits}
\newcommand{\mhits}{Missing inner hits}
\newcommand{\veto}{Pass conversion veto}

%%%% Fit and PDF parameterization
\newcommand{\Npass}{\ensuremath{N^{\textrm{P}}(\mee) = N_{\textrm{sig}} \times \varepsilon \times P^{\textrm{P}}_{\textrm{sig}}(\mee) +  N^{\textrm{P}}_{\textrm{bkg}} \times \varepsilon \times P^{\textrm{P}}_{\textrm{bkg}}(\mee)}}
\newcommand{\Nfail}{\ensuremath{N^{\textrm{F}}(\mee) = N_{\textrm{sig}} \times (1-\varepsilon) \times P^{\textrm{F}}_{\textrm{sig}}(\mee) +  N^{\textrm{F}}_{\textrm{bkg}} \times \varepsilon \times P^{\textrm{F}}_{\textrm{bkg}}(\mee)}}
\newcommand{\RooCMSShape}{\ensuremath{\mathrm{erfc}\left[\beta \times (\alpha - \mee)\right] \times \mathrm{exp}\left[(m_{\mathrm{Z}} - \mee) \times \gamma\right]}}
\newcommand{\tinCB}{\ensuremath{t=(\mee-\textrm{mean})/ \sigma_{1}}}
\newcommand{\tzeroinCB}{\ensuremath{t_0=(\mee-\textrm{mean})/ \sigma_{2}}}
\newcommand{\RooCBExGaussShape}
{\ensuremath
  {
    \begin{cases}
      \begin{array}{cc}
        \mathrm{exp}\left(-\frac{1}{2}t_{0}^{2}\right) & t>0\, ,\\
        \mathrm{exp}\left(-\frac{1}{2}t^{2}\right) & t>-|\alpha|\, ,\\
        \left(b-t\right)^{-n},\ b=n/|\alpha|-|\alpha| & {\textrm{otherwise}}\, .
    \end{array}\end{cases}
  }
}


%%%% NUMERICAL VALUES:
\newcommand{\highEnergy}{\ensuremath{\sqrt{s}=13\,\mathrm{TeV}}\xspace}
\newcommand{\myEnergy}{\ensuremath{\sqrt{s}=5.02\,\mathrm{TeV}}\xspace}
\newcommand{\lhcOrbit}{\ensuremath{f_{\textrm{r}}=11\,2455\,\mathrm{Hz}}\xspace}
\newcommand{\myFillNumber}{4634\xspace}
\newcommand{\myFillScheme}{\texttt{Multi\_44b\_22\_22\_22\_4bpi12inj}\xspace}
\newcommand{\myNBunchColl}{22\xspace}
\newcommand{\myXAngle}{\ensuremath{170\, \mu \mathrm{rad}}\xspace}
\newcommand{\myBetaStar}{\ensuremath{\beta^{*}=4\,{\mathrm{m}}}\xspace}
\newcommand{\myDate}{November 2015\xspace}
\newcommand{\myBCID}{(644, 1215, 2269, 2389, 2589)\xspace}
\newcommand{\mypPbEnergy}{\ensuremath{\sqrt{s_{\mathrm{NN}}}=8.16\,{\mathrm{TeV}}}\xspace}
\newcommand{\myPbpFillNumber}{5527\xspace}
\newcommand{\mypPbFillNumber}{5563\xspace}
\newcommand{\myPbpFillScheme}{\texttt{100\_200ns\_702p\_584Pb\_474\_216\_163\_20inj}\xspace}
\newcommand{\mypPbFillScheme}{\texttt{100\_200ns\_540Pb\_684p\_513\_224\_162\_20inj}\xspace}
\newcommand{\myPbpNBunchColl}{474\xspace}
\newcommand{\mypPbNBunchColl}{513\xspace}
\newcommand{\mypPbXAngle}{\ensuremath{140\,\mu\mathrm{rad}}\xspace}
\newcommand{\mypPbBetaStar}{\ensuremath{\beta^{*}=0.6\mathrm{m}}\xspace}
\newcommand{\mypPbDate}{November--December 2016\xspace}
\newcommand{\mypPbYear}{2016\xspace}
\newcommand{\myPbpBCID}{(177, 1420, 2311, 3015)\xspace}
\newcommand{\mypPbBCID}{(958, 1486, 2032, 2576)\xspace}
\newcommand{\lumiFB}{5.0\xspace}
\newcommand{\lumiunc}{?\,\%\xspace}
\newcommand{\runrange}{262081-262174,\xspace}
\newcommand{\runRangeFull}{\runrange}
\newcommand{\invmub}{\ensuremath{\mu{\mathrm{b}}^{-1}}\xspace}
\newcommand{\mubinv}{\ensuremath{\mu{\mathrm{b}}^{-1}}\xspace}
\newcommand{\invpb}{\ensuremath{{\mathrm{pb}}^{-1}}\xspace}
\newcommand{\pbinv}{\ensuremath{{\mathrm{pb}}^{-1}}\xspace}
\newcommand{\invnb}{\ensuremath{{\mathrm{nb}}^{-1}}\xspace}
\newcommand{\nbinv}{\ensuremath{{\mathrm{nb}}^{-1}}\xspace}
\newcommand{\invfb}{\ensuremath{ {\mathrm{fb}}^{-1}}\xspace}
\newcommand{\invab}{\ensuremath{ {\mathrm{ab}}^{-1}}\xspace}
\newcommand{\nb}{\ensuremath{{\mathrm{nb}}}\xspace}
\newcommand{\pb}{\ensuremath{{\mathrm{pb}}}\xspace}
\newcommand{\fb}{\ensuremath{{\mathrm{fb}}}\xspace}
\newcommand{\mylumi}{26\,\invpb\xspace}
\newcommand{\mylumipA}{\ensuremath{174\pm 6\,\nbinv}\xspace}
\newcommand{\mylumipAPre}{\ensuremath{174\pm 9\,\nbinv}\xspace}
\newcommand{\stat}{\ensuremath{\textrm{stat}\xspace}}
\newcommand{\syst}{\ensuremath{\textrm{syst}\xspace}}
\newcommand{\lumi}{\ensuremath{\textrm{lum}\xspace}}
\newcommand{\ecal}{\ensuremath{ \lvert\eta\rvert<2.5}\xspace}
\newcommand{\betrans}{\ensuremath{1.4442<\lvert\eta\rvert<1.560}\xspace}
%%%% RESULTS:                                                                                                 
%\newcommand{\PCCUnCorr}{\ensuremath{\sigma_{vis}^{PCC}=6.33\pm0.006\textrm{(stat.)}\ b}\xspace}
%\newcommand{\PCCCorr}{\ensuremath{\sigma_{vis}^{PCC}=6.33\pm0.006\textrm{(stat.)}\pm0.25{(syst.)}\ barn}\xspace}

\newcommand{\PCCUnCorr}{\ensuremath{\sigma_{\textrm{vis}}^{\textrm{PCC}}=6.16\pm0.005\textrm{(stat)}\ Barn}\xspace}
\newcommand{\PCCCorr}{\ensuremath{\sigma_{\textrm{vis}}^{\textrm{PCC}}=6.16\pm0.005\textrm{(stat)}\pm0.14\textrm{(syst.)}\ Barn}\xspace}




\begin{preface}

\InitialCharacter{T}he Large Hadron Collider (\texttt{LHC}) at \texttt{CERN} is a 26.7\,$\mathrm{km}$ two-ring, synchrotron accelerator and collider that initially approved for construction as a ``missing magnet machine'' 
in two stages and routinely started its operation in November 2009. By colliding high- and lower-intensity proton and lead---recently also xenon---beams with momentum of up to 6.5\, and $6.5$\,$Z$\,\TeV, 
\texttt{LHC} has set new world records at the beginning of 2015 and the end of 2016, respectively. 
The achieved particle momenta correspond to unrivaled regimes in terms of the stored beam energy in both the proton and heavy ion programs, reaching values of more than 310\,$\mathrm{MJ}$. 
To bend and focus such rigid beams \texttt{LHC} is equipped with over one thousand superconducting magnets, most of which are operated at temperatures as low as 1.9\,$\mathrm{K}$, 
which can quench, if tiny fractions of the stored beam energy are deposited inside their coils. This hence puts high demands on the collimation system, 
which so far provided excellent cleaning with proton beams, whereas the cleaning performance has been sufficient for the heavy ion operation, characterized by high production yield 
of effectively off-momentum ion fragments.

In most years the \texttt{LHC} apparatus is reconfigured for a month-long heavy ion run. 
However, asymmetric collisions were not included in the \texttt{LHC} design, and hence the physics case was based on a luminosity of $1.15 \times 10^{29}$\,$\mathrm{cm}^{-2}\mathrm{s}^{-1}$ 
at a beam energy of 7\,$Z$\,\TeV\ (``design'' parameters). Following up on a feasibility test and pilot physics Fill in October 2011 and September 2012, respectively, 
the first one month-long run took place in January 2013, meaning asymmetric proton-nucleus collisions remain a novel mode of operation at \texttt{LHC}. 
The 2015 operational period with heavy ions started with a ``reference'' proton run at 2.51\,\TeV\ to obtain the same center-of-mass energy as in the proton-nucleus run of 2013. 
For the same reason, the ensuing PbPb operation in November--December 2015 was carried out at an energy of 6.37\,$Z$\,\TeV. 
The second proton-nucleus collision run in November--December 2016 offered the tremendous opportunity to answer a range of crucial physics questions, yet opening up the possibility to measure, 
for the first time in heavy ion collisions, various large-mass elementary particles, like the top quark. 
Despite the complex strategy for repeated recommissioning and operation of \texttt{LHC}, the peak luminosity surpassed the design value by a factor 7.8, 
and the amount of integrated luminosity substantially exceeded the requests of the majority of the \texttt{LHC} experiments.

The Compact Muon Solenoid (\texttt{CMS}) is one of the seven experiments at \texttt{LHC}, featuring  a superconducting solenoid of 3.8\,$\mathrm{T}$. 
Within the solenoid volume are a silicon pixel and strip tracker, a lead tungstate crystal electromagnetic calorimeter, and a brass and scintillator hadron calorimeter, 
while forward calorimeters extend the pseudorapidity coverage up to $\pm 5.2$. Muons are detected in gas-ionization chambers embedded in the steel flux-return yoke outside the solenoid. 
\texttt{CMS} is therefore particularly suited for a global event description that aims to reconstruct and identify most of the produced particle types (photons, electrons, muons, charged and neutral hadrons), with an optimized combination of information from the various subdetectors. 
Events of interest are selected in real time using a two-tiered trigger system which reduces the event rate from the bunch crossing frequency to around 1\,$\mathrm{kHz}$ before data storage.

For almost all measurements performed at \texttt{LHC}, one crucial ingredient is the precise knowledge about the integrated luminosity. 
Despite being  a key parameter in any particle collider, the task of calibrating its absolute scale has been proven particularly challenging at hadron colliders. 
The determination and precision of the integrated luminosity have direct implications on cross section measurements, and its instantaneous measurement 
gives essential feedback on the conditions at the experimental insertions and the accelerator performance. 
To determine the absolute luminosity dedicated beam-separation techniques are used, the so-called ``van der Meer scans.''
With the exception of the reference proton and heavy ion runs, these scans are not typically performed during normal physics operation,
but under carefully controlled conditions and with beam parameters tailored to achieve the desired precision of $\mathcal{O}$(2--4\%). 
More recently, the advent of vertex-based ``beam-imaging'' techniques opened up interesting perspectives.

Soon after the discovery of the bottom quark~\cite{bottomobs}, the quest for the top quark had begun. 
The search carried out for nearly 20 years because the mass of the top quark turned out to be unexpectedly large, around 40 times the mass of the bottom quark. 
The first study from the \texttt{CDF} Collaboration~\cite{topevidence_CDF_prl,topevidence_CDF_prd} included the results from the 1992-1993 run at the Fermilab Tevatron Collider, 
firmly established the existence of the top quark (Fig.~\ref{topmass_evolution:a}). 
Simultaneous reports from \texttt{CDF}~\cite{topobs_CDF} and \texttt{D{\O}}~\cite{topobs_D0} Collaborations later provided sufficient statistical significance 
to definitively claim its observation from the 1994--1995 run. 
The discovery of the top quark in nuclear collisions had to further wait another 20 years for the 2016 proton-nucleus run at the \texttt{CERN} \texttt{LHC}, 
as illustrated in Fig.~\ref{topmass_evolution:b} and described in the following.
 Of the heavy particles expected in the standard model, only the $\tau$ lepton and the Higgs boson have not yet been detected in nuclear interactions.

\begin{figure}[!ht]
\begin{minipage}{.5\linewidth}
\centering
\subfloat[]{\label{topmass_evolution:a}\includegraphics[width=0.8\textwidth]{figures/topmass_PhysRevD_50_2966.png}}
\end{minipage}
\begin{minipage}{.5\linewidth}
\centering
\subfloat[]{\label{topmass_evolution:b}\includegraphics[width=0.8\textwidth]{figures/mthad_comb_singlecat_1l4j2b_nopull.png}}
\end{minipage}
\caption[The first top quark mass reconstruction in $\textrm{p\overline{p}}$ and pPb collisions]{ 
The top quark mass as first reconstructed in samples of (a) proton-antiproton~\cite{topevidence_CDF_prl,topevidence_CDF_prd} and (b) proton-nucleus~\citeTH{topobs_CMS_prl} collisions
 at 1.8 and 8.16\,\TeV, corresponding to integrated luminosities of 19.3\,\invpb and 174\,\invnb~\citeTH{CMS-PAS-LUM-17-002}~(equivalent to 36\,\invpb of nucleon-nucleon collision data), respectively. 

}
\label{fig:topmass_evolution}
\end{figure}

At hadron colliders the higher the center-of-mass energy the more top quarks are produced in pairs ($\mathrm{t\overline{t}}$). 
The large top quark mass uniquely provides a hard scale for the associated cross section ($\sigma_{\mathrm{t\overline{t}}}$), 
a quantum chromodynamics (\QCD) process determined with high accuracy based on perturbation expansions.
Using the data sample of $27.4 \pm 0.6$\,\invpb~\citeTH{CMS-PAS-LUM-16-001} collected by the \texttt{CMS} experiment during the proton-proton run at 5.02\,\TeV\ in 2015, 
the first measurement of the inclusive $\mathrm{t\overline{t}}$ cross section is presented for events with one or two high-\pt leptons (electrons or muons), 
and at least two jets. 
The measurement is separately performed in four final states, i.e., using $\ell$+jets ($\ell=\mathrm{e},\mu$) and dilepton ($\mathrm{e}^\pm \mu^\mp$ and $\mu^\pm \mu^\mp$) events, 
and is then obtained from the combination of the individual measurements.  The result is  $\sigma_{\mathrm{t\overline{t}}} =  69.5 \pm 6.1\,(\textrm{stat}) \pm 5.6\,(\textrm{syst}) \pm 1.6\,(\textrm{lumi})\,\pb$, 
with a total relative uncertainty of 12\%~\citeTH{topobs_CMS_jhep},  which  is  consistent with the standard model prediction (Fig.~\ref{fig:top_xsections}). 

Measurements of $\sigma_{\mathrm{t\overline{t}}}$ at various center-of-mass energies ($\sqrt{s}$) probe different values of $x$, 
the fractional momentum of the proton carried by the partons, and thus can provide complementary information on the parton distribution functions (PDFs). 
The  impact  of  the  measured cross  sections  in  the  determination  of  the  proton PDFs is studied in a \QCD\ analysis at next-to-next-to-leading order. 
A moderate decrease of the uncertainty in the gluon distribution is observed in the less-explored kinematic range of $x \gtrsim 0.1$, 
consistent with the expectation of the high-$x$ region being probed at $\sqrt{s}=5.02$\,\TeV. 
Future measurements of $\sigma_{\mathrm{t\overline{t}}}$ in nucleus-nucleus collisions at the same nucleon-nucleon center-of-mass energy ($\sqrt{\smash[b]{s_{_{\mathrm{NN}}}}}$) 
would profit from the availability of such a reference measurement, without the need to extrapolate from measurements at different $\sqrt{\smash[b]{s_{_{\mathrm{NN}}}}}$.  

\begin{figure}[!ht]
\centering
\includegraphics[width=0.8\columnwidth]{figures/CMS-HIN-17-002_Figure-aux_009.pdf}
\caption[Top quark pair production cross section in pp and pPb collisions at different center-of-mass energies]{\label{fig:top_xsections} 
    Top quark pair production cross section in proton-nucleus~\citeTH{topobs_CMS_prl} and proton-proton~\citeTH{topobs_CMS_jhep} collisions as function of the center-of-mass energy; 
    the measurements from the \texttt{CMS} Collaboration in the dilepton and $\ell$+jets final states are compared to the NNLO+NNLL theory predictions employing state-of-the-art proton and nuclear PDFs. 
    The total experimental error bars (theoretical error bands) include statistical and systematic (PDF and scale) uncertainties added in quadrature. 
}
\end{figure}

The feasibility of top quark measurements in nuclear collisions is demonstrated with the first observation of the $\mathrm{t\overline{t}}$ process~\citeTH{topobs_CMS_prl},
using $174 \pm 6$\,\invnb~\citeTH{CMS-PAS-LUM-17-002} of proton-nucleus collisions at the higher $\sqrt{\smash[b]{s_{_{\mathrm{NN}}}}}=8.16$\,\TeV.
The measurement is performed by analyzing events with exactly one isolated electron or muon and at least four jets. 
The inclusive cross section that is simultaneously measured in the two final states is $\sigma_{\mathrm{t\overline{t}}}=45 \pm 8\,(\textrm{total})\,\nb$, 
consistent with perturbative \QCD\ calculations---employing state-of-the-art nuclear PDFs---as well as the expectations from scaled proton-proton data (Fig.~\ref{fig:top_xsections}).
The statistical significance of the $\mathrm{t\overline{t}}$  signal against the background-only hypothesis is above five standard deviations. 
This first measurement paves the way for further detailed investigations of top quark production in nuclear interactions, providing, in particular, a new tool for studies 
of the strongly interacting matter created in nucleus-nucleus collisions.

With a total of only about eight weeks of operational experience, the luminosity with asymmetric proton-nucleus collisions surpassed the design value, 
and the threshold could have been pushed even further 
%crossed the intensity barrier% 
if the luminosity were not restrained for machine protection reasons. 
The long-term integrated luminosity goal of 100\,\nb\ was clearly surpassed in some experiments, albeit it represented the last proton-nucleus run for several years. 

When proton and lead collide with each other again, one can reasonably expect to manage the luminosity debris better, to further increase the proton bunch intensity, 
and to deliver a few times more integrated luminosity than in 2016 within a similar time frame. 
At the long interval before the next run,  the number and intensity of lead bunches meant for nucleus-nucleus collisions should increase substantially.
Meanwhile, the excellent performance achieved in the 2018 PbPb run brought \texttt{LHC} one step closer to its high-luminosity era~\cite{HI_NewPhys}.
A series of improvements both in \texttt{LHC} and its injector chain, including an increase in the average colliding bunch intensity and a decrease in the nominal bunch spacing,
resulted in reaching about six times higher the instantaneous luminosity than the design value of $1 \times 10^{27}$\,$\mathrm{cm}^{-2}\mathrm{s}^{-1}$.

\end{preface}